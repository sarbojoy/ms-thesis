\addcontentsline{toc}{chapter}{Conclusions}

\hspace{1.5em}The work presented here covers enormous literature and concepts vital for the evolution of non-Hermitian quantum mechanics. Our goal was to understand the physicality of non-Hermitian Hamiltonians in standard quantum mechanics. To build a framework consistent with the axioms of quantum mechanics, we saw that conventional quantum theory is not suffcient, and a different approach was required. $\mathcal{PT}$-symmetry allowed us to descibe the energy spectrum of a non-Hermitian Hamiltonian, dividing it into regimes of \textit{spontaneously broken} and \textit{unbroken} $\mathcal{PT}$\textit{-symmetry}. But this was again not enough to understand non-Hermitian Hamiltonians that were not $PT$ invariant,  and yet had the whole or a part of their spectra to be real. Pseudo-Hermitian quantum mechanics was introduced to counter these shortcomings of $\mathcal{PT}$-symmetry. Results and predictions of pseudo-Hermitian formalism subsumed the work of $\mathcal{PT}$-symmetry and allowed for a more broader formalism which descibed all time-independent non-Hermitian systems. \par

\subsection*{Future Plans}

The work we have presented to the reader can be followed by an investigation oftime-dependent Hamiltonians and linear automorphisms $H(t)$, and $\eta(t)$ respectively. Applications of time-dependent non-Hermitian Hamiltonian systems can prove to be indispensable in terms of solutions of Schr\"{o}dinger equations for systems that were not conceivable before. \par

We have seen transitions in the energy spectra for different non-Hermitian Hamiltonians with $\mathcal{PT}$-symmetry. Therefore, moving from an unbroken $\mathcal{PT}$-symmetric regime to the broken one can have far-reaching consequences in quantum many-body physics. Especially in the context of codensed matter physics to realise $\mathcal{PT}$-symmetry in experiments. \par

Pseudo-Hermitian and $\mathcal{PT}$-symmteric quantum mechanics can be investigated within the path-integral formulation. Relativistic Pseudo-Hermitian quantum mechanics would also be an intriguing sequel to the work on. But as we have seen throughout this thesis, in the static scenario itself, the theory of non-Hermitian quantum mechanics proves to be extremely rich and interesting.  