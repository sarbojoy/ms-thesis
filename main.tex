\documentclass[a4paper,12pt]{report}

%\documentclass[phd]{urithesis}
\usepackage[left=2.5cm,right=2.5cm,top=2.5cm,bottom=2.5cm,marginparwidth=20mm]{geometry}
%\pagestyle{myheadings}
\usepackage{pdfpages}
\usepackage[english]{babel}
\usepackage[utf8x]{inputenc}
\usepackage[colorlinks=true,linkcolor=blue,anchorcolor=black,citecolor=red,filecolor=black,menucolor=black,runcolor=black,urlcolor=green]{hyperref}
\setlength{\parindent}{1.5em}
\setlength{\parskip}{1em}
\renewcommand{\baselinestretch}{1.5}
\usepackage[font=footnotesize,labelfont=bf]{caption}
\usepackage{tcolorbox}
\tcbuselibrary{theorems}


\usepackage{longtable}
\usepackage{epigraph}

\usepackage{mathrsfs}
\usepackage{array}
\usepackage{tocloft}
\usepackage{tabularx}
\usepackage{amsmath}
\allowdisplaybreaks
\usepackage{enumitem}
\usepackage{xcolor}
\usepackage{authblk}
\usepackage{braket}
\usepackage{amsthm}
\usepackage{thmtools}
\usepackage{{hyperref}}
\usepackage{bm}
\usepackage{scrextend}
\usepackage{amssymb}
\usepackage{physics}
\usepackage{gensymb}

\usepackage{graphicx}
\graphicspath{ {images/} }
\usepackage{caption}
\usepackage{subcaption}
\usepackage{ulem} % Enables sout command

\usepackage{fancyhdr}
\pagestyle{fancy}


\renewcommand{\chaptermark}[1]{%
\markboth{#1}{}}  

\renewcommand{\qedsymbol}{$\blacksquare$}

%---------------latexdiff type environments--------------------
\newcommand{\msout}[1]{\Red{\text{\sout{\ensuremath{#1}}}}} % strikethrough equations via \msout command
\newcommand{\tsout}[1]{\Red{\sout{#1}}} % strikethrough texts via \tsout command
\newcommand{\muwave}[1]{\Blue{\text{\uwave{\ensuremath{#1}}}}} % weavy underline of maths via \muwave command
\newcommand{\tuwave}[1]{\Blue{\uwave{#1}}} % weavy underline of texts via \tuwave command
\newcommand{\txout}[1]{\Red{\xout{#1}}} % Remove texts via \txout command
\newcommand{\mxout}[1]{\Red{\text{\xout{\ensuremath{#1}}}}} % Remove maths via \mxout command
\newcommand{\margincomment}[2]{\marginpar{\textcolor{violet}{#1}}\textcolor{violet}{#2}} % comment on the margin by using  \margincomment{}{} command
%---------------colours---------------------------------------
\newcommand{\Red}[1]{\textcolor{red}{#1}}
\newcommand{\Blue}[1]{\textcolor{blue}{#1}}

\fancyhead{}
\fancyhead[R]{\leftmark}
\fancyhead[L]{Chapter \thechapter}
\fancyfoot[C]{\thepage}

\renewcommand{\headrulewidth}{0.4pt}
\setlength\cftafterloftitleskip{20mm}
\setlength\cftaftertoctitleskip{15mm}

\declaretheorem[]{definition}
\declaretheorem[]{theorem}
\declaretheorem[]{corollary}

\deffootnote[1em]{0em}{1em}{\textsuperscript{\makebox[1em][l]{\thefootnotemark}}}

\newcounter{storesubequations}

\begin{document}


%%%%%%%%%%%%%%%%%%%%%%%%%%%%%%%%%%%%%%%%%%%%%%%%%%%%%%%%%%%%%%%%%%%%%%%%%%%%

\begin{titlepage}
   \begin{center}
       \vspace*{1cm}

       \textbf{\Huge{\textbf{Non-Hermitian Quantum Mechanics: $\mathcal{PT}$-symmetry versus pseudo-Hermiticity}}}

       \vspace{2.0cm}

       \textbf{Sarbojoy Das}

       \vspace{3.0cm}
            
       \textit{A dissertation submitted for the partial fulfilment of BS-MS dual degree in Science}
            
       \vspace{2.0cm}
     
       \includegraphics[width=0.4\textwidth]{images/instilogo.jpg}
       
       \vspace{1.0cm}
       \textbf{Indian Institute of Science Education And Research, Mohali}\\
       \textbf{December 2022}
            
   \end{center}
\end{titlepage}
\newpage

%%%%%%%%%%%%%%%%%%%%%%%%%%%%%%%%%%%%%%%%%%%%%%%%%%%%%%%%%%%%%%%%%%%%%%%%%%%%

\thispagestyle{empty}

\vspace*{30mm}

\begin{center}
    \textbf{\huge{Certificate of Examination}}
\end{center}

\vspace{20mm}

This is to certify that the dissertation titled \textbf{``Non-Hermitian Quantum Mechanics: $\mathcal{PT}$-symmetry versus pseudo-Hermiticity"} submitted by \textbf{Mr. Sarbojoy Das} (Reg. No. MS16139) for the partial fulfilment of BS-MS dual degree programme of the Institute, has been examined by the thesis committee duly appointed by the Institute. The committee finds the work done by the candidate satisfactory and recommends that the report be accepted. \\ 


\vspace{20mm}

\begin{center}
    Dr. Sanjib Dey \hspace{7mm} Dr. Manabendra Nath Bera \hspace{7mm} Dr. Sandeep Kumar Goyal
\end{center}




\newpage 

%%%%%%%%%%%%%%%%%%%%%%%%%%%%%%%%%%%%%%%%%%%%%%%%%%%%%%%%%%%%%%%%%%%%%%%%%%%

\thispagestyle{empty}

\vspace*{3mm}

\begin{center}
    \textbf{\huge{Declaration}}
\end{center}

\vspace{10mm}
The work presented in this dissertation has been carried out by me under the
guidance of Dr. Sanjib Dey at the Indian Institute of Science Education and
Research, Mohali. \\ 

This work has not been submitted in part or in full for a degree, a diploma, or a fellowship to any other university or institute. Whenever contributions of others are involved, every effort is made to indicate this clearly, with due acknowledgement of collaborative research and discussions. This thesis is a bona fide record of original work done by me and all sources listed within have been detailed in the bibliography. \\ \\ 

\begin{flushright}
    Sarbojoy Das \\
    (Candidate) \\ \vspace{5mm}
\end{flushright}
\hspace{12cm}Date:

\vspace{7mm}

In my capacity as the supervisor of the candidate’s project work, I certify that the above statements by the candidate are true to the best of my knowledge. \\  \\

\begin{flushright}
    Dr. Sanjib Dey\\
    (Supervisor)
\end{flushright}
\newpage 

%%%%%%%%%%%%%%%%%%%%%%%%%%%%%%%%%%%%%%%%%%%%%%%%%%%%%%%%%%%%%%%%%%%

%%%%%%%%
\thispagestyle{empty}

\begin{center}
    \textbf{\huge{Acknowledgements}}
\end{center}
\vspace{1cm}

The first person I would like to thank is my supervisor, Dr. Sanjib Dey, for his undeterred support and guidance throughout this work. Our discussions on physics, as well as life played a crucial role in seeing this dissertation to fruition. I am indebted to his invaluable inputs and critical comments of my work. \par 

I express my heartfelt gratitude to Dr. Yogesh Singh, and Dr. Abhishek Chaudhuri for pushing me to complete this work even when I doubted myself and was effectively lost. Their constant nudges were essential in keeping me motivated. \par 

My life at IISER Mohali for the past year would not have been complete without my friends who made me laugh even in the darkest of times. For them I would like revise Professor Albus Dumbledore's words from Harry Potter and the Deathly Hallows to say, ``Help would always be given at IISER Mohali, to those who deserve and cherish it. Thank you Vidhika, Sourav, Sandip, Sougata, Rajeshree, Abhimanyu, Antriksh, Ayan, Nilaj, Abhinandan, Bhavya, Sanat, Babai, Subhashree Di, Gaurav, Deepraj, Yuvraj, Rahul, Monu, Nikhil for all the cherishable memories. \par

I would not have come this far without the immense sacrifice and patience of my Maa, and Papa. Your believe in me and my passion for science allowed me to keep going even in times of despair. Thank you for all the unconditional love, and privileges that you have offered me and keep giving.   
\newpage 


%%%%%%%%%%%%%%%%%%%%%%%%%%%%%%%%%%%%%%%%%%%%%%%%%%%%%%%%%%%%%%%%%%%%%%%%%%%
\thispagestyle{empty}

\vspace*{20mm}

\listoffigures

\pagenumbering{roman}

\addcontentsline{toc}{chapter}{List of Figures}

\newpage 

%%%%%%%%%%%%%%%%%%%%%%%%%%%%%%%%%%%%%%%%%%%%%%%%%%%%%%%%%%%%%%%%%%%%%%%%%%%%%%%%


\thispagestyle{plain}
\addcontentsline{toc}{chapter}{Abstract}

\vspace*{1.5cm}

\begin{center}
    \textbf{\huge{Abstract}}
\end{center}
\vspace{15mm}
The development of quantum mechanics has been phenomenal since the beginning of the 20th century. But over time physicists realised that the standard Hermitian framework originally fomalised by Paul Dirac is inadequate and not inclusive toward non-Hermitian operators. Although it has been shown, through rigorous mathematical and well-founded physical principles, that non-Hermitian operators often have a real spectrum. \par

This dissertation tries to outline the major developments in the field of non-Hermitian quantun mechanics. In \textbf{Chapter 1} we explore $\mathcal{PT}$-symmetry, which was initially successful in explaining and justifying the consequences of non-Hermitian Hamiltonians that are invariant under the application of parity-time operator. Later in the chapter, a symmetry operator $\mathcal{C}$, similar to the charge conjugation operator in particle physics, is introduced to complete the theory of $\mathcal{PT}$-symmetric quantum mechanics. \par

\textbf{Chapter 2} introduces pseudo-Hermiticity, which subsumes $\mathcal{PT}$-symmetric theory, and establishes a more general framework. Pseudo-Hermitian quantum mechanics is the natural approach in understanding non-Hermitian systems. We briefly touch upon quasi-Hermitian operator theory, an important early development in the field. And finally end with our conclusions and remarks.  


\newpage
%%%%%%%%%%%%%%%%%%%%%%%%%%%%%%%%%%%%%%%%%%%%%%%%%%%%%%%%%%%%%%%%%%%%%%%%%%%%%%%%
\vspace*{15mm}

\fancyhf{}
\renewcommand{\headrulewidth}{0.0pt}
\fancyfoot[C]{\thepage}

\tableofcontents

\newpage


%%%%%%%%%%%%%%%%%%%%%%%%%%%%%%%%%%%%%%%%%%%%%%%%%%%%%%%%%%%%%%%%%%%%%%%%%%%%%%%%

\fancyhead{}
\fancyhead[R]{\leftmark}
\fancyhead[L]{Chapter \thechapter}
\fancyfoot[C]{\thepage}

\renewcommand{\headrulewidth}{0.4pt}
\setlength\cftafterloftitleskip{20mm}
\setlength\cftaftertoctitleskip{15mm}

\pagenumbering{arabic}

\chapter{Introduction to Non-Hermitian Quantum Mechanics}
\vspace{0.5mm}
\epigraph{\textbf{In a dark place we find ourselves, and a little more knowledge lights our way.}}{\textit{Yoda, Star Wars Episode III: Revenge Of The Sith}}
 \vspace{1cm}

\hspace{0.7cm} Quantum Theory has proved to be very successful in understanding the nature of matter and interactions at minor scales, yet over time we have realised that it is inadequate to explain all fundamental physical theories. The development of Quantum Mechanics was very haphazard during the first half of the 20\textsuperscript{th} century until Paul Adrien Maurice Dirac \textbf{(The Master!)} formulated quantum mechanics the way we know it today and can be found in any standard textbook on the subject. Like any other scientific venture, quantum mechanics is built on some fundamental axioms or assumptions founded on sound reasons. However, starting from 1990, many physicists, mostly of the mathematical kind, questioned the basis of these assumptions trying to find theories that were more general than the then-contemporary notions. Quite surprisingly, through the efforts of many (several of whom we will mention in this dissertation), a more generalised approach was eventually constructed. This approach takes the assumptions of the standard formulation as special cases of its own. Let us first understand the axioms of Quantum Mechanics that define and characterise the theory. Once certain mathematical and physical subtleties are stated, we will move on to the actual scope of this thesis, i.e. Non-Hermitian Quantum Mechanics. The essence of a quantum theory can be found using the Hamiltonian approach, and hence it is imperative that we study the properties of the Hamiltonian for a system. The exceptional physical axiom is that the energy spectrum of a quantum Hamiltonian must yield real results \textbf{(Axiom 1)}. After all, energy is a real quantity!\par



The other axioms that define a quantum theory are as follows : \par



\begin{table}[h]
    \centering
        \begin{tabular}{l  p{11.7cm}}
           \textbf{Axiom 2} \vline & The energy spectrum must be bounded below so that a stable lowest-energy state exists. As a consequence, we take the Hamiltonian of a system to be a bounded operator (bounded below) in the Hilbert Space. For a quick review of bounded operators please refer to Appendix A at the end. \\
           
           \\
           
           \textbf{Axiom 3} \vline & Time evolution of a quantum system must be unitary or probability conserving. \\
           
           \\
           
           \textbf{Axiom 4} \vline & A time evolving system must accord with causality and Lorentz covariance.  \\
           
           \\
           
           \textbf{Axiom 5} \vline & The Hamiltonian operator that describes the dynamics of the system must be Hermitian. It guarantees that the energy spectrum is real and time evolution is unitary. Also called the \textit{\textbf{Dirac Hermiticity}}, \textbf{this axiom is Mathematical, and it is more of a convenience than a necessity !} \textbf{[} $\mathbf{H} = \mathbf{H^\dagger}$ \textbf{]} 
           
        \end{tabular}
        
\end{table}
\par

The breakthroughs in non-Hermitian quantum physics have shown that \textbf{Axiom 5} has certain aspects that require more investigation. The work started by Bender\cite{PhysRevLett.80.5243,Bender_2007}, Mostafazadeh\cite{doi:10.1063/1.1418246,mostafazadeh2010pseudo} and many others has conclusively shown that \textbf{Axiom 5} is a sufficient condition, but it is \textbf{\textit{not necessary}}. The crux of non-Hermitian quantum mechanics lies in this very fact. Although, at first glance, the statement seems quite simple, it has far-reaching consequences and has led to the discovery of various theoretical and experimental advances. A first step towards this journey started with the idea of developing a non-Hermitian framework not only in terms of the Hamiltonian but also in terms of what consequent changes it would have on the metric of the system. An early work in 1991 by Scholtz \textit{et al}\cite{SCHOLTZ199274} provided the idea of Quasi-Hermitian Operators and the way we can construct a metric or inner product for a set of non-Hermitian operators such that they are Hermitian with respect to the redefined metric. Another vital observation in non-Hermitian Hamiltonians was pointed out by Bessis in a private communication to Carl M. Bender\cite{PhysRevLett.80.5243} that certain Hamiltonians violating Dirac Hermiticity showed a \textit{real and positive spectrum} under properly defined boundary conditions. It was shown on the basis of numerical studies that the spectrum of 
\begin{align}
     \label{rsnhh}
     &\nonumber\\
     H = p^2 + x^2 + ix^3, \\
     &\nonumber
\end{align}
is real and positive, but it is not Hermitian \cite{PhysRevLett.80.5243}. The reason for such results was attributed to \textbf{parity-time symmetry or} $\mathbf{\mathcal{PT}}$\textbf{-symmetry}. 


\section{Basic principles of $\mathcal{PT}$-symmetric Quantum Mechanics}

In $\mathcal{PT}$-symmetry the Dirac Hermiticity requirement or \textbf{Axiom 5} is replaced with a more natural and physically transparent condition of parity-time symmetry, 
\begin{align}
    \label{ptsymm}
    &\nonumber\\
    H = H^{ \, \mathcal{ PT}}, \\
    &\nonumber
\end{align} 
without violating any of the axioms mentioned above. A good counterexample would be to look at the Hamiltonian 
\vspace{-1cm}
\begin{align}
    \label{rsnhwpts}
    H = p^2 + ix^3 + x, \\
    &\nonumber 
\end{align}
which has complex spectra and is not invariant under parity-time symmetry (does not follow Eq.(\ref{ptsymm})). Later in this Introduction, we will see that the $\mathcal{PT}$-symmetry has a \textit{complex extendion}, $\mathcal{CPT}$-symmetry. But the most surprising discovery will be introduced in Chapter 2, where we will understand that a larger framework called \textit{Pseudo-Hermiticity} is at play, which gives a more rigorous formalism responsible for the reality of the spectrum. We need to understand the workings of the \textit{Parity} and \textit{Time Reversal} operators before going any further. \par

The parity or space-reflection operator affects the position operator $x$ and the momentum operator $p$ to change signs.
\begin{align}
    \label{parity on x and p}
    &\nonumber\\
    \mathcal{P} \, x \, \mathcal{P} \, = \, -x \quad \text { and } \quad \mathcal{P} \, p \, \mathcal{P} \, = \, -p. \\
    &\nonumber
\end{align}
$\mathcal{P}$ is a linear operator, and hence the Heisenberg Algebra is left invariant under its action,
\begin{align}
    \label{piha}
    &\nonumber\\
    \mathcal{P}\,[x p-p x] \, = \, (-x)(-p)-(-p)(-x) \, = \, x p-p x \, = \, i \, \hbar \, \mathbf{1}. \\
    &\nonumber
\end{align}
The time-reversal operator acts on position $x$ and momentum $p$ to give: 
\begin{align}
    \label{time reversal on x and p}
    &\nonumber\\
    \mathcal{T} \, x \, \mathcal{T} \, = \, x, \quad \text { and } \quad \mathcal{T} \, p \, \mathcal{T} \, = \, -p. \\
    &\nonumber
\end{align}
The time-reversal operator is fundamental to nature and hence must leave the Heisenberg algebra invariant. But there is a slight problem, 
\begin{align}
    \label{trha}
    &\nonumber\\
    \mathcal{T} \, [x p-p x] \, = \, (x)(-p)-(-p)(x) \, = \, px-xp \, = \, -i \, \hbar \, \mathbf{1}. 
\end{align}
To make this consistent $\mathcal{T}$ must reverse the sign of complex $i$:
\begin{align}
    \label{trc}
    &\nonumber\\
    \mathcal{T} \, i \, \mathcal{T} \, = \, -i. \\ 
    &\nonumber
\end{align}
$\mathcal{T}$ is \textit{antilinear} because a map $A: V \rightarrow W$, where $V$ and $W$ are complex vector spaces and $c$ a complex scalar, is antilinear if,
\begin{align}
    \label{antiliniear definition}
    &\nonumber\\
    A (c \alpha + \beta) \, = \, c^{\, *} A \alpha + A \beta \quad \quad \quad \text{where} \quad \alpha,\beta \in V. \\
    &\nonumber
\end{align}
$c^{*}$ is the complex conjugate of c. \textbf{[Note that, the action of a linear operator would have kept $c$ unchanged]}. Both $\mathcal{P}$ and $\mathcal{T}$ are reflection symmetry operators; hence, acting on them twice, i.e. their squares, must be identity,
\begin{align}
    \label{squares of P and T}
    &\nonumber\\
    \mathcal{P}^{\, 2} \, = \, \mathcal{T}^{\, 2} \, = \, \mathbf{1}. \\
    &\nonumber
\end{align}\par
Space-reflection and time-reversal can be measured to high accuracy simultaneously without disturbing the other. That is how they are interpreted in the quantum mechanical scaffolding. Hence, they must commute:
\begin{align}
    \label{P and T commute}
    &\nonumber\\
    [\mathcal{P},\mathcal{T}] \, = \, 0. \\
    &\nonumber
\end{align}
The consequence of the above properties also makes the parity-time operator, $\mathcal{PT}$, self-invertible. From Eq.(\ref{P and T commute}) we get,
\begin{align}
    \label{PT is invertible}
    &\nonumber\\
    &\mathcal{PT} - \mathcal{TP} = 0 \nonumber\\
    \implies & \mathcal{PT} - (\mathcal{P}^{-1}\mathcal{T}^{-1})^{-1} = 0 \nonumber\\
    \implies \mathcal{PT} - (\mathcal{PT})^{-1} = 0 \qquad & \text{or} \qquad \mathcal{PT} = (\mathcal{PT})^{-1}  &&\text{[Using Eq.(\ref{squares of P and T})]}.  \\
    &\nonumber
\end{align}
Eq.(\ref{PT is invertible}) helps us define the $\mathcal{PT}$-transformation of an operator,
\begin{align}
    \label{PT reflected hamiltonian}
    &\nonumber\\
    H^{\mathcal{PT}} \equiv (\mathcal{PT})^{-1}H(\mathcal{PT}) = (\mathcal{PT})H(\mathcal{PT}) &&\text{[Using Eq.\ref{PT is invertible}]}. \\
    &\nonumber
\end{align}
This allows us to properly categorise the $\mathcal{PT}$-symmetric Hamiltonians as follows.
\begin{align}
    \label{PT symmetric hamiltonian}
    &\nonumber\\
    H \, = \, H^{\mathcal{PT}} \quad \text{or} \quad H \, = \, (\mathcal{PT}) \, H \, (\mathcal{PT}). \\
    &\nonumber
\end{align}\par
The definition in Eq.(\ref{PT symmetric hamiltonian}) allows for a more general yet consistent framework of quantum mechanics. It has been seen that Hamiltonian, $H$, that follows Eq.(\ref{PT symmetric hamiltonian}), surprisingly have real eigenvalues and specifies a unitary-time evolution even though it is not necessary for $H$ to be Hermitian. \par

Examples of Hamiltonians that are non-Hermitian but adhere to Eq.(\ref{PT symmetric hamiltonian}),
\begin{align}
    \label{PT symm H1}
    &\nonumber\\
    H \, = \,  p^2 + ix^3, \\
    &\nonumber
\end{align}
and
\begin{align}
    \label{PT symm H2}
    &\nonumber\\
    H \, = \, p^2 - x^4. \\
    \nonumber
\end{align}
Eq.(\ref{PT symm H1}) and Eq.(\ref{PT symm H2}) are part of a general parametric family of Hamiltonians that conform to Eq.(\ref{PT symmetric hamiltonian}). This family has been studied extensively, and in the coming sections we will explore the deep insights that are gained by investigating it. $\mathbf{\epsilon}$ parameterises this group as
\begin{align}
    \label{Gen Parametric Family of PT symm H}
    &\nonumber\\
    H \, = \, p^2 + x^2(ix)^{\, \epsilon}. \\
    \nonumber
\end{align}\par
Eq.(\ref{Gen Parametric Family of PT symm H}) are complex extensions of the harmonic oscillator Hamiltonian. Bender and Boettcher \cite{PhysRevLett.80.5243} showed that, when $\epsilon \geq 0$ the eigenvalues of Eq.(\ref{Gen Parametric Family of PT symm H}) are \textit{real}, \textit{discrete}, and \textit{positive}. When $\epsilon < 0$ the eigenvalues become complex and the adjacent energy levels pair up to give conjugate pairs. This property is crucial for understanding $\mathcal{PT}$-symmetric quantum mechanics and will be presented in more detail as we progress in this chapter.

\section{Why Non-Hermitian Quantum Mechanics ?}

We look at the answers to this question via a dyadic approach: First, we will recognise the theoretical reinforcements that provide excellent reasons for conducting this investigation. Second, we give ample experimental evidence that supports the broad ideas of non-Hermitian quantum theory.

\subsection{Theoretical Evidence}

Symmetry plays a vital role in all areas of physics, and investigators often stumble upon significant results just by scrutinising symmetries of the system. In the case of $\mathcal{PT}$-symmetry, the homogeneous Lorentz groups of rotations and boosts are indispensable because both $\mathcal{P}$ and $\mathcal{T}$ are parts of it. Non-Hermitian Hamiltonians are quite common in the study of dissipative processes, such as radioactivity. However, these Hamiltonians are phenomenological descriptions and cannot be taken as fundamental interpretations of nature. Radioactive decay modelled using non-Hermitian Hamiltonians is \textit{non-unitary} and hence, \textit{non-fundamental} narratives.
\\
\\
The question is then: \textbf{Why is this investigation important, and is it fundamental?}
\par
The answer is that non-Hermitian processes are fundamental in nature. The Lorentz group has four parts that are disconnected: 

\begin{itemize}
    \item[1.] \textbf{The proper orthochronous sub-group :} It is continuously connected in the group space through identity.
    \item[2.] $\mathbf{\mathcal{P}}$ multiplied \textbf{to 1. :} Here, the elements are the product of \textit{proper orthochronous group} and $\mathcal{P}$. \textbf{\textit{Does not contain $\mathbf{1}$}}.
    \item[3.] $\mathbf{\mathcal{T}}$ multiplied \textbf{to 1. :} Here, the elements are the product of \textit{proper orthochronous group} and $\mathcal{T}$. \textbf{\textit{Does not contain $\mathbf{1}$}}.
    \item[4.] $\mathbf{\mathcal{PT}}$ multiplied \textbf{to 1. :} Here, the elements are the product of \textit{proper orthochronous group} and $\mathcal{PT}$. \textbf{\textit{Does not contain $\mathbf{1}$}}.
\end{itemize}\par
Since the last 3 parts do not contain the identity, we cannot consider them as sub-groups. Hence, all four parts are not connected continuously. Physics is Lorentz invariant under \textit{proper orthochronous sub-group} (Part 1.) of the total homogeneous Lorentz group. We know from experiments with the weak nuclear forces that they do not respect parity and time-reversal symmetry. The question of parity violation in weak nuclear forces and insufficient experimental evidence for the same was first pointed out by Lee and Yang \cite{lee1956question} in 1956. This observation was first confirmed by Wu \textit{et al} \cite{PhysRev.105.1413} in 1957 and paved the way for the creation of the standard model in particle physics. Time-reversal violations are observed in quantum entangled $B$ mesons \cite{RevModPhys.87.165}. There is a very interesting and informative \href{https://www.youtube.com/watch?v=yArprk0q9eE&t=92s}{\underline{\textit{YouTube video}}} on this topic of symmetry violation by Veritasium. A \textit{complex} Lorentz group can be constructed \cite[see Chpater 1]{StreaterandWightman} but it assumes that the eigenvalues of the Hamiltonian are real and bounded below (check Appendix \ref{Boundedness of a linear operator} for boundedness). In this complex group, there are two separate continuous paths: (i) The proper orthochronous group (Part 1 above) is continuously connected to the $\mathcal{PT}$ multiplied portion (Part 4 above). (ii) While the other two portions (Part 2 \& 3 above) are continuously connected in the complex Lorentz group space. The properties we see here in the study of complex Lorentz groups allow us to make a bold claim $\mathbf{\longrightarrow}$ \textbf{Physics is invariant under} $\mathcal{P}$ \textbf{and} $\mathcal{T}$ \textbf{as well as their proper orthochronous multiplied parts are connected continuously in the complex Lorentz group space. Now, physics is invariant under the proper orthochronous subgroup, and in the complex Lorentz group space it is continuously connected to} $\mathcal{PT}$\textbf{-symmetric multiplied to it part. Hence, we can argue that} $\mathcal{PT}$\textbf{-symmetry is a fundamental discrete symmetry of nature.} 
\par
The above claim has the consequence that new kinds of Hamiltonians can define allowable quantum theories and many Haamiltonian that we were considered to be unphysical can be brought back into the picture. This is the most thoroughgoing theoretical argument that one can make to justify the exploration of non-Hermitian quantum mechanics. 

\subsection{Experimental Evidence}
\label{Experimental Evidence}
%%begin novalidate
Non-Hermitian Hamiltonians that are $\mathcal{PT}$-symmetric have indirect experimental consequences. My study did not involve unswerving evidence of non-Hermitian Hamiltonians that occur in nature, but a recent \href{https://scholar.google.com/scholar?hl=en&as_sdt=0%2C5&q=physical+evidence+of+pt+symmetry&btnG=}{Google Scholar search} churns out tons of paper that study $\mathcal{PT}$-symmetric systems in experiments. In my work I did encounter a few experiments that survey experimental aspects of the subject.   
%%end novalidate
\par
Non-Hermitian Hamiltonians have been used to study delocalisation transitions in superconductors, such as vortex depinning \cite{PhysRevLett.77.570} or represent population biology models \cite{PhysRevE.58.1383}. Both these papers by Nelson \textit{et al} show remarkable similarity to theoretical predictions. Let us look at depinning of vortex flux lines in type-II superconductors induced by an imaginary external field rendering the Hamiltonian to be non-Hermitian \cite{PhysRevLett.77.570}. Experimental results will help us to build our theory of $\mathcal{PT}$-symmetric quantum mechanics in a way that we can corroborate our abstract propositions to real-world attestation. Non-Hermitian quantum theory has been used in Quantum Field Theory \cite{PhysRevLett.54.1354,PhysRevLett.40.1610,CARDY1989275,BROWER1978213,HARMS1980392,PhysRevLett.93.251601}, Complex crystals\cite{M_V_Berry_1998,khare2005pt,BENDER1999272,KHARE2004406,Khare_2006,khare2006complex}, solving Schr\"{o}dinger equations for complicated systems\cite{PhysRevA.95.010102,FRING20172318,PhysRevD.90.084005}, and especially in condensed matter systems involving superconductors\cite{PhysRevLett.77.570,PhysRevB.48.13060,PhysRevB.56.8651,PhysRevB.48.1167}. We look here at one such experiment and its observations, which will help us to know if we are going in the right direction. \par

\subsubsection{Localization Transitions in a Cylindrical Superconducting Shell}

In this experiment \cite{PhysRevLett.77.570} a cylindrical superconductor with columnar defects is subjected to a constant imaginary vector potential. The flux lines are produced as a result of a strong magnetic field along the vertical axis of the superconductor. A current along this axis generates an imaginary vector potential that depins the created flux lines from columnar defects. A pictorial depiction of the experiment is shown in Fig. \ref{fig:superconductor}.
\\

\begin{figure}[h]
    \centering
    \frame{\includegraphics[scale=0.4]{images/Localization Transitions In Non-Hermitian Quantum Mechanics.jpg}}
    \caption[Depinning of vortex flux lines in a superconducting shell subjected to a perpendicular imaginary vector potential]{A superconducting shell with radial thickness less than the penetration depth of the defect-free material with columnar defects. The scribbled red streak is the flux line induced by the constant magnetic field $\bm{H_{Z}}$. Flux lines interact with the blue bordered columnar defects. The imaginary vector potential field $\bm{H_{\perp}}$ is induced by the current $\bm{I}$.} 
    \label{fig:superconductor}
\end{figure}
\par
A non-Hermitian quantum Hamiltonian with randomness is introduced in the system to study its consequences. The experiment observes the mapping between flux lines of ($d+1$)-dimensional superconductors to the world lines of $d$-dimensional bosons. Flux lines such as the scribbled red track in Fig. \ref{fig:superconductor} are pinned by the experimentally introduced columnar defects (blue bordered columns in Fig. \ref{fig:superconductor}) \cite[see e.g.]{PhysRevLett.67.648}. These defects, in turn, give rise to a random potential in the boson system \cite{PhysRevB.48.13060}. $\bm{H_z}$ is applied vertically giving a chemical potential, while $H_\perp$ acts as an imaginary vector potential for the bosons. As the current $\bm{I}$ increases, the perpendicular magnetic field, $H_\perp$, also increases as a result of the depinning of the flux lines from columnar defects. You can see in Fig. \ref{fig:superconductor} the red streak moves from one column to another. These paths can be theoretically calculated using path-integral methods. Here, an imaginary potential, $\bm{H_\perp}$ (responsible for the non-Hermticity), depins the flux lines and hence must have extended states for a large zone where a delocalisation transition point exists at a critical strength of $\bm{H_\perp}$. The non-Hermitian Hamiltonian that describes this system is,
\begin{align}
    \label{Superconductor NH Hamiltonian}
    & \nonumber\\
    &\mathcal{H} = \frac{(\bm{p} + i\bm{h})^2}{2m} + V(\bm{x}) , \qquad \quad \text{$V(\bm{x})$ is a \textbf{random potential}.}\\
    & \nonumber
\end{align}
\hspace{1.5em} Here, the non-Hermitian part of the Hamiltonian is observed in the transverse magnetic field: $\bm{h} = \phi_{0}\bm{H_\perp}/4\pi$. $\phi_{0}$ is the charge, while the appearance of the complex $i$ is the result of mapping onto imaginary time quantum mechanics. The vortex (red streak) in Fig. \ref{fig:superconductor} is described by Eq.(\ref{Superconductor NH Hamiltonian}) with some periodic boundary conditions. The penetration depth of the material (defect free) is smaller than the radial thickness, with the temperature described by the Planck constant $\hbar$. The position of the flux line for a distance $\tau$ from the bottom surface (see Fig. \ref{fig:superconductor}) is given by
\begin{align}
    \label{flux line distance}
    &\nonumber\\
    \langle x\rangle_{\tau} \equiv Z^{-1} \times \left\langle\psi^{f}\left|e^{-\left(L_{\tau}-\tau\right) \mathcal{H} / \hbar} \bm{x} e^{-\tau \mathcal{H} / \hbar}\right| \psi^{i}\right\rangle,& \quad \text{$\psi^i$ \& $\psi^f$ at $\tau = 0,L_\tau$, respectively.}\\
    &\nonumber\\
    \text{where,} \qquad Z \equiv\left\langle\psi^{f}\left|e^{-L_\tau \mathcal{H} / \hbar}\right| \psi^{i}\right\rangle.& \qquad \quad \text{[\textbf{Partition Function}]} \label{Partition Function}\\
    & \nonumber
\end{align}
A current operator is also defined, $\boldsymbol{J} \equiv-i\, \partial \mathcal{H}/ \partial \boldsymbol{h}=(\boldsymbol{p}+i \boldsymbol{h}) / m$. This current is related to the commutator of $\mathcal{H}$ and $\bm{x}$ as : $[\mathcal{H},\bm{x}] = -i\hbar\boldsymbol{J}$, which in turn gives $(\partial / \partial \tau) \langle \bm{x} \rangle_\tau =-i\langle I \rangle_{\tau}= \text{Im} \langle \bm{J} \rangle_\tau $. A total displacement of the flux between the bottom and top surfaces is calculated as $\langle\bm{x}\rangle_{L_\tau} - \langle\bm{x}\rangle_0 = \hbar(\partial/\partial\bm{h})\:\text{ln} Z = \text{Im} \int_{0}^{L_\tau}\langle L \rangle_\tau \: d\tau $. This imaginary quantity stipulates the delocalisation transition caused by the imaginary vector potential. \par

To solve this system we make a few assumptions: (i) Eigenfucntions, $\psi_n (\bm{x})$, are known for $\bm{h} = 0$, (ii) the corresponding eigenvalues $\varepsilon_n$ are also known. The ways of decoding the form of these eigenfunctions are taken as left and right (for small $\bm{h}$) $\rightarrow \psi\,_{n}^{R}\,(\bm{x} ; \bm{h})=e^{\bm{h} \cdot \bm{x} / \hbar}\, \psi_{n}(\bm{x} ; \bm{h} =0)$ and $\psi\,_{n}^{L}\,(\bm{x} ; \bm{h})=e^{-\bm{h} \cdot \bm{x} / \hbar}\, \psi_{n}^{*}(\bm{x} ; \bm{h} =0)$. The $\varepsilon_n$'s are unchanged under this ``imaginary" gauge transformation \cite{PhysRevB.48.1167}. The normalisation condition for the left and right wave functions is achieved for $|\bm{h}| < \hbar \kappa_n$, where $\kappa_n$ is the inverse localisation length of $\psi_n(\bm{x}; \bm{h}=0)$. The normalization condition, 
\begin{align}
    \label{wave fucntion superconductor}
    &\nonumber\\
    &\int d\,^d\bm{x} \ \psi\,_{n}^{R}(\bm{x}) \ \psi\,_{n}^{L}(\bm{x}) = 1, \\
    & \nonumber\\
    \text{which gives,}\hspace{1em} &\psi\,_{n}^{R}(\bm{x}) \ \simeq \ \sqrt{\frac{\left(2 \kappa_{n}\right)^{d}}{\Gamma(d) \Omega_d}} \ e^{\bm{h}\cdot\left(\bm{x}-\bm{x}_{n}\right)/\hbar-\kappa_{n}\left|\bm{x}-\bm{x}_{n}\right|}.\\
    & \nonumber
\end{align}
Here, $\bm{x}_n$ is the localisation centre of $\bm{h} = 0$ and $\Omega_d$ is the total solid angle of the $d$ dimensional space. The delocalisation point occurs at $|\bm{h}| = \hbar \kappa_n$, for $|\bm{h}| > \hbar \kappa_n$ extended eigenfunctions are obtained (this region will give us a deep insight in a moment!). \par

Imposing periodic boundary conditions: $\psi\,_{n}^{R}\left(L_{x} / 2, y, \ldots\right)=\psi\,_{n}^{R}\left(-L_{x} / 2, y, \ldots\right)$, with the $x$ axis parallel to $\bm{h}$ produces \textbf{interesting results} (there is a wave function mismatch at $x = \pm L_{x} / 2$ of order $e^{-\left(\kappa_{n}-h / \hbar\right) L_{x}}$), 

\begin{table}[h]
    \centering
        \begin{tabular}{l  p{11.95cm}}
        \textbf{(1)} \framebox{$h<\hbar \kappa_{n}$} $\ \longrightarrow$ & The mismatch is very small and hence, the change necessary to meet the periodic boundary conditions. \\ 
        
        \\
        
        \textbf{(2)} \framebox{$h \geq \hbar \kappa_{n}$} $ \ \longrightarrow$ & \textbf{Complex eigenvalue appears}, and there are substantial changes to the wave fucntion. \textbf{\textit{The Hamiltonian}} $\bm{\mathcal{H}}$ \textbf{\textit{if parameterized by}} $\bm{h}$\textbf{\textit{ (or}} $\bm{\kappa_n}$\textbf{\textit{) will give a family that produces complex eigenvalues after a point (or a transition).}}\\

        \end{tabular}
\end{table}    
\hspace{-1em}Understanding the result in \textbf{(2)} requires $|\bm{h}| \rightarrow \infty$, where we can ignore the random potential $V(\bm{x})$ \cite{PhysRevB.56.8651}. In this case, the periodic boundary is satisfied by the extended wave function, which is $e^{i \bm{k} \cdot \bm{x}}$. The wave number in the direction $x_\nu$ is given by $k_{\nu}=2 n_{\nu} \pi / L_{\nu}$, where $n_{\nu}$ is an integer, and $L_{\nu}$ is the size of the system. Then the left eigenvector $\longrightarrow e^{-i \bm{k} \cdot \bm{x}}$. Hence, the eigenvalue is given by
\begin{align}
    \label{complex eigen value}
    & \nonumber\\
    \varepsilon(\bm{k})\ = \ \frac{(\hbar \bm{k} \ + \ i \bm{h})^{2}}{2 m}. 
\end{align}
\textbf{When the random potential cannot be neglected : } A non-Hermitian tight-binding model is used to represent the system with a second quantisation Hamiltonian with the boson field operators, 
\begin{align}
    \label{tight-binding model 2nd quant hamiltonian}
    & \nonumber\\
    \mathcal{H} \equiv -\frac{t}{2} \sum_{\boldsymbol{x}} \sum_{\nu=1}^{d}\left(e^{\boldsymbol{h} \cdot \boldsymbol{e}_{\nu} / \hbar}\, b_{\boldsymbol{x}+\boldsymbol{e}_{\nu}}^{\dagger}\, b_{x}+e^{-\boldsymbol{h} \cdot \boldsymbol{e}_{\nu} / \hbar}\, b_{\bm{x}}^{\dagger}\, b_{\bm{x}+\boldsymbol{e}_{\nu}}\right)+\sum_{\boldsymbol{x}} V_{\bm{x}} \, b_{\bm{x}}^{\dagger}\, b_{\bm{x}}, \\
    & \nonumber
\end{align}
\hspace{0.5pt}where $b_{\bm{x}}^{\dagger}\, , \, b_{\bm{x}}$ are boson creation and annihilation operators, respectively, and $\boldsymbol{e}_{\nu}$ are unit lattice vectors. Here, $t \sim$ $V_{\text {bind }} \exp \left(-\sqrt{2 m V_{\text {bind }}} a / \hbar\right)$ is the hopping parameter with $V_{\text{bind}}$ being the binding energy of the columnar defect and $a$ being the lattice spacing. Again, a periodic boundary condition is imposed, $b_{\bm{x}+N_{\nu} \bm{e}_{\nu}}\, = \, b_{x}$ for $\nu=1,2, \ldots, d$, where $N_{\nu} \, \equiv \, L_{\nu} / a$. Then we find another set of interesting results that give valuable insights: 

\begin{table}[h]
    \centering
        \begin{tabular}{l  p{11.95cm}}
        \textbf{(1)\footnotemark} &\normalsize{The non-Hermitian system here possesses complex conjugate pairs of eigenvalues: complex eigenvalue $\varepsilon$ associated with the right eigenfunction $\psi^R$ has a corresponding conjugate pair $\varepsilon^*$ associated with the complex conjugate of right eigenfunction $(\psi^R)^*$. [It guarantees the reality of the partition function $Z$ in Eq.(\ref{Partition Function})]} \\
        & \\
        \textbf{(2)} & A symmetry is obtained: $\mathcal{H}(
        \bm{h})^T \ = \ \mathcal{H}(-\bm{h})$. This implies that a right eigenfunction of $\mathcal{H}(\bm{h})$ is equal to the left eigenfunction of $\mathcal{H}(-\bm{h})$. Surprisingly, they have the same eigenvalue. 
        \end{tabular}
\end{table}    

\footnotetext{This result is the primary study of the next few sections, it is supported by the idea of \textit{spontaneous} \\ $\mathcal{PT}$-symmetry breaking and later (Chapter 2), expanded by \textit{Pseudo-Hermitian Quantum Mechanics.}}

Now we have some tangible evidence to move on and work on building the theory. In the next section, it will become clear why we have introduced this experiment and its result. \textbf{The most important of them is (1)} and we will see that it corroborates with the theoretical predictions. \\ \\ \\ 

\section{Eigenvalues of a $\mathcal{PT}$-symmetric Hamiltonian \& How to Compute Them}

The experimental evidence at the end of the last section clearly pointed out that a very distinctive characteristic of a Hamiltonian with parity-time symmetry lies in the analysis of it’s eigenvalues i.e. solving it’s Schr\"{o}dinger eigenvalue problem. Physical properties in a quantum theory are governed by the Hamiltonian of the system. To explain, we present a three-fold rationalisation: 

\begin{longtable}{l  p{11.7cm}}
        &\\
        \textbf{Characteristic 1} \vline & Energy levels of a quantum system are determined by the Hamiltonian. The time-independent Schr\"{o}dinger eigenvalue problem is solved to find the eigenvalues with certain boundary conditions depending on the problem,
        \begin{equation}
            \label{time-independent eigenvalue equation}
            H \, \psi \, = \, E \,\psi. 
        \end{equation}
        Eq.(\ref{time-independent eigenvalue equation}) must be solved carefully using proper boundary conditions, and the energy eigenvalues must be real and positive. \\
        &\\
        \textbf{Characteristic 2} \vline & Time-evolution is described by the Hamiltonian. States (Schr\"{o}dinger picture) and operators (Heisenberg picture) can evolve depending on the interpretation. The states evolve according to the time-dependent Sch\"{o}dinger equation, 
        \begin{equation}
            \label{time-dependent Schrondinger equation}
            i \, \frac{\partial}{\partial t}\, \psi(t) \, = \, H \, \psi(t).
        \end{equation}
        If the Hamiltonian is assumed to be independent of time, i.e. there is no question of time ordering arising due to non-commutativity of the Hamiltonian between different times, then the solution is
        \begin{equation}
            \label{time-dependent schrodinger eq solution}
           \psi(t) \, = \, e^{-i H t} \, \psi(0).
        \end{equation} 
        The way this time-translation acts on a state results in the norm of any evolved state that remains constant in time. This is because\\ 
        
        \hspace{33.08mm} \vline & the Hamiltonian is Hermitian, i.e. $H^\dagger = H$. You can see this very easily: assume a general state $|\alpha(t_0)\rangle$ at time $t_0 = 0$. Evolve the state, $e^{-i H t}|\alpha(0)\rangle=|\alpha(t)\rangle \text {,}$ and hence,
        {\begin{align}
            \label{time evolved bra}
            \langle\alpha(t)| \, &= \, \langle\alpha(0)| \, (e^{-i H t})^{\dagger} \nonumber\\
            &= \, \langle\alpha(0)| \, e^{i H^{\dagger} t} \nonumber\\
            &= \, \langle\alpha(0)| \, e^{i H t} \qquad \bm{[ \, \because \, H^{\dagger}=H \, ]}. 
        \end{align}} 
        The norm of a state is calculated as $\sqrt{\ | \! \braket{\alpha} \! | \ }$. Therefore,
        {\begin{align}
            \label{norm of time evolved state}
            \sqrt{\ | \! \braket{\alpha(t)} \! | \;} \ &= \ \sqrt{ \ |  \langle\alpha(0)|e^{i H t} e^{-i H t}| \alpha(0) \rangle  | \ } \nonumber \\
            &= \ \sqrt{\ |\!\braket{\alpha(0)} \!| \ } \nonumber \\
            &= \ \text{\textbf{Norm}} \, \bm{(}|\alpha(0)\rangle\bm{)}.
        \end{align}}
            The square of the norm in Eq.(\ref{norm of time evolved state}) is interpreted as probability and, hence, it is essential that it remains constant in time. In the case of a $\mathcal{PT}$-symmetric Hamiltonian, the norm remains constant but \textit{Dirac Hermiticity} may not be respected. \\
        &\\
        \textbf{Characteristic 3} \vline & Symmetries of a quantum system are subsumed by its Hamiltonian. Any anti-linear operator that commutes with the Hamiltonian will have \textit{simultaneous eigenstates} with the Hamiltonian. These \textit{simultaneous  energy eigenstates} will inherit the properties necessary to be also the eigenstates of the commuting operator. \\
\end{longtable}
\par
The study of non-Hermitian quantum mechanics involves the scrutiny of \textbf{Characteristic 2} and its results. As we saw in the Experimental Evidence part of the previous section, the bifurcation of eigenvalues into complex conjugate pairs from real ones plays a pivotal role in understanding the nature of non-Hermitian Hamiltonians. We now look at this idea and further extrapolate on how to calculate the eigenvalues. 

\subsection{Spontaneous $\mathcal{PT}$-symmetry Breaking}

In his paper \cite{PhysRevLett.80.5243}, Bender \textit{et al} examines the numerical and asymptotic properties of the class of non-Hermitian Hamiltonians parameterized by a real number $N$:
\begin{align}
    \label{parameterized hamiltonian by N}
    & \nonumber\\
    H \ = \ p^2 \ + \ m^{2}x^{2} \ - \ (ix)^N \qquad \quad \text{($N \in \mathbb{R}$).}\\
    & \nonumber
\end{align}
It was found that when we transform Eq.(\ref{parameterized hamiltonian by N}) into a Lagrangian for a quantum field theory, it produces asymptotically free theories that have stable critical points \cite{Carl_M_Bender_1999,PhysRevD.57.3595,PhysRevD.62.085001}. This means that the Hamiltonians in Eq.(\ref{parameterized hamiltonian by N}) are physical even though they turn out to be non-Hermitian and, sometimes, contain upside-down potentials for certain values of $N$. $\mathcal{PT}$-symmetry has been widely used in quantum field theories, often leading to riveting results. Parity invariance alone is not very strong, and hence $\mathcal{PT}$-invariance is a more natural choice. Bender \textit{et al} studied the Hamiltonian in Eq.(\ref{parameterized hamiltonian by N}) using various methods such as moving the $x$ into the complex plane \cite{BENDER1993442}, using the WKB phase integral methods \cite{PhysRevLett.80.5243,Bender_2007}, numerical methods like Runge-Kutta \cite{PhysRevLett.80.5243,Bender_2007}, and comparison of known Schr\"{o}dinger equations with Eq.(\ref{parameterized hamiltonian by N}) \cite{PhysRevLett.80.5243,Bender_2007}. We will look at some of the techniques in the next subsection. Here, we present the findings of their study and it's implications. The Schr\"{o}dinger eigenvalue problem associated with Eq.(\ref{parameterized hamiltonian by N}) is (for mass or $m = 0$): 
\begin{align}
    \label{Schrondinger Eigenvalue problem for parameterized Hamiltonian}
    & \nonumber\\
    -\psi^{\prime \prime}(x) \ - \ (i x)^{N} \, \psi(x) \ = \ E \, \psi(x). \\
    & \nonumber
\end{align}
The results obtained are against the real number $N$, which essentially makes Eq.(\ref{parameterized hamiltonian by N}) a family. Here is the summary (for $m = 0$):

\begin{longtable}{l  p{11.7cm}}
    &\\
    \textbf{(1)} \hspace{0.2em} \framebox{$N \, \geq \, 2$} $\hspace{1.27em} \longrightarrow$ & The calculated spectrum is infinite,discrete and entirely real and positive. This result is what makes the theory physical. We see here that even non-Hermitian Hamiltonians are producing eigen\\
    &-values that can be observed in experiments because they belong to $\mathbb{R}^{+}$. \\
    &\\
    \textbf{(2)} \hspace{0.2em} \framebox{$N \, = \, 2$} $\hspace{1.27em} \longrightarrow$ & This is a phase transition point for the spectrum. It is quite simple to see that this case is that of a simple harmonic oscillator that is known and exactly solvable. The spectrum of the harmonic oscillator is: $E_n = 2n + 1$, which is again infinite, discrete, and is in $\mathbb{R}^{+}$.   \\
    &\\
    \textbf{(3)} \framebox{$1< N <  2$} $\ \longrightarrow$ & In this region finite number of real and positive eigenvalues are obtained. But an interesting thing happens as we move towards 1 from 2: \textit{\textbf{adjacent eigenvalues start merging together to form complex conjugate pairs.}} We have encountered this kind of results at the end of Sub-Section \ref{Experimental Evidence} where the system with imaginary (non-Hermitian) vector potentials is studied with a non-neglected random potential. \textit{\textbf{This effect is termed as spontaneous $\mathcal{PT}$-symmetry breaking.}} After a critical value of $N$ no real eigenvalues remain except the ground state. \\
    &\\
    \textbf{(4)} \hspace{0.01em} \framebox{$N \rightarrow 1^{+}$} $\hspace{0.89em} \longrightarrow$ & Ground state energy diverges as $N$ approaches $1$ from above.\\
    &\\
    \textbf{(5)} \hspace{0.2em} \framebox{$N \, \leq \, 1$} $\hspace{1.23em} \longrightarrow$ & No real eigenvalue endures and the spectrum becomes entirely complex.\\
\end{longtable}

The above results showed that more research was needed in the field of non-Hermitian quantum mechanics, especially due to results \textbf{(1)} and \textbf{(2)}. These two discoveries were the most fascinating. It took many years (after 1998) to rigorously prove that \textit{unbroken} $\mathcal{PT}$-symmetry leads to the reality of the spectrum and was finally shown by Dorey \textit{et al} in 2001 \cite{Dorey_2001,dorey2004reality}. \par

If a Hamiltonian is parity-time invariant, then it commutes with the $\mathcal{PT}$ operator. Any operator invariant with the Hamiltonian is sufficient to commute with it. But the $\mathcal{PT}$ operator is not linear; it is antilinear. Hence, even though the Hamiltonian and the parity-time operator commute, there is no guarantee that they will have \textit{simultaneous eigenstates}, nondegenerate or otherwise. \par

We can do a small calculation to see why it is erroneous to assume that the Hamiltonian and the $\mathcal{PT}$ operator have simultaneous eigenstates. Let $\psi$ be an eigenstate for both H and $\mathcal{PT}$, then we have
\begin{align}
    \label{PT EV eq}
    & \nonumber\\
    \mathcal{PT}\,\psi \ = \ \lambda \, \psi, \qquad  \quad \text{$\lambda$ is the eigenvalue.} \\
    & \nonumber
\end{align}
$\mathcal{PT}$ multiplied from the left and using Eq.(\ref{PT is invertible}), which gives $(\mathcal{PT})^2 \ = \ \bm{1}$, we get
\begin{align}
    \label{PT multiplied from left}
    &\nonumber\\
    &\qquad \; \qquad \qquad \qquad \quad \psi \  = \ (\mathcal{PT})\lambda(\mathcal{PT})^{2}\,\psi \nonumber \\
    &\nonumber\\
    \qquad \qquad &\Longrightarrow \qquad \psi \ = \ \mathcal{PT} \lambda(\mathcal{T}\mathcal{P})(\mathcal{P T}) \, \psi \qquad \qquad [\, \because \ \mathcal{P T}-\mathcal{T P}\ = \ 0, \ \text{i.e. Eq.(\ref{P and T commute})} \, ]\nonumber\\
    &\nonumber\\
    &\Longrightarrow \qquad \psi \ = \ \lambda^{*} \: \mathcal{P}^{2}(\mathcal{P T}) \, \psi \ = \ \lambda^{*}\: \bm{1} \: \lambda \, \psi \qquad [\, \because \ \mathcal{T}\, i \, \mathcal{T} \, = \, -i \quad \text{and} \quad \mathcal{P}^{2}=1 \,]  \nonumber\\
    &\nonumber\\
    &\Longrightarrow \qquad \psi \ = \ \lambda^{*} \, \lambda \, \psi \ = \ |\lambda|^{2} \, \psi \qquad \quad \text{or} \qquad \quad |\lambda|^{2} \ = \ 1.  \\
    & \nonumber
\end{align}
Hence, Eq.(\ref{PT multiplied from left}) infers that the eigenvalue $\lambda$ is just a phase factor,
\begin{align}
    \label{eigenvalue is a phase}
    &\nonumber\\
    \lambda \ = \ e^{i \,\theta}.\\
    &\nonumber
\end{align}
\hspace{1.5em}Now, $\mathcal{PT}$ is multiplied from the left to the time-independent Sch\"{o}dinger eigenvalue equation, i.e. Eq.(\ref{time-independent eigenvalue equation}) and again, using the property $(\mathcal{PT})^2 \ = \ 1 $, we get
\begin{align}
    \label{Eigenvalue is real}
    &\nonumber\\
    & \qquad \qquad \qquad \qquad \qquad \qquad (\mathcal{P} \mathcal{T}) \, H \, \psi \ = \ (\mathcal{P} \mathcal{T}) \, E \, (\mathcal{P} \mathcal{T})^{2}\, \psi \nonumber\\
    &\nonumber\\
    &\Longrightarrow \qquad H \, (\mathcal{PT}) \, \psi \ = \ (\mathcal{PT}) \, E \, (\mathcal{PT})^2 \, \psi \qquad [ \, \text{H is $\mathcal{PT}$-symmetric} \ \Longleftrightarrow \ \bm{[}H, \, \mathcal{PT}\bm{]} \ = \ 0 \,] \nonumber\\
    &\nonumber\\
    &\Longrightarrow \qquad H \, \lambda \, \psi \ = \ (\mathcal{PT})\, E \, (\mathcal{TP}) \, (\mathcal{PT}) \, \psi \qquad \quad \ [\, \because \ \mathcal{P T}-\mathcal{T P}\ = \ 0, \ \text{i.e. Eq.(\ref{P and T commute})} \, ] \nonumber\\
    &\nonumber\\
    & \Longrightarrow \qquad E \, \lambda \, \psi \ = \ E^{*} \, \mathcal{P}^{2} \, \lambda \, \psi \quad \text{or} \quad E \, \lambda \, \psi \ = \ E^{*} \, \lambda \, \psi \quad \ [\, \because \ \mathcal{T}\, i \, \mathcal{T} \, = \, -i \quad \text{and} \quad \mathcal{P}^{2}=1 \,] \nonumber\\
    & \nonumber \\
    & \Longrightarrow \qquad \qquad \qquad \qquad \quad \ \  E \ = \ E^{*} \qquad \qquad [ \, \because \ \lambda \ \neq \ 0, \ \text{from Eq.(\ref{eigenvalue is a phase})} \, ] \\
    &\nonumber
\end{align}
Here, Eq.(\ref{Eigenvalue is real}) points to the fact that tbe eigenvalues must be real, but that false. We have regimes in which the spectrum is real while on others it might be complex. \par

The calculation of Eq.(\ref{Eigenvalue is real}) gives us a different view of the results we have outlined above based on values of $N$. It tells us that we can also understand \textit{spontaneous} $\mathcal{PT}$\textit{-symmetry breaking} with the eigenfunctions of the $\mathcal{PT}$ operator: If all the eigenstates of a parity-time invariant Hamiltonian are also eigenstates of the $\mathcal{PT}$ operator, then $\mathcal{PT}$-symmetry is \textit{unbroken}. But if some of the eigenstates of a $\mathcal{PT}$-symmetric Hamiltonian are not simultaneous eigenstates of the $\mathcal{PT}$ operator, then the $\mathcal{PT}$ -symmetry of the Hamiltonian is \textit{broken}. \par

It is time to show how we can obtain the results discussed in this section, i.e., solve the time-independent Schr\"{o}dinger eigenvalue problem in Eq.(\ref{time-independent eigenvalue equation}) which is a second order differential equation in position coordinates. 

\subsection{Boundary conditions in case of $\mathcal{PT}$-symmetric eigenvalue problem; Solving the Schr\"{o}dinger equation}\label{Boundary Conditions and Eigenvalue Problem}

The complicated Schr\"{o}dinger equations involving non-Hermitian Hamiltonians are very difficult to solve, although they might look simple. Therefore, it is of utmost importance that the boundary conditions for the problem are chosen prudently. For the sake of clarity, we mention the Hamiltonian family again, in a different form and with a different parameter. We do this remodification for lucidity and hence will be the consistent notation we use for the rest of the treatise.
\begin{align}
    \label{m equals 0 epsilon Hamiltonian}
    &\nonumber\\
    H \ = \ p^{2} \, + \, x^{2} \, (i x)^{\epsilon}, \qquad \quad \text{here, mass or $m \ = \ 0$ and $\epsilon \in \mathbb{R}$}   \\
    &\nonumber
\end{align}
To convert it into a differential equation, we use
\begin{align}
    \label{postion basis transformation}
    &\nonumber\\
    x \ \rightarrow \ x \qquad \text { and } \qquad p \ \rightarrow \ -i \, \frac{\mathrm{d}}{\mathrm{d} x},\\
    &\nonumber
\end{align}
but we take $x$ as complex for reasons that will be clear in the following subsection. Hence, the differential equation obtained is of the form
\begin{align}
    \label{schrondinger equation in position basis for epsilon hamiltonian}
    &\nonumber\\
    -\psi^{\prime \prime}(x) \, + \, x^{2} \, (\mathrm{i} x)^{\epsilon}  \, \psi(x) \ = \ E \, \psi(x).\\
    &\nonumber
\end{align}
\hspace{1.5em}To solve Eq.(\ref{schrondinger equation in position basis for epsilon hamiltonian}) is a herculean task and cannot be done for arbitrary values of $\epsilon$. But the asymptotic behaviour of this differential equation can be studied using WKB methods. For differential equations of the form $y^{\prime \prime}(x) \, + \, V(x) \, y(x) \ = \ 0$, where $V(x)$ grows as $|x|  \rightarrow  \infty$ the exponential component of asymptotic $y(x)$ for substantial $|x|$ is given by,
\begin{align}
    \label{Asymptotic Exponential}
    &\nonumber\\
    y(x) \ \sim \ \exp [\pm \int^{x} \mathrm{~d} s \, \sqrt{V(s)}].\\
    &\nonumber
\end{align}
\hspace{1.5em}First we check for $\epsilon \ = \ 0$, which is exactly the case for the harmonic oscillator. From Eq.(\ref{Asymptotic Exponential}) we see that in this case $y(x) \sim \exp(\pm \, \frac{1}{2} \, x^{2})$ and since it must also be square-integrable, we should choose the one with the negative sign which vanishes as we move toward large values of $x$. This quality can also be extended to the complex plane, in which case we conclude: If the states vanish exponentially on the real axis as $|x|  \rightarrow  \infty$ then in the complex-$x$ plane they should vanish in two wedges with opening angle of $\frac{\pi}{2}$ centred about the positive and negative real axes. To determine these areas in the complex plane, a method of extrapolating the eigenvalue problem to the complex plane is used and it was studied by Bender and Turbiner in their 1992 paper \cite{BENDER1993442} on extending eigenvalue problems to the complex plane. The wedges we talk about are called \textit{Stokes wedges}. For $\epsilon > 0$ ($\epsilon \in \mathbb{R}^{+}$) a logarithmic branch point occurs at $x = 0$ and the branch cut is along the imaginary axis, i.e. $x=0$ to $x=i\infty$. The solutions are single valued in the cut plane created as a result of the branch cuts \cite{dorey2006differential} and it is seen that the Stokes wedges form below the real axis and the opening angles decrease as $\epsilon$ increases \cite{Dorey_2005}. \par

The criteria for $\psi(x) \rightarrow 0$ for large $|x|$ is a basic necessity for a realisable quantum theory and as so happens that many wedges in the complex plane allow this condition, but to keep things interesting, the solution of Eq.(\ref{Schrondinger Eigenvalue problem for parameterized Hamiltonian}) is studied away from $\epsilon \, = \, 0$ (the already well established case of harmonic oscillator). It is observed for $\epsilon > 0$ that $\psi(x)$ vanishes most rapidly along the centre of \textit{Stokes wedges}, which are called the \textit{anti-Stokes lines}. For the Hamiltonian in Eq.(\ref{m equals 0 epsilon Hamiltonian}) with $\epsilon > 0$ the \textit{anti-Stokes lines} are found at angles, 
\begin{align}
    \label{anti-Stokes lines angle}
    &\nonumber\\
    \theta_{\mathrm{left}}=-\pi+\frac{\epsilon}{\epsilon+4} \frac{\pi}{2} \qquad \quad \text { and } \qquad \quad \theta_{\text {right }}=-\frac{\epsilon}{\epsilon+4} \frac{\pi}{2} .\\
    &\nonumber
\end{align}
We can integrate along any path in these wedges as long as $\psi(x)$ vanishes at the end of the paths. $\Delta = 2\pi/(\epsilon + 4)$ is the equation for the opening angles, but as $\epsilon > 2$ the wedges shift below the real axis, as can be seen from Eqs.(\ref{anti-Stokes lines angle}). \par

We show the \textit{Stokes wedges} and \textit{anti-Stokes lines} for $\epsilon = 5$ in Fig. Here we have the \textit{anti-Stokes lines} in $\theta_{\mathrm{left}} = - 13\pi/18$ and $\theta_{\mathrm{right}} = -5\pi/18$. The opening angle(s) is calculated as $\Delta = 2\pi/9$. Therefore, the left wedge runs from $-13\pi/18 \, - \, \Delta/2 \;$ to $-13\pi/18 \, + \, \Delta/2 \;$ or $-15\pi/18 \;$ to $-11\pi/18$ and the right wedge runs from $-5\pi/18 \, - \, \Delta/2 \;$ to $-5\pi/18 \, + \, \Delta/2 \;$ or $-7\pi/18 \;$ to $-3\pi/18$. The dashed orange line in Fig. represents the \textit{anti-stokes lines}, while the black curve is some eigenfunction $\psi(x)$ that vanishes in these wedges as $|x| \rightarrow \infty$. \par
\vspace{1cm}
\begin{figure}[h]
    \centering
    \frame{\includegraphics[scale=2]{images/Stokes_Wedges_Epsilon_5.jpg}}
    \caption[\textit{Stokes Wedges} and \textit{Anti-Stokes lines} in the complex plane for $\epsilon = 5$]{Here we can see that the eigenfucntion $\psi(x)$ is integrable inside the two wedges (marked with grey) below the real-$x$ axis. The function dies down most rapidly along the \textit{anti-Stokes lines} (marked with orange dashed lines) as $|x| \rightarrow \infty$. [\textit{Plotted using} \textbf{ComplexRegionPlot}\textit{ in }\href{https://www.wolfram.com/mathematica/}{\textit{Mathematica}}]}
    \label{Stokes Wedges for Epsilon Equals 5 with wave function and anti-stokes lines}
\end{figure}    

\textsuperscript{*}Note: In the complex coordinate space, there is a left-right symmetry about the imaginary$x$ axis. This is precisely because of $\mathcal{PT}$-symmetry. Let $x = Re \, x \, + \, i \, Im \, x$ be the coordinate in the complex-$x$ plane. Then, applying the parity operator negates the coordinate, i.e. $-x = - Re \, x \, - \, i \, Im \, x$ and then acts the time-reversal operator. Since $\mathcal{T}$ is anti-linear (Eq.(\ref{trc})), we have $-x^{*} = - Re \, x \, + \, i \, Im \, x $, i.e. a reflection about the imaginary-$x$ axis.
\vspace{-1.5em}
\subsubsection{Analytically Extending Eigenvalue Problems to the Complex Plane}

The concept of \textit{Stokes wedges} emerges from the idea of closely investigating eigenvalue problems in the complex plane, i.e., taking the position as a complex number. A good place to understand this scheme can be found in Bender and Turbiner’s work in 1992 \cite{BENDER1993442}. Here, we recognise why it is necessary to move to the complex coordinate space and why one might arrive at inconsistent outcomes if this extension is not taken into account. Various eigenvalue problems involving potentials with coupling-constant parameters can produce eigenvalue problems that are mathematically rich with very involved results. One eigenvalue problem may contain multiple internal eigenvalue problems, like is the case for the anharmonic oscillator potential,
\begin{align}
    \label{Anharmonic Potential}
    &\nonumber\\
    V(x) \ = \ a^{2} \, x^{6} \, - \, 3 \, a \, x^{2}.\\
    &\nonumber
\end{align}
The time-independent Schrodinger eigenvalue problem has an \textit{exact} ground-state solution for $a > 0$,
\begin{align}
    \label{ground state solution for AHO}
    &\nonumber\\
    \psi_{0}(x) \ = \ \exp(-\frac{1}{4} \, a \, x^{4}), \qquad \ \text{ground state energy $E_{0}(a) = 0$}.\\
    &\nonumber
\end{align}
The analytical continuation of $E_{0}(a) = 0$ into negative values of $a$ will lead to the conclusion that $E_{0}(a) = 0$ for all real values of $a$. But if we look at the spectra of the anharmonic potential for $a<0$, $V(x) \, = \, a^{2} \, x^{6} \, - \, 3 \, a \, x^{2}$, then it is positive because the first term dominates over the second for large values of $x$ and the parameter $a$ is squared. But the lowest energy state for this potential is found to be,
\begin{align}
    \label{AHO lowest energy state}
    &\nonumber\\
    E_0 \ = \ c \, |a|^{\frac{1}{2}}, \qquad \quad \text{with \; $c = 1.9355\dots$} \\
    &\nonumber
\end{align}
\hspace{1.5em} \textbf{The question then arises: \textit{How can the analytic continuation of a zero function, i.e. $E_{a} = 0$, be not zero?}} The answer lies in the fact that such analytic continuation must be done with critical supervision and cannot be a mere replacement of variables from positive to negative or vice versa. The way forward was first proposed by Bender and Wu in 1969 \cite{PhysRev.184.1231} and can be used to understand this oddity. \par

The anharmonic oscillator potential falls under a more general family of potentials described by the equation,
\begin{align}
    \label{Gen family of AH potentials}
    &\nonumber\\
    V(x) \ = \ a^{2} \, x^{6} \, + \, 2 \, a \, b \, x^{4} \, + \, (b^{2} \, - \, 3 \, a) \, x^{2} &&\text{[\,Eq.(\ref{AHO lowest energy state}) occurs for $b = 0$\,]} \\
    &\nonumber
\end{align}
Therefore, the Schr\"{o}dinger equation and the associated boundary condition on the real axis become
\begin{align}
    \label{AHO SE with BC}
    &\nonumber\\
    -\psi^{\prime \prime}(x) \, + \, V(x) \, \psi(x) \ = \ E(a, b) \, \psi(x) \, ; \qquad \quad \lim_{|x| \rightarrow \infty} \psi(x) \ = \ 0.\\
    &\nonumber
\end{align}
In the case of Eq.(\ref{m equals 0 epsilon Hamiltonian}) we extended the $x$ coordinate into the complex plane, similarly, we must take $x$ into the complex plane in order to also take $a$ to be complex. Therefore, again we study the asymptotic behaviour of $\psi(x)$ for $|x| \rightarrow \infty$. Eq.(\ref{ground state solution for AHO}) suggest two types of asymptotic behaviour: $\psi_{+}(x) \approx \exp (\frac{1}{4}ax^4)$ and $\psi_{-}(x) \approx \exp (-\frac{1}{4}ax^4)$. The square integral $\psi(x)$ must not diverge and the parameter $a$ should be chosen accordingly. For $a > 0$, we have
\begin{align}
    \label{for a>0 asymptotic behaviour}
    &\nonumber\\
    \psi_{-}(x) \ \longrightarrow \ 0, \qquad \text{for} \ |x| \rightarrow \infty, \ \text{with} \ |\arg x| < \frac{1}{8}\pi \  \text{and} \  \frac{7}{8}\pi < \arg x < \frac{9}{8}.    \\
    &\nonumber
\end{align}
Check Fig.\ref{Complex Region Plot 1} for the designated areas on the complex-$x$ plane. The shaded regions represent the area (in case of the $\mathcal{PT}$-symmetric Hamiltonian in Eq.(\ref{m equals 0 epsilon Hamiltonian}) it was the \textit{wedges}) within which the desired boundary condition of Eq.(\ref{for a>0 asymptotic behaviour}) is satisfied. \par

Since the eigenvalue differential equation in Eq.(\ref{AHO SE with BC}) is of second order we must have two kinds of asymptotic behaviour for $|x| \rightarrow \infty$. The other asymptotic behaviour is: $\psi_{+}(x) \rightarrow 0$ as $|x| \rightarrow \infty$, where $\psi_{+}(x) \approx \exp (\frac{1}{4}ax^4) $. It is represented by the unshaded region in Fig.\ref{Complex Region Plot 1}. \par

We obtain \textit{four} independent eigenvalue problems centred on four lines:

\begin{itemize}
    \item[i.] $Im \, x \ = \ 0 \ \longrightarrow$ Centred about horizontal shaded region in Fig.\ref{Complex Region Plot 1}.  
    \item[ii.] $Re \, x \ = \ 0 \ \longrightarrow$ Centred about vertical shaded region in Fig.\ref{Complex Region Plot 1}.  
    \item[iii.] $Re \, x \ = \ Im \, x \ \longrightarrow$ Centred about right slanted unshaded region in Fig.\ref{Complex Region Plot 1}.
    \item[iv.] $Re \, x \ = \ - Im \, x \ \longrightarrow$ Centered about left slanted unshaded region in Fig.\ref{Complex Region Plot 1}.
\end{itemize}

\begin{figure}[h]
    \centering
    \frame{\includegraphics[scale = 0.7]{images/Complex Region for integrable wave function 1.jpg}}
    \caption[Complex Region Plot for satisfaction of Boundary Conditions of Eigenfunctions for the generalised Anharmonic Oscillator Potential Eigenvalue Problem]{The shaded regions(in gray) in the complex-$x$ plane where the boundary condition: $\psi_{-}(x) \rightarrow 0 $ as $|x| \rightarrow \infty$ is satisfied for the eigenstates of Eq.(\ref{AHO SE with BC}). For $a>0$ this is the case. Although, two separate regions emerge (shaded and unshaded) from a single eigenvalue problem in order to generalize the parameter $a$. [\textit{Plotted using} \textbf{ComplexRegionPlot}\textit{ in }\href{https://www.wolfram.com/mathematica/}{\textit{Mathematica}}]}
    \label{Complex Region Plot 1}
\end{figure}
\textbf{The solutions to these four eigenvalue problems for the ground-state energy:}
\begin{itemize}
    \item[i.] Here, $\psi_{-}(x)$ has no nodes on the real axis, and hence,
                \begin{align}
                \label{Solution for i}
                \psi_{(i)}(x) \ = \ \exp (-\frac{1}{4} \, a \, x^{4} \, - \, \frac{1}{2} \, b \, x^{2}) \qquad \text{and} \qquad E_{(i)}(a,b) \ = \ b \ \cite{BENDER1993442}.
                \end{align} 
    \item[ii.] For $Re \, x \ = \ 0$, i.e., the imaginary axis, the real axis is rotated by $\frac{\pi}{2}$ so that $x = ir$ where $r \in \mathbb{R}$. This gives 
                \begin{subequations}
                    \label{Solution for ii}
                    \begin{align}
                    (-\frac{\mathrm{d}^{2}}{\mathrm{~d} r^{2}}  \, + \, a^{2} \, r^{6} \, -2 \, a \, b \, r^{4} \, + \, (b^{2} \, - \, 3 \, a) \, r^{2} \, + \, E) \, \psi(x) \ = \ 0, \\
                    \text{with the boundary condition,} \qquad \psi(r) \rightarrow 0 \quad \text{as} \ |r| \rightarrow \infty.
                    \end{align}
                    \setcounter{storesubequations}{\value{equation}}
                \end{subequations}
                Eq.(\ref{Solution for ii}) is Eq.(\ref{AHO SE with BC}) with $b$ replaced by $-b$, and $E$ replaced by $-E$ giving a completely negative spectrum,
                \addtocounter{equation}{-1}
                \begin{subequations}\setcounter{equation}{\value{storesubequations}}
                    \begin{align}
                    E_{(ii)}(a,b) \ = \ - \, b.
                \end{align}
                \end{subequations}
     \item[iii.] For $Im \, x = Re \, x $ we rotate by $\frac{\pi}{4}$ to the centre of the right slanted unshaded region and therefore make the substitution $x = r \exp(\frac{1}{4}\pi i)$, again with $r \in \mathbb{R}$ which gives,
                \begin{align}
                \label{Solution for iii}
                (-\frac{\mathrm{d}^{2}}{\mathrm{~d} r^{2}} \, + \, a^{2} \, r^{6} \, - \, 2 \, a \, \beta \, r^{4} \, + \, (\beta^{2} \, + \, 3 \, a) \, r^{2} \, - \, i E) \, \psi(x) \ = \ 0.
                \end{align}
                Here, $\beta = ib.$ Eq.(\ref{Solution for iii}) is Eq.(\ref{AHO SE with BC}) with $a$ replaced by $-a$ and $E$ replaced by $-iE$. It cannot be solved analytically, and we have to take the help of numerical methods such as the WKB approximation or the variational method \cite{Turbiner_1984}.  
     \item[iv.] For $Re \, x = - Im\, x$, we have the complex conjugate of iii. which gives 
                \begin{align}
                \label{Solution for iv}
                (-\frac{\mathrm{d}^{2}}{\mathrm{~d} r^{2}} \, + \, a^{2} \, r^{6} \, - \, 2 \, a \, \beta^{*} \, r^{4} \, + \, ({(\beta^{*})}^{2} \, + \, 3 \, a) \, r^{2} \, - \, (-i) E) \, \psi(x) \ = \ 0.
                \end{align}
\end{itemize}
\hspace{1.5em} The next step would be to take the parameter $a$ into the complex plane. To do that we must rotate $a$ onto some angle,
\begin{align}
    \label{Rotating a into the complex plane}
    &\nonumber\\
    a \ = \ \rho \, \exp \, (i\theta), \qquad\quad \theta \; \text{increasing from} \; 0 \; \text{to} \; \pi. \\
    &\nonumber
\end{align}
This makes the centre lines of the shaded and unshaded regions in Fig.\ref{Complex Region Plot 1} will rotate clockwise as $a$ rotates anticlockwise. Therefore, taking $a = \rho \exp(i \pi) = -\rho$ or rotating $a$ by $\pi$ makes the solution of i. as $\psi(x) \approx \exp(\frac{1}{4}ax^4)$, and $\psi_{+}(x) \rightarrow 0$ as $|x| \rightarrow \infty$ on the line $Re \, x = -Im \, x $. The final rotated problem looks like
\begin{align}
    \label{Rotated a to pi}
    &\nonumber\\
    &(-\frac{\mathrm{d}^{2}}{\mathrm{~d} x^{2}} \, + \, a^{2} \, x^{6} \, - \, 2 \, a \, b \, x^{4} \, + \, (b^{2} \, + \, 3 \, a) \, x^{2} \, - \, E \, \psi(x) \ = \ 0, \nonumber\\
    \text{with bou}&\text{ndary condition,} \quad \psi(x) \longrightarrow 0 \; \text{as} \; |x| \rightarrow \infty \quad \text{\&} \quad x \ = \ r \, \exp(-i \, \frac{\pi}{4}).   
\end{align}
Replacing $x = \exp(-i \, \frac{\pi}{4}) $ in Eq.(\ref{Rotated a to pi}) gives,
\begin{align}
    \label{Rotated a to pi eigenvalue problem}
    &\nonumber\\
    (-\frac{\mathrm{d}^{2}}{\mathrm{~d} r^{2}} \, &+ \, a^{2} \, r^{6} \, - \, 2 \, i \, a \, b \, r^{4} \, + \, (-b^{2} \, - \, 3 \, a) \, r^{2} \, + \, i \, E) \, \psi(r) \ = \ 0, \nonumber\\
    &\text{with boundary condition,} \quad \psi(r) \longrightarrow 0 \; \text{as} \; |r| \rightarrow \infty.
\end{align}
Eq.(\ref{Rotated a to pi eigenvalue problem}) is the eigenavlue problem in case i. with $b = -ib$, and $E = -iE$. This gives the solution $-iE = -ib$ or $E = b$ [from Eq.(\ref{Solution for i})]. Therefore, we can see that $E$ does not change with a sign change of $a$ as $E$ is not a function of $a$. But if we had just replaced $a$ with $-a$ in the eigenvalue equation, we would get a different problem, that is, 
\begin{align}
    \label{a replaced with -a eigenvalue problem}
    &\nonumber\\
    (-\frac{\mathrm{d}^{2}}{\mathrm{~d} x^{2}} \, + \, a^{2} \, x^{6} \, - \, 2 \, a \, b \, x^{4} \, + \, (b^{2} \, + \, 3 \, a) \, x^{2} \, - \, E) \, \psi(x).\\
    &\nonumber
\end{align}
This is the case where $Re \, x = Im \, x$ with the coordinate space rotated by an angle of $\frac{\pi}{4}$, and with an undetermined eigenvalue problem that must be solved numerically.  \par

\textbf{The above example shows that merely changing the sign of} $\bm{a}$ \textbf{replaces the problem entirely to a different eigenvalue problem and hence, is too naive.} This careful treatment is necessary so that we do not lead to paradoxes and can be applied to even simpler systems, say \textit{harmonic oscillator}. A similar region plot of the harmonic oscillator eigenvalue problem is shown in Fig.\ref{Complex Region Plot 2}. The specific eigenvalue problem is
\begin{align}
    \label{Harmonic Oscillator Eigenvalue Problem}
    &\nonumber\\
    &(-\frac{\mathrm{d}^{2}}{\mathrm{~d} x^{2}} \, + \, \frac{1}{4} \, a^{2} \, x^{2} \, - \, E) \, \psi(x) \ = \ 0, \nonumber\\ 
    \text{wit}&\text{h boundary condition,} \quad \lim_{|x| \rightarrow \infty} \psi(x)=0.  \\
    &\nonumber
\end{align}
The solution to this eigenvalue problem is very well established: $\psi(x) = \exp (- \frac{1}{4}ax^2)$ with energy eigenvalues, $E_n = (n + \frac{1}{2})a$. The energy of the ground state is $E_0 = \frac{1}{2}a$. Again, if we replace here $a$ with $-a$ then the eigenvalues become negative but the eigenvalue problem remains the same. \textbf{\textit{A paradox, again!}} \par 

\begin{figure}[h]
    \centering
    \frame{\includegraphics[scale = 0.7]{images/Complex Region for integrable wave function 2.jpg}}
    \caption[Complex Region Plot for satisfaction of Boundary Conditions of Eigenfunctions for the Harmonic Oscillator Eigenvalue Problem]{Shaded region(in gray) in the complex-$x$ plane where the boundary condition: $\psi(x) \rightarrow 0 $ as $|x| \rightarrow \infty$ is satisfied for eigenfunctions of the Harmonic oscillator eigenvalue problem in Eq.(\ref{Harmonic Oscillator Eigenvalue Problem}). In case of the Anharmonic Oscillator the problem got divided into four parts and here it gets divided into two part: one on the real-$x$ axis and other on the imaginary-$x$ axis.[\textit{Plotted using} \textbf{ComplexRegionPlot}\textit{ in }\href{https://www.wolfram.com/mathematica/}{\textit{Mathematica}}]}
    \label{Complex Region Plot 2}
\end{figure}

It would be too naive to replace the negative of $a$, as we have learnt from the previous example. Instead, we again examine the asymptotic behaviour of the eigenfunction moving into the complex-$x$ plane (see Fig.\ref{Complex Region Plot 2}). Taking $a = \rho \exp(i \theta)$ and rotating $\theta$ from $0$ to $\pi$ we see that clockwise rotation of the centre line in the shaded region in Fig.\ref{Complex Region Plot 2} moves to the centre line of the unshaded part. This means that the eigenvalue problem becomes
\begin{align}
    \label{Rotating harmonic oscillator a to pi}
    &\nonumber\\
    &(-\frac{\mathrm{d}^{2}}{\mathrm{~d} x^{2}} \, + \, \frac{1}{4} \, a^{2} \, x^{2} \, - \, E) \, \psi(x) \ = \ 0, \nonumber\\ 
    \text{wit}&\text{h boundary condition,} \quad \lim _{x \rightarrow \pm i \infty} \psi(x)=0. \\ 
    &\nonumber
\end{align}
Replacing $x=ir$ gives the solution

\begin{align}
    \label{Taking x to be purely imaginary in rotated a of harmonic oscillator}
    &(-\frac{\mathrm{d}^{2}}{\mathrm{~d} r^{2}} \, + \, \frac{1}{4} \, a^{2} \, r^{2} \, + \, E) \, \psi(r) \ = \ 0, \nonumber\\
    \text{wit}&\text{h boundary condition,} \quad \lim _{r \rightarrow \pm \infty} \psi(r)=0. \\
    &\nonumber
\end{align}
This is Eq.(\ref{Harmonic Oscillator Eigenvalue Problem}) with $E$ replaced by $-E$, giving the ground-state eigenvalue as
\begin{align}
    \label{Ground State Energy of complex a HO}
    &\nonumber\\
    E \ = \ - \, a.\\
    &\nonumber
\end{align}
\textbf{Simply replacing} $\bm{a}$ \textbf{by} $\bm{- a}$ \textbf{gives an independent eigenvalue problem and hence we must extend the problem to the complex plane to get consistent results.}

\subsection{WKB phase integral approach}

In Sub-Section \ref{Boundary Conditions and Eigenvalue Problem}, we saw that the analytic continuation of the eigenvalue problem can churn out the mathematical subtleties as well as the paradoxes associated with them. But once the problem is set, we may find ourselves solving differential equations that cannot always be solved analytically, and hence may require numerical approximation. WKB techniques are one of those numerical approximation methods that can be used to solve the final eigenvalue problems in the case that the mathematical analysis fails. It is certainly necessary to use this approach not only to solve the equations but also to verify the solved ones. \par

Returning to the original problem at hand, that is, the eigenvalue problem in Eq.(\ref{schrondinger equation in position basis for epsilon hamiltonian}) for the $\mathcal{PT}$-symmetric Hamiltonian in Eq.(\ref{m equals 0 epsilon Hamiltonian}), we can use the WKB approximation technique for $\epsilon > 0$. It gives accurate and reliable results. More importantly, the approximation must be made while being in the complex-$x$ plane. In this case, the turning points for the WKB approximation are given by the roots of $E = x^2 \, (ix)^\epsilon$ because this term in Eq.(\ref{m equals 0 epsilon Hamiltonian}) is responsible for taking the problem into the complex-$x$ plane. The roots/turning points are given by 
\begin{align}
    \label{Turning Points WKB}
    &\nonumber\\
    x_{-} \ = \ E^{ \, \frac{1}{\epsilon+2}} \, \mathrm{e}^{ \, i \pi(\frac{3}{2}-\frac{1}{\epsilon+2})}, \qquad \quad \text{and} \qquad \quad x_{+} \ = \ E^{ \,\frac{1}{\epsilon+2}} \, \mathrm{e}^{ \,-i \pi(\frac{1}{2}-\frac{1}{\epsilon+2})}. \\
    &\nonumber
\end{align}
As we can see from Eq.(\ref{Turning Points WKB}) that they lie in the lower-half (upper-half) for $\epsilon > 0$ ($\epsilon < 0$). The leading order WKB phase integral quantisation is found to be \cite{Bender_2007},
\begin{align}
    \label{WKB Approx integral}
    &\nonumber\\
    (n+\frac{1}{2}) \pi \ = \ 2 \sin (\frac{\pi}{\epsilon+2}) \, E^{ \, \frac{1}{\epsilon+2}+\frac{1}{2}} \int_{0}^{1} \mathrm{~d} s \sqrt{1-s^{ \, \epsilon+2}}.\\
    &\nonumber
\end{align}
Eq.(\ref{WKB Approx integral}) can be solved for $E_n$, giving \cite{Bender_2007}:
\begin{align}
    \label{WKB Eigenvalues}
    &\nonumber\\
    E_{n} \ \sim \ \left[\frac{\Gamma(\frac{3}{2}+\frac{1}{\epsilon+2}) \, \sqrt{\pi} \, (n+\frac{1}{2})}{\sin (\frac{\pi}{\epsilon+2}) \, \Gamma(1+\frac{1}{\epsilon+2})} \right]^{\frac{2 \epsilon+4}{\epsilon+4}} \qquad \quad (n \rightarrow \infty). \\
    &\nonumber
\end{align}
A quick calculation of shows that the above values are real and positive, which is a mark of its physicality. The eigenvalue problem for Eq.(\ref{m equals 0 epsilon Hamiltonian}) is also analytically approached and the differential equations are solved numerically using Runge-Kutta techniques. A table comparing these values are shown in Bender's review of $\mathcal{PT}$-symmetry published in 2007 \cite[see Pg. 961]{Bender_2007}. \par

\section{The complex symmetry operator, $\mathcal{C}$} \label{C operator}

We have talked about the nature of non-Hermitian Hamiltonians and how to solve their Schr\"{o}dinger equations to obtain the spectrum but, one thing that was consistently ignored was the \textit{positive definiteness} of the norm generated by a non-Hermitian theory. Surely, if the Hamiltonian is not Hermitian, then we cannot expect its eigenvectors to be orthogonal, and hence, the norm must be modified. In this section we will try to paint a picture in which we built quantum theory from the basics, using our knowledge of non-Hermitian Hamiltonian theory, while being consistent with its standard interpretation. Since the seasoned researcher in physics is very much familiar with standard quantum theory, we will straight away get into non-Hermitian theory. For the amateur reader, the first few chapters (till dynamics) of any standard book on quantum mechanics should do the trick.

\subsection{Designing a quantum theory based on $\mathcal{PT}$-symmetric Hamiltonians}
\label{Recipe for PT-symm. QM}

The essence of building a Quantum Theory from scratch, considering the fact that the Hamiltonian is non-Hermitian, lies in the redefinition of the inner product. As we discuss further, it will be natural for us to see that in a $\mathcal{PT}$-symmetric theory the inner product varies based on the given Hamiltonian. A $\mathcal{PT}$-symmetric theory will choose its own Hilbert space and the inner product in this space. Now we look at how the standard results of ordinary quantum theory stack up against $\mathcal{PT}$-symmetric theory:

\begin{longtable}{l p{108.5mm}}
        &\\
        \textbf{Eigenfunctions and} \vline & \hspace{1.5em} The techniques to determine the eigenvalues of a non- \\
        \textbf{Eigenvalues  } \hspace{12.67mm} \vline & Hermitian $\mathcal{PT}$-symmetric Hamiltonian, using analytical and numerical methods, were extensively covered in the last few sections. The physicality of parity-time quantum theory lies in the \textit{spontaneously unbroken} regime, where we find real eigenvalues. Therefore, it is wise to develop our design assuming that the eigenvalues of the $\mathcal{PT}$-symmetric Hamiltonian are real. Phrased differently, $\mathcal{PT}$ and H commute with each other and thereby have simultaneous eigenfunctions. \\
        &\\
        \textbf{Orthogonal set of} \hspace{1.61mm} \vline & \hspace{1.5em} The geometry of the Hilbert space and orthogonality of \\ 
        \textbf{Eigenfuncions} \hspace{9.64mm} \vline & vectors can only be determined after defining an inner product or norm in the space. Hilbert spaces with such complex Hamiltonians, when expounded, produce indefinite metrics that are physically unrealisable \cite{PhysRevLett.89.270401}. In order to obtain a consistent theory, a symmetry operator, $\mathcal{C}$, will be introduced in due course. This operator will allow for an accordant inner product to be defined, and consequently, orthogonality of vectors can be introduced. \textbf{(Remember: The orthogonality of a pair of vectors depends on the inner product.)} An educated guess of an inner product would be: \\
        
        \hspace{39.64mm} \vline & 
        {\begin{align}
        \label{Guess for PT inner product}
        (\psi, \phi) \equiv \int_C \mathrm{~d} x \, [\psi(x)]^{\mathcal{P T}} \, \phi(x) \, = \,  \int_C \mathrm{~d} x \, [\psi(-x)]^* \, \phi(x).
        \end{align}}
        Here, $C$ is a contour, and it must be in the \textit{Stoke’s Wedges} that we have elaborated extensively in previous sections. You can guess that the form has been borrowed from the general definition used in standard quantum mechanics. It can be shown that (\ref{Guess for PT inner product}) confers the orthogonality of degenerate states, but does not keep it \textit{positive definite}. \\
        &\\
        \textit{\textbf{Redefining the }} \hspace{5.87mm} \vline & \hspace{1.5em} We introduce a new symmetry operator, $\mathcal{C}$, and then use\\
        \textit{\textbf{Inner Product}} \hspace{7.87mm} \vline & it to form a redefined inner product. $\mathcal{C}$ will commute with both $\mathcal{P}$ and $\mathcal{T}$, and hence represent a symmetry for the Hamiltonian. This operator is similar to the charge conjugation operator in particle physics. The new inner-product ($\mathcal{CPT}$ inner-product) is defined as: 
        {
        \begin{align}
        \label{CPT inner product}
        \langle\psi\mid\chi\rangle^{\, {\mathcal{C P T}}}  = \int \mathrm{d} x \, \psi^{\, \mathcal{C P T}}(x) \, \chi(x), \nonumber\\
        \text{where,} \quad \psi^{ \, \mathcal{C P T}}(x) \, = \, \int \mathrm{d} y \, \mathcal{C}(x, y) \, \psi^*(-y).
        \end{align}
        }
        This inner-product allows us to build a positive norm and a unitary Hilbert space. Later in the chapter, we will see how to construct this $\mathcal{C}$ operator (for any general Hamiltonian, $H$) and represent it as a sum of the eigenfunctions of $H$. \\ 
        &\\
        \textbf{Normalising the} \hspace{4.92mm} \vline & \hspace{1.5em} We saw in Eqs.(\ref{PT EV eq}) to (\ref{eigenvalue is a phase}) that $H$ and $\mathcal{PT}$ has simul-\\
        \textbf{eigenfunctions \&} \hspace{3.46mm} \vline  & taneous eigenstates with eigenvalues of the form $\lambda = e^{i\alpha}$,\\
        \textbf{Completeness} \hspace{9.84mm} \vline & (replaced $\theta$ with $\alpha$) where phase $\alpha$ depends on the subscript, $n$, of the eigenstate $\psi_n(x)$. This allows us to construct $\mathcal{PT}$-normalised eigenstates as, \\
        \hspace{39.7mm} \vline &
        {\begin{align}
        \label{Simultaneous Eigenstates of H and PT}
        \phi_n(x) \, \equiv \, e^{-i \alpha / 2} \, \psi_n(x).
        \end{align}}  
        Thus, $\phi_n(x)$, by construction, is a simultaneous eigenstate of $H$ and $\mathcal{PT}$ with unit eigenvalue. The $\mathcal{PT}$-symmetric inner product defined in Eq.(\ref{Guess for PT inner product}) gives a $(-1)^n \ \forall \, n$ algebraic sign which is alternating and hence not positive throughout. \textbf{\textit{The actual reason for introducing the}} $\bm{\mathcal{C}}$ \textit{\textbf{operator is to get rid of this alternating signs generated by}} $\bm{(-1)^n}$ \textbf{\textit{and keep it postive, always.}} Eq.(\ref{Guess for PT inner product}) has the advantage of being phase independent, and hence the states can be represented with a space of rays. Therefore, we can define the eigenfunctions of $H$ such that their $\mathcal{PT}$ norms are consistently $(-1)^n$:
        {\begin{align}
        \label{PT norm of energy eigenstates}
        (\phi_n \, , \, \phi_n) &= \int_C \mathrm{~d} x\left[\phi_n(x)\right]^{\mathcal{P} \mathcal{T}} \phi_n(x)=\int_C \mathrm{~d} x\left[\phi_n(-x)\right]^* \phi_n(x) \nonumber\\
        &=(-1)^n.
        \end{align}
        }
        Here, the contour $C$ lies inside \textit{Stokes wedges}. The completeness relation w.r.t. these eigenstates was calculated numerically and analytically in \cite{Mezincescu_2000,Bender_2001} and given by,
        {\begin{align}
        \label{Completeness Relation}
        \sum_{n=0}^{\infty}(-1)^n \, \phi_n(x) \, \phi_n(y) \, = \, \delta(x-y).
        \end{align}}\\
        &\\
        \textbf{Representing $\bm{H}$,} \hspace{3.72mm} \vline & \hspace{1.5em} The parity operator, in general, is defined as $\mathcal{P}(x,y) = \delta$\\
        \textbf{Green's Function,} \hspace{1.4mm} \vline & $(x+y)$ in position or coordinate basis. Then from Eq.(\ref{Completeness Relation})\\
        \textbf{\& Parity Operator} $\! \! \!$ \hspace{1pt} \vline & we see that,
        {\begin{align}
        \label{Parity in coordinate basis}
        \mathcal{P}(x, y) \, = \, \sum_{n=0}^{\infty} \, (-1)^n \, \phi_n(x) \, \phi_n(-y).     
        \end{align}}
        The Hamiltonian in the coordinate space representation is \\
        \hspace{39.7mm} \vline & given by, 
        {\begin{align}
         \label{Hamiltonian in coordinate basis}    
         H(x, y) \, = \, \sum_{n=0}^{\infty} \, (-1)^n \, E_n \, \phi_n(x) \, \phi_n(y).    
        \end{align}
        }
        Green's function for the system is essentially the functional inverse of the Hamiltonian ($\int \mathrm{d} y H(x, y) G(y, z)=\delta(x-z)$), and represented as,
        {\begin{align}
         \label{Greens Function in coordinate basis}
         G(x, y) \, = \, \sum_{n=0}^{\infty} \, (-1)^n \, \frac{1}{E_n} \, \phi_n(x) \, \phi_n(y).
         \end{align}
        }
        \\
\end{longtable}        

\subsection{Significance of the $\mathcal{C}$ operator and its construction}

The alternating sign of the $\mathcal{PT}$ norm leads us to question the validity of $\mathcal{PT}$ quantum mechanics as a physically viable theory. This is because the norm is interpreted as probability. But for the $\mathcal{PT}$ norm, we have half of the eigenstates with positive norms and half with negative norms. If we take the $\mathcal{PT}$ norm in Eq.(\ref{Guess for PT inner product}) to be our standard inner product, then a finite-dimensional $2n \times 2n$ matrix Hamiltonian quantum theory based on this assumption manifests a $SU(n,n)$ symmetry rather than $SU(2n)$ \cite{PhysRevLett.89.270401}. The whole space is divided into two separate and disjoint spaces: (1) Eigenstates with positive norms, and (2) Eigenstates with negative norms. The situation here is very similar to the one faced by Dirac while trying to formulate relativistic quantum mechanics. \textbf{\textit{Hence the symmetry operator here is represented by the letter C, just like charge-conjugation!}} The $\mathcal{C}$ operator represents a hidden symmetry in the unbroken $\mathcal{PT}$-symmetric regime: Equal number of positive and negative norm states. We thus have a physical interpretation for the negative norm states. $\mathcal{C}$ measures the sign of the $\mathcal{PT}$ norm of a state. Both $H$ and $\mathcal{PT}$ commute with $\mathcal{C}$. $\mathcal{C}$ is self-invertible and therefore has $\pm 1$ as eigenvalues. The $\mathcal{C}$ operator is represented on the position basis as the sum of the $\mathcal{PT}$ normalised eigenstates:
\begin{align}
    \label{C operator in position basis}
    &\nonumber\\
    \mathcal{C}(x, y) \, = \, \sum_{n=0}^{\infty} \, \phi_n(x) \, \phi_n(y).
\end{align}
This is the same as completeness, i.e. Eq.(\ref{Completeness Relation}), without the $(-1)^n$ factor. We can see that $\mathcal{C}$ is self-invertible using this completeness relation and the $\mathcal{PT}$ inner product in a position basis,
\begin{align}
    \label{C is self-invertible}
    &\nonumber\\
    \int d y \, C(x, y) \, C(y, z) \, &= \, \int d y \, \sum_m^\infty \, \phi_m(x) \, \phi_m(y) \, \sum_n^\infty \, \phi_n(y) \, \phi_n(z) \nonumber\\
    &= \, \int d y \, \sum_{n, m}^\infty \, \phi_m(x) \, \phi_n(z) \, \phi_m(y) \, \phi_n(y)  \nonumber\\
    &= \, \sum_{n, m}^\infty  \left\{\left(\phi_m(x) \, \phi_n(z)\right) \times \int d y \, \phi_m(y) \, \phi_n(y)\right\} \nonumber\\
    &= \, \sum_{n, m}^\infty  \, (-1)^n \, \delta_{m n} \, \phi_m(x) \, \phi_n(z) \nonumber \\
    &= \, \sum_n^\infty  \, (-1)^n \, \phi_n(x) \, \phi_n(z) \, = \, \delta(x-z). \\
    &\nonumber
\end{align}
Recall that $\mathcal{C}$ and $H$ commute and therefore must have simultaneous eigenstates. The eigenvalues of $\mathcal{C}$ are $\pm1$ and $\mathcal{C}$ acted on the eigenstates of $H$, $\phi_n$' s, giving
\begin{align}
    \label{C acting on H eigenstates}
    &\nonumber\\
    \mathcal{C} \, \phi_n(x) \, &= \, \int \mathrm{d} y \, \mathcal{C}(x, y) \, \phi_n(y) \nonumber\\
    &= \, \sum_{m=0}^{\infty} \, \phi_m(x) \int \mathrm{d} y \, \phi_m(y) \, \phi_n(y) \, = \, (-1)^n \, \phi_n(x). \\
    &\nonumber
\end{align}
\hspace{1.5em} In conventional Hermitian quantum theory, $\mathcal{C}$ becomes identical to the parity operator. $\mathcal{C}$ and $\mathcal{P}$ are distinct square roots of the identity operator, but $\mathcal{C}$ is complex, while $\mathcal{P}$ is real. In the Hermitian limit, the $\mathcal{CPT}$ operator becomes $\mathcal{T}$ and thus we finally see that $\bm{\mathcal{CPT}}$ \textbf{symmetry is the natural extension of Dirac Hermiticity}. The culminating condition is then represented as 
\begin{align}
    \label{CPT symmetry of the Hamiltonian}
    &\nonumber\\
    H^{\, \bm{\mathcal{CPT}}} \, = \, H, \\
    &\nonumber
\end{align}
and the completeness becomes, 
\begin{align}
    \label{CPT Completeness}
    &\nonumber\\
    \sum_n^\infty \, \phi_n(x)\left[ \, \mathcal{CPT} \, \phi_n(y) \, \right] \, = \, \delta(x-y). \\
    &\nonumber
\end{align}
\subsection{Example of a $\mathbf{SU(2)}$ system with a $\mathbf{\mathcal{PT}}$-symmetric Hamiltonian}

We have extensively covered the introductory theory of $\mathcal{PT}$-symmetric quantum mechanics but without giving any example. The ideas we have shared are useless if we cannot demonstrate that they work in a particular scenario. Therefore, we present a $2 \time 2$ matrix Hamiltonian example to show the calculations to generally obtain the $\mathcal{C}$ operator \cite{PhysRevLett.89.270401} . For a more general approach, using any antiunitary operator, take a look at \cite{Bender_Berry_Mandilara_2002}.\par    

A two-dimensional parity operator representing reflection about the line $y = x$ (straight line with 45$\degree$ slant about the real x direction) is given by
\begin{align}
    \label{Parity in 2D about slanted staright line}
    &\nonumber\\
    \mathcal{P} \, = \, \left(\begin{array}{cc}
                            0 & 1 \\
                            1 & 0
                        \end{array}\right)\\
    &\nonumber
\end{align}
This is unique up to any unitary transformation, and hence the most general form of $\mathcal{PT}$-symmetric two-dimensional matrix Hamiltonian is a four-parameter family of matrices of the form
\begin{align}
    \label{PT symm 2D matrix Hamiltonian }
    &\nonumber\\
    H \, = \, \left(\begin{array}{cc}
                        r e^{i \theta} & s \\
                        t & r e^{-i \theta}
                    \end{array}\right)
\end{align}
where $r$,$s$,$t$, and $\theta$ are real quantities. To make things simpler, let us take $t=s$ so that the final Hamiltonian becomes
\begin{align}
    \label{PT symm Hamiltonian for t=s}
    &\nonumber\\
    H \, = \, \left(\begin{array}{cc}
                        r e^{i \theta} & s \\
                        s & r e^{-i \theta}
                    \end{array}\right)\\
    &\nonumber
\end{align}
The time-reversal operator, being anti-linear, essentially changes the sign of $i$ (complex conjugation) and, therefore, we define it as
\begin{align}
    \label{T as Complex Conjugation}
    &\nonumber\\
    \mathcal{T} \, = \, \text{\textbf{\textit{`` Complex Conjugation "}}}\\
    &\nonumber
\end{align}
\hspace{1.5em}Now that our system is defined, we can start doing some calculations. Let us first check the $\mathcal{PT}$-invariance of the Hamiltonian in Eq.(\ref{PT symm Hamiltonian for t=s}),
\begin{align}
    \label{PT invariance of 2D Matrix Hamiltonian}
    &\nonumber\\
    &(\mathcal{PT})^{-1} \, H \, \mathcal{PT} \, = \, \mathcal{PT} \, H \, \mathcal{PT} \, = \, \mathcal{P} \, H^* \, \mathcal{P} \nonumber\\
    & \nonumber\\
    \implies \mathcal{P} \, H^* \, \mathcal{P} \, =& \, \left(\begin{array}{ll}
                                                                0 & 1 \\
                                                                1 & 0
                                                             \end{array}\right)\left(\begin{array}{cc}
                                                                r e^{-i \theta} & s \\
                                                                s & r e^{+i \theta}
                                                             \end{array}\right)\left(\begin{array}{ll}
                                                                0 & 1 \\
                                                                1 & 0
                                                             \end{array}\right) \nonumber\\
    & \nonumber\\                                                             
    =& \, \left(\begin{array}{cc}
                s & r e^{i \theta} \\
                r e^{-i \theta} & s
             \end{array}\right)\left(\begin{array}{ll}
                0 & 1 \\
                1 & 0
             \end{array}\right)=\left(\begin{array}{cc}
                r e^{i \theta} & s \\
                s & r e^{-i \theta}
             \end{array}\right) \, = \, H. \\
    &\nonumber
\end{align}
We can further calculate the eigenvalues of this Hamiltonian by solving the characteristic polynomial.
\begin{align}
    \label{Characteristic Polynomial}
    &\nonumber\\
    &\left(r e^{i \theta}-\lambda\right)\left(r e^{-i \theta}-\lambda\right)-s^2 \, = \, 0 \nonumber\\
    \implies& \, r^2-\lambda \, r e^{i \theta}-\lambda \, r e^{-i \theta}+\lambda^2-s^2 \, = \, 0 \nonumber\\
    \implies& \, \lambda^2-\lambda \, r\left(e^{i \theta}+e^{-i \theta}\right)+r^2-s^2 \, = \, 0 \nonumber\\
    \implies& \, \lambda^2-2 \, \lambda \, r \cos \theta+r^2-s^2 \, = \, 0 \nonumber\\
    \implies& \, \lambda \, = \, \frac{2 \, r \cos \theta \pm \sqrt{4 \, r^2 \cos ^2 \theta-4 r^2+4 s^2}}{2} \nonumber\\
    &\nonumber\\
    \implies& \, \boxed{\lambda \, = \, r \cos \theta \pm\left(s^2-r^2 \sin ^2 \theta\right)^{1 / 2} \,= \, E_{\pm}.} \\
    &\nonumber
\end{align}
We can see in Eq.(\ref{Characteristic Polynomial}) the eigenvalues are divided into two different domains based on the real variable values. These are
\begin{itemize}
    \item[(i)] $s^2 \, < \, r^2 \, sin^{2}\theta \, \rightarrow \,$ The eiegnvalues are complex conjugate pairs and therefore fall into the broken $\mathcal{PT}$-symmetric regime.
    \item[(ii)] $s^2 \, \geq \, r^2 \, sin^{2}\theta \, \rightarrow \,$ Here we have real eigenvalues, and thus this is the region of unbroken $\mathcal{PT}$ symmetry. 
\end{itemize}
Taking into account region (ii) of unbroken $\mathcal{PT}$ symmetry, the simultaneous eigenstates of the Hamiltonian and the parity-time operator are calculated to be
\begin{align}
    \label{Eigenstates of H and PT}
    &\nonumber\\
    \left|\varepsilon_{+}\right\rangle \, = \, \frac{1}{\sqrt{2 \cos \alpha}}
    \left(\begin{array}{l}
            e^{i \alpha / 2} \\
            e^{-i \alpha / 2}
          \end{array}\right)
    \qquad \text { and } \qquad 
    \left|\varepsilon_{-}\right\rangle \, = \, \frac{i}{\sqrt{2 \cos \alpha}}
    \left(\begin{array}{l}
            e^{-i \alpha / 2} \\
            -e^{i \alpha / 2}
          \end{array}\right). \\
    &\nonumber
\end{align}
Here, $sin \, \alpha \, = \, \frac{r}{s} \, sin \, \theta$ and we also see that the form in Eq.(\ref{Simultaneous Eigenstates of H and PT}) is maintained. \textbf{We can easily check the} $\bm{\mathcal{PT}}$ \textbf{norm for various combinations of these eigenstates.} 
\begin{itemize}
    \item[(i)] $\mathcal{PT}$ norm of the '$+$' state is given by  $\left(\left\langle\varepsilon_+\right|\mathcal{PT}\right) \, \cdot \, \left(\left|\varepsilon_+\right\rangle\right)$ and therefore,
                \begin{align}
                \label{PT acts on plus eigenstate}
                &\nonumber\\
                \mathcal{PT} \, \left|\varepsilon_{+}\right\rangle \, &= \, 
                \left(\begin{array}{ll}
       	                0 & 1 \\
                        1 & 0
                      \end{array}\right) \, \mathcal{T} \, \frac{1}{\sqrt{2 \cos \alpha}} \,
                \left(\begin{array}{l}
       	                e^{i \alpha / 2} \\
                        e^{-i \alpha / 2}
                      \end{array}\right) \nonumber\\
                &\nonumber\\      
                &= \, \frac{1}{\sqrt{2 \cos \alpha}} \,
                \left(\begin{array}{ll}
       	                0 & 1 \\
       	                1 & 0
                      \end{array}\right) \,
                \left(\begin{array}{l}
                        e^{-i \alpha / 2} \\
                        e^{i \alpha / 2}
                      \end{array}\right) \nonumber\\
                &= \, \frac{1}{\sqrt{2 \cos \alpha}} \, 
                \left(\begin{array}{l}
                        e^{i \alpha / 2} \\
                        e^{-i \alpha / 2}
                       \end{array}\right). \\
                &\nonumber\\
                \label{PT inner product of plus eigenstate}
                \therefore \qquad \left(\left\langle\varepsilon_+\right|\mathcal{PT}\right) \, \cdot \, \left(\left|\varepsilon_+\right\rangle\right) \, &= \, \frac{1}{2 \cos \alpha} \,   
                \begin{pmatrix}
                    e^{i \alpha / 2} & e^{-i \alpha / 2}
                \end{pmatrix}
                \,
                \left(\begin{array}{l}
                      e^{i \alpha / 2} \\
                      e^{-i \alpha / 2}
                      \end{array}\right) \nonumber\\
                &\nonumber\\
                &= \, \frac{1}{2 \cos \alpha} \, \left[ \, e^{i \alpha} \, + \, e^{-i \alpha} \, \right] \, = \, 1. \quad \bm{[}\, \text{\textbf{Positive!}} \, \bm{]} \\
                &\nonumber
                \end{align}
    
    \item[(ii)]$\mathcal{PT}$ norm of the '$-$' state is given by  $\left(\left\langle\varepsilon_-\right|\mathcal{PT}\right) \, \cdot \, \left(\left|\varepsilon_-\right\rangle\right)$ and therefore,
                \begin{align}
                \label{PT inner product of minus eigenstate}
                &\nonumber\\
                \left(\left\langle\varepsilon_-\right|\mathcal{PT}\right) \, \cdot \, \left(\left|\varepsilon_-\right\rangle\right) \, &= \, \frac{-i^2}{2 \cos \alpha} \,
                \begin{pmatrix}
                    -e^{-i \alpha / 2} & e^{i \alpha / 2}
                \end{pmatrix} \,
                \left(\begin{array}{c}
                        e^{-i \alpha / 2} \\
                        -e^{i \alpha / 2}
                    \end{array}\right) \nonumber\\
                &\nonumber\\
                &= \, \frac{1}{2 \cos \alpha} \, \left[-e^{-i \alpha} \, - \, e^{i \alpha / 2}\right] \, = \, -1. \quad \bm{[}\, \text{\textbf{Negative!}} \, \bm{]} \\
                &\nonumber
                \end{align}
                \textbf{\textit{This is not required because we need a positive definite norm for a consistent and correctly interpretable quantum theory.}}
    
    \item[(iii)] The inner products of the mixed-sign eigenstates vanish. This makes us see that orthogonality is realised when taking the $\mathcal{PT}$-invariant inner product. We could have taken this inner product as the standard, but only if it were postive definite (Eq.(\ref{PT inner product of minus eigenstate}) showed us this).
                \begin{align}
                \label{Mixed-Sign Eigenstates PT Inner Products}
                &\nonumber\\
                \left(\left\langle\varepsilon_+\right|\mathcal{PT}\right) \, \cdot \, \left(\left|\varepsilon_-\right\rangle\right) \, &= \, \frac{i}{2 \cos \alpha} \,   
                \begin{pmatrix}
                    e^{i \alpha / 2} & e^{-i \alpha / 2}
                \end{pmatrix}
                \,
                \left(\begin{array}{l}
                      e^{-i \alpha / 2} \\
                      -e^{i \alpha / 2}
                      \end{array}\right) \nonumber\\
                &= \, \frac{i}{2 \cos \alpha} \, \left[ \, e^0 \, - \, e^0 \, \right] \, = \, 0.  \\
                &\nonumber\\
                \left(\left\langle\varepsilon_-\right|\mathcal{PT}\right) \, \cdot \, \left(\left|\varepsilon_+\right\rangle\right) \, &= \, \frac{-i}{2 \cos \alpha} \, \begin{pmatrix}
                    -e^{-i \alpha / 2} & e^{i \alpha / 2}
                \end{pmatrix}
                \,
                \left(\begin{array}{l}
                      e^{i \alpha / 2} \\
                      e^{-i \alpha / 2}
                      \end{array}\right) \nonumber\\
                &= \, \frac{-i}{2 \cos \alpha} \, \left[ \, -e^0 \, + \, e^0 \, \right] \, = \, 0.\\
                &\nonumber
                \end{align}
\end{itemize}

The last three points have shown the problem of using a $\mathcal{PT}$ norm. \textbf{Although orthogonality is maintained, positive definiteness is not guaranteed}. Therefore, it is imperative that the symmetry operator $\mathcal{C}$ be deployed and the inner product is redefined to a $\mathcal{CPT}$ inner product where $\mathcal{C}$ checks the sign of the $\mathcal{PT}$ inner product so that the final product is positive definite. 

\subsubsection{Caculating the $\mathcal{C}$ operator}

The $\mathcal{C}$ operator can be obtained using the prescription in Eq.(\ref{C operator in position basis}) and we show here one of the entries, namely the $(1,1)$ element.
\begin{align}
    \label{C(1,1)}
    &\nonumber\\
    \mathcal{C}(1,1) \, &= \, \phi_1(1) \, \phi_1(1) \, + \, \phi_2(1) \, \phi_2(1) \nonumber\\
    &\nonumber\\
    &= \, \frac{1}{2 \cos \alpha} \, \left(e^{i \alpha / 2}\right) \, \left(e^{i \alpha / 2}\right) \, + \, \frac{i^2}{2 \cos \alpha} \, \left(e^{-i \alpha / 2}\right) \, \left(e^{-i \alpha / 2}\right) \nonumber\\
    &\nonumber\\
    &= \, \frac{1}{2 \cos \alpha} \, \left(e^{i \alpha} \, - e^{-i \alpha / 2}\right) \, = \, \frac{1}{\cos \alpha} \, i \, \sin \alpha. \\
    &\nonumber
\end{align}
Here, ``1" corresponds to ``$+$" and ``2" corresponds to ``$-$". For other elements of the $\mathcal{C}$ matrix, see Appendix \ref{Calculate C operator}. Thus, for the parity operator in Eq.(\ref{Parity in 2D about slanted staright line}), the $\mathcal{C}$ operator is
\begin{align}
    \label{C operator for y=x Parity}
    &\nonumber\\
    \mathcal{C} \, = \, \frac{1}{\cos \alpha} \, \left(\begin{array}{cc}
    i \sin \alpha & 1 \\
    1 & -i \sin \alpha
    \end{array}\right).
\end{align}
\hspace{1.5em} The $\mathcal{C}$ operator also has the property of self-ivertibility, that is, $\mathcal{C}^{\, 2} \, = \, \bm{1}$. See Appendix \ref{C self-invertibility} for verification. When this operator is applied to the eigenstates, the sign in the subscript is revealed, and this allows the $\mathcal{CPT}$ inner product to be positive even for the minus signed eigenvalue. We can see that
\begin{align}
    \label{C applied to minus eigenstate}
    &\nonumber\\
    \mathcal{C}\left|\varepsilon_{-}\right\rangle \, &= \, \frac{1}{\cos \alpha} \, \left(\begin{array}{cc}
            i \sin \alpha & 1 \\
            1 & -i \sin \alpha
          \end{array}\right) \, \frac{i}{\sqrt{2 \cos \alpha}} \, 
    \left(\begin{array}{c}
            e^{-i \alpha / 2} \\
            -e^{i \alpha / 2}
          \end{array}\right) \nonumber\\
    &= \, \frac{i}{\cos \alpha \, \sqrt{2 \cos \alpha}} \,
    \left(\begin{array}{l}
            i \, \sin \alpha \, e^{-i \alpha / 2}-e^{i \alpha / 2} \\
            e^{-i \alpha / 2}+i \, \sin \alpha \, e^{i \alpha / 2}
          \end{array}\right) \nonumber\\
    &= \, \frac{i}{\cos \alpha \, \sqrt{2 \cos \alpha}} \, 
    \left(\begin{array}{l}
            e^{i \alpha / 2} \, \left(i \, \sin \alpha \, e^{-i \alpha}-1\right) \\
            e^{-i \alpha / 2} \, \left(1+i \, \sin \alpha \, e^{i \alpha}\right)
          \end{array}\right) \nonumber\\
    &= \, \frac{i}{\cos \alpha \, \sqrt{2 \cos \alpha}} \,
    \left(\begin{array}{c}
            e^{i \alpha / 2} \, \left(i \sin \alpha \cos \alpha+\sin ^2 \alpha-1\right) \\
            e^{-i \alpha / 2} \, \left(1+i \sin \alpha \cos \alpha-\sin ^2 \alpha\right)
          \end{array}\right) \nonumber\\
    &= \, \frac{i}{\cos \alpha \, \sqrt{2 \cos \alpha}} \, 
    \left(\begin{array}{l}
            e^{i \alpha / 2} \, (i \sin \alpha-\cos \alpha) \cos \alpha \\
            e^{-i \alpha / 2} \, (\cos \alpha+i \sin \alpha) \cos \alpha
          \end{array}\right) \nonumber\\
    &= \, \frac{i}{\sqrt{2 \cos \alpha}} \, 
    \left(\begin{array}{c}
            e^{i \alpha / 2} \, \left(-e^{-i \alpha}\right) \\
            e^{-i \alpha / 2} \, \left(e^{i \alpha}\right)
          \end{array}\right) \, = \, \frac{-i}{\sqrt{2 \cos \alpha}} \, 
    \left(\begin{array}{c}
            -e^{-i \alpha / 2} \\
            e^{i \alpha / 2}
          \end{array}\right) \, \nonumber\\
    &= \, -\left|\varepsilon_{-}\right\rangle.  \\
    \nonumber
\end{align}

\vspace{-0.2cm}
This shows that $\mathcal{C}$ shows a negative sign when acted on the negative-subscripted eigenstate and a positive sign for the positive-subscripted eigenstate. And hence, we have
\begin{align}
    \label{C acting on the positive-negative eigenstates}
    &\nonumber\\
    \mathcal{C} \left|\varepsilon_{\pm}\right\rangle \, = \, \pm\left|\varepsilon_{\pm}\right\rangle.\\
    &\nonumber
\end{align}
\hspace{1.5em} The $\mathcal{CPT}$-dual of a state, $\left|u\right\rangle$, can now be defined as
\begin{align}
    \label{CPT-dual state }
    &\nonumber\\
    \left\langle u \right| \, = \, \left( \mathcal{C P T} \left|u \right\rangle \right)^{T}, \quad \text { where } T \, = \, \text { Transpose.} 
\end{align}
Using this definition, the inner product of two states, $\left|u\right\rangle$ and $\left|v\right\rangle$ becomes
\begin{align}
    \label{The CPT matrix inner product}
    &\nonumber\\
    (v \, , \, u)_{ \, \mathcal{CPT}} \, = \, \left(\left(\mathcal{C P T} \left|v \right\rangle \right)^{T}\right) \cdot \left(\left|u\right\rangle\right).
\end{align}
For this particular case of a $SU(2)$ system, the inner products of the eigenstates are postive definite and orthogonal, essentially orthonormal.
\begin{align}
    \label{The CPT matrix inner product for SU(2)}
    &\nonumber\\
    (\varepsilon_{\pm} \, , \, \varepsilon_{\pm})_{ \, \mathcal{CPT}} \, = \, 1, \quad \text{and} \quad (\varepsilon_{\mp} \, , \, \varepsilon_{\pm})_{ \, \mathcal{CPT}} \, = \, 0. \\
    &\nonumber
\end{align}
See Appendix \ref{CPT inner product of minus-plus eigenstates} for the verification of Eq.(\ref{The CPT matrix inner product}). The completeness condition can be shown for these eigenstates
\begin{align}
    \label{Completeness Condition minus-plus eigenstates}
    &\nonumber\\
    \left|\varepsilon_{+}\right\rangle\left\langle\varepsilon_{+}|+| \varepsilon_{-}\right\rangle\left\langle\varepsilon_{-}\right| \, = \, \left(\begin{array}{ll}
            1 & 0 \\
            0 & 1
          \end{array}\right). \\
    &\nonumber
\end{align}
Check Appendix \ref{completeness condition for the plus-minus eigenstates} for the calculation of the completeness condition. 

\section{How is $\mathcal{PT}$-symmetric quantum mechanics is different from Hermitian quantum mechanics?}

\begin{itemize}
    \item[]
    \item[(i)] The Hilbert space is predefined in standard quantum mechanics and the inner product in this space is with respect to the Dirac Hermitian conjugation. Any operator or state is then defined over this structure, including the Hamiltonian. In the case of $\mathcal{PT}$-symmetric quantum mechanics, the Hilbert space is determined by the Hamiltonian along with the inner product. Once the eigenstates are found, the Hilbert space structure is revealed. More complex concepts are built on this newfound space, making $\mathcal{PT}$-symmetric quatum theory a sort of 'boot-strap' approach for hypothesising the quantum world.
    \item[(ii)] If an operator in standard quantum mechanics follows the Hermiticity condition then we say that it is an obserable. Similarly in $\mathcal{PT}$-symmetric quantum mechanics, any operator $A$ must follow the $\mathcal{CPT}$ invariance condition (at time $t \, = \, 0$) that $A^{T} \, = \, \mathcal{CPT} \, A \, \mathcal{CPT}$, where $A^{T}$ denotes the transpose of $A$. If the Hamiltonian is symmetric, then this condition will hold for all times. \textbf{Note:} The $\mathcal{C}$ operator itself is an observable. 
    \item[(iii)] The $\mathcal{C}$ operator is not realised in standard Hermitian quantum mechanics because, in the limit the Hamiltonian of the system reaches a Hermitian limit, the $\mathcal{C}$ operator becomes identical to the parity operator. Now the parity operator is self-invertible and hence the $\mathcal{CPT}$ conjugate reduces to just $\mathcal{T}$ or ``complex conjugation" ($\mathcal{CPT} \rightarrow \mathcal{PPT} \rightarrow \mathcal{P}^2\mathcal{T} \rightarrow \mathcal{T}$). Complex conjugation is the ``natural duality" assumed in standard quantum mechanics, and hence the $\mathcal{C}$ symmetry goes unnoticed.   
    \item[(iv)] The greatest similarity between the two theories is perhaps that the time evolution or the dynamics of the system is given by $e^{-iHt}$ regardless of the Hamiltonian. In the case of $\mathcal{PT}$-symmetric quantum mechanics, the unitary evolution of the norm is maintained with respect to the $\mathcal{CPT}$ inner product. 
\end{itemize}


%%%%%%%%%%%%%%%%%%%%%%%%%%%%%%%%%%%%%%%%%%%%%%%%%%%%%%%%%%%%%%%%%%%%%%%%%%%%%%%%%


\chapter{Pseudo-Hermiticity: A general framework; Quasi-Hermitian Operators}
\epigraph{\textbf{Where the mind is without fear and the head is held high \\
Where knowledge is free \\
Where the world has not been broken up into fragments \\
By narrow domestic walls}}{\textit{Rabindranath Tagore, Gitanjali: Poem 35}}
\vspace{1cm}

\hspace{0.7cm} We have traversed the vast land of $\mathcal{PT}$-symmetric quantum mechanics in the previous chapter, but the journey is not complete; in fact, it is about to begin. $\mathcal{PT}$-symmetric quantum mechanics, when closely scrutinised, has many loopholes and inconsistencies. This was first pointed out rightfully by Mostafazadeh in a series of papers in 2002 \cite{doi:10.1063/1.1418246,doi:10.1063/1.1461427,doi:10.1063/1.1489072} and became the holy grail of non-Hermitian quantum mechanics. The arguments made by Mostafazadeh were regarding the validity of obtaining real eigenvalues if the Hamiltonian was non-Hermitian and $\mathcal{PT}$-symmetric. It turned out that $\mathcal{PT}$-symmetry was not the genuine rationale behind the reality of eigenstates of non-Hermitian Hamiltonians, and something more general/broader was required. Thus came the need for the concept of \textbf{\textit{Pseudo-Hermiticity}} which essentially encompassed $\mathcal{PT}$-symmetric quantum mechanics. Pseudo-Hermiticity will eventually allow us to move into the time-dependent picture in the next chapter. But before that can happen, we introduce and elaborate in this chapter the shortcomings of $\mathcal{PT}$-symmetric theory, Pseudo-Hermiticity, and quasi-hermiticity. The last idea was well known in the 1990s but needed some tweaks to fit into the non-Hermitian framework \cite{SCHOLTZ199274} and, in due course, became pseudo-Hermiticity as we know it today. Let us begin with the drawbacks of $\mathcal{PT}$-symmetric quantum mechanics.

\section{Shortcomings of $\mathcal{PT}$-symmetric quantum mechanics}
\label{PT-symm shortcomings}

The reality of the spectrum of a non-Hermitian Hamiltonian has been attributed to its $\mathcal{PT}$ symmetry until now. After we are done with this chapter, we will realise that this is not the whole story, and rather a more fundamental property, pseudo-Hermiticity, is at play. This property is the basic structure that is responsible for the real eigenvalues. Mostafazadeh argued that the correlating $\mathcal{PT}$-symmetry and reality of the spectrum is mostly because most non-Hermtian Hamiltonians having real eigenvalues coincidentally have $\mathcal{PT}$-symmetry, but there are $\mathcal{PT}$-symmetric Hamiltonians that do not have real spectrum \cite{doi:10.1063/1.1418246}. Therefore, we can firmly say that $\mathcal{PT}$-symmetry alone does not justify the reality of the spectrum, and we need an alternative hypothesis. Since the ideas of $\mathcal{PT}$-symmetric theory have already been thoroughly discussed in the last chapter, here we will try to build pseudo-Hermiticity and then compare the two. \par

In \cite{CANNATA1998219}, Cannata \textit{et al} describes a class of non-Hermitian and non-$\mathcal{PT}$-symmetric Hamiltonians that have a real spectrum. In this investigation, the authors use the Darboux method to construct complex potentials and then exactly solve their Schr\"{o}dinger equations. In some cases where the potential is not $\mathcal{PT}$-symmetric, the eigenvalues are still real. 

\section{Peudo-Hermiticty: Definition and basic theorems}
\label{pseudo-Hermiticity def and theorems}

\textbf{\textit{Definition}:} \textbf{[}$\bm{\eta}$\textbf{-pseudo-Hermitian operator]} Let $V_{\pm}$ be two complex inner product spaces endowed with Hermitian linear automorphisms $\eta_{\pm}$ which satisfy
\begin{align}
    \label{Lin automorphisms etas for pseudo-H}
    &\nonumber\\
    \forall \, v_{\pm}, \, w_{\pm} \in V_{\pm}, \qquad \left(v_{\pm}, \eta_{\pm} \,  w_{\pm}\right)_{\pm} \, = \, \left(\eta_{\pm} \, v_{\pm}, w_{\pm}\right)_{\pm}.\\
    &\nonumber
\end{align}
 
\begin{align}
    \label{eta pseudo hermitian adjoint}
    &\nonumber\\
    \mathcal{O}^{\#}: V_{-} \rightarrow V_{+} \quad \text{such that,} \quad \mathcal{O}^{\#} \, := \, \eta_{\pm}^{-1} \, \mathcal{O}^\dag \, \eta_{\pm}.\\
    &\nonumber
\end{align}
If a single Hilbert space is used to define the quantum system and one automorphism is considered, i.e.
\begin{align}
    \label{Use single Hilbert space}
    &\nonumber\\
    V_{+} \, = \, V_{-} \, = \, \mathcal{H} \quad \text{and} \quad \eta_\pm \, = \, \eta. \\
    &\nonumber
\end{align}
Then $\mathcal{O}$ is called $\eta$-pseudo-Hermitian if $\mathcal{O}^{\#} = \mathcal{O}$ or more explicitly
\begin{align}
    \label{Pseudo-Hermiticity Condition}
    &\nonumber\\
    \mathcal{O} \, = \, \eta^{-1} \, \mathcal{O}^\dag \, \eta. \\
    &\nonumber
\end{align}
The relation in Eq.(\ref{Pseudo-Hermiticity Condition}) is the crux of this chapter and will ultimately help us introduce time into the picture in the next chapter. \par

\hspace{-1.5em}\textbf{\textit{Definiton}:} \textbf{[pseudo-Hermitian operator]} Linear operator $\mathcal{O}: V \rightarrow V$ on complex inner product space $V$ is called pseudo-Hermitian if $\exists$ a Hermitian linear automorphism $\eta$  such that $\mathcal{O}$ is $\eta$-pseudo-Hermitian (as defined above). \par

Although it is hard to believe, we actually have all the major ingredients to build a quantum theory based on pseudo-Hermiticity. Consider a system with a non-Hermitian time-dependent Hamiltonian $H(t)$ and Hilbert space $\mathcal{H}$. There exists a Hermitian automorphism $\eta$, of the Hilbert space $\mathcal{H}$ onto itself. \par 

\hspace{-1.5em}\textbf{\textit{Theorem 1}:} We define a Hermitian indefinite inner product $\langle\langle \, \mid \, \rangle\rangle_{\eta}$ based on $\eta$ as 
\begin{align}
    \label{Eta indefinite inner product}
    &\nonumber\\
    \left\langle\left\langle\psi_{1} \mid \psi_{2}\right\rangle\right\rangle_{\eta} \, := \, \left\langle\psi_{1}|\eta| \psi_{2}\right\rangle, \quad \forall\left|\psi_{1}\right\rangle,\left|\psi_{2}\right\rangle \in \mathcal{H}.\\
    &\nonumber
\end{align}
This inner product is indefinite, in general, and is invariant under time-translation generated by the Hamiltonian $H$ \textit{iff} $H$ is $\eta$-pseudo-Hermitian. \par

\hspace{-1.5em}\textbf{\textit{Proof}:} If $H$ is $\eta$-pseudo-Hermitian then (rearranging the $\eta$-pseudo-Hermitian condition in Eq.(\ref{Pseudo-Hermiticity Condition}) for $H$)
\begin{align}
    \label{Rearranged pseudo-Hermiticity condition for H}
    &\nonumber\\
    H^{\dagger}\, = \, \eta \, H \, \eta^{-1}. \\
    &\nonumber
\end{align}
Now the Schr\"{o}dinger equation for a general state $|\psi(t)\rangle$ of the system with Hamiltonian $H = H(t)$ is 
\begin{align}
    \label{Schrodinger Time Evolution Equation }
    &\nonumber\\
    i \, \frac{d}{d t}|\psi(t)\rangle \, = \, H \, |\psi(t)\rangle.\\
    &\nonumber
\end{align}
Let us take two state vectors, $\left|\psi_{1}(t)\right\rangle$ and $\left|\psi_{2}(t)\right\rangle$. Evolving them in time using the Schr\"{o}dinger equation and taking their adjoints
\begin{align}
    &\nonumber\\
    i \, \frac{d}{d t}|\psi_{1}(t)\rangle \, = \, H \,|\psi_{1}(t)\rangle \quad \stackrel{\text{DC}}{\longleftrightarrow} \quad -i \, \frac{d}{d t}\langle\psi_{1}(t)| \, = \, \langle\psi_{1}(t)| \, H^{\dagger}, \nonumber\\
    &\nonumber\\
    i \, \frac{d}{d t}|\psi_{2}(t)\rangle \, = \, H \,|\psi_{2}(t)\rangle \quad \stackrel{\text{DC}}{\longleftrightarrow} \quad -i \, \frac{d}{d t}\langle\psi_{2}(t)| \, = \, \langle\psi_{2}(t)| \, H^{\dagger}.
\end{align}
We require that the time derivative of the indefinite inner product be zero or
\begin{align}
    \label{Time derivative of indefinite inner product is zero 1}
    &\nonumber\\
    &i \, \frac{d}{d t}\left(\left\langle\left\langle\psi_1(t) \, | \, \psi_2(t)\right\rangle\right\rangle_\eta\right) \, = \, i \, \frac{d}{d t}\left(\left\langle\psi_1(t)\right| \eta \left|\psi_2(t)\right\rangle\right) \, = \, 0, \nonumber\\
    \nonumber\\
    \text{or,} \quad i &\left(\frac{d}{d t}\left\langle\psi_1(t)\right|\right) \cdot\left(\eta\left|\psi_2(t)\right\rangle\right)+i\left(\left\langle\psi_1(t)\right|\right) \cdot\left(\frac{d}{d t} \, \eta\left|\psi_2(t)\right\rangle\right) \, = \, 0. \\
    &\nonumber
\end{align}
Here, we have assumed that $\eta$ is a time-independent automorphism of $\mathcal{H}$ but this is to make our lives easier. In the next chapter, we will talk about time-dependent backgrounds, where $\eta(t)$ will take center stage. Taking this assumption, we can rewrite Eq.(\ref{Time derivative of indefinite inner product is zero 1})
\begin{align}
    \label{Time derivative of indefinite inner product is zero 2}
    &\nonumber\\
    i \, \frac{d}{d t}\left\langle\left\langle\psi_1 \, | \, \psi_2\right\rangle\right\rangle_\eta \, = \, \left\langle\psi_1(t)\left|\left(\eta H-H^{\dagger} \eta\right)\right| \psi_2(t)\right\rangle \, = \, 0 .\\
    &\nonumber
\end{align}
Now, we have chosen the two evolving state vectors arbitrarily and thus it must be this.
\begin{align}
    \label{The Pseudo-Hermitian Condition is required}
    &\nonumber\\
    &\eta H - H^{\dagger}\eta \, = \, 0, \nonumber\\
    \nonumber\\
    \text{or,} \quad &H \, = \, \eta^{-1} H^{\dagger} \eta. \\
    &\nonumber
\end{align}
We get back the \textbf{\textit{Pseudo-Hermiticity condition}} and, therefore, if the $\eta$ inner product is time-invariant, then the Hamiltonian of the system must be pseudo-Hermitian. \qedsymbol \par
\textbf{\textit{It is easily noticed that if}} $\eta$ \textbf{\textit{is taken to be the identity,}} $\bm{1}$\textbf{\textit{, then we get back the Hermiticity condition of standard quantum mechanics. This shows that we are extending the known quantum mechanics formalism to a more general framework, the former being a special case of pseudo-Hermitian theory.}} \par
It should be clear that $\mathcal{PT}$-symmetry and pseudo-Hermiticity are different properties. One might assume that the connection between pseudo-Hermitian and $\mathcal{PT}$-symmetric quantum mechanics is natural, and the latter might be a particular case of the former. But this is not true at all. Let us establish this with the help of a simple exercise. We take two Hamiltonians: $H_1:=p^2+x^2 p$ and $H_2:=p^2+i\left(x^2 p+p x^2\right)$. $H_1$ is $\mathcal{PT}$-symmetric, while $H_2$ is pseudo-Hermitian with respect to the parity operator.
\begin{align}
    \label{Example of a pseudo-Hermitian Hamiltonian}
    &\nonumber\\
    &H_2 \, = \, p^2+i \left(x^2 \, p+p \, x^2\right), \\
    \nonumber\\
    \implies \qquad \ \ \, &H_2^{\dagger} \, = \, p^2+(-i)\left(x^2 \, p+ x^2\right),\\
    \nonumber\\
    \text{and,} \qquad \mathcal{P} & H_2 \, \mathcal{P}^{-1} \, = \, p^2+i\left(-x^2 \, p-p \, x^2\right) \, = \, H_2^{\dagger}.  \\  
    &\nonumber
\end{align}
But $H_1$ is not $\mathcal{P}$-pseudo-Hermitian and also $H_2$ is not $\mathcal{PT}$-symmetric. Therefore, it should be clear that these two properties are separate concepts (of course $H_1$ can be pseudo-Hermitian with respect to some other linear automorphism $\eta$). They have a subtle relationship, although it has highly impactful consequences, which we will discuss in the next section. \par

In standard Hermitian quantum mechanics, compact results often point to important conclusions. One such result here is 
\begin{align}
    \label{Two eigenvectors eta inner product}
    &\nonumber\\
    \left(E_i^{\, *}-E_j\right)\left\langle\left\langle E_i \mid E_j\right\rangle\right\rangle_\eta \, = \, 0.\\
    &\nonumber
\end{align}
Before moving on to the consequences of Eq.(\ref{Two eigenvectors eta inner product}) let us see how we arrived at this result. We use the pseudo-Hermiticity condition in Eq.(\ref{The Pseudo-Hermitian Condition is required}) in different form, called the \textit{intertwining relation}
\begin{align}
    \label{Intertwining Relation}
    &\nonumber\\
    \eta \, H \, = \, H^{\dagger} \, \eta.\\
    &\nonumber
\end{align}
Using this with the simple eigenvalue equation of the Hamiltonian will give us Eq.(\ref{Two eigenvectors eta inner product}). Eigenvalue equations for two distinct eigenstates of the Hamiltonian $H$ and their conjugates
\begin{align}
    \label{Two eigenstates of H and conjugates}
    &\nonumber\\
    &H\left|E_i\right\rangle \, = \, E_i\left|E_i\right\rangle \quad \stackrel{\text{DC}}{\longleftrightarrow} \quad \left\langle E_i\right| H^{\dagger} \, = \, \left\langle E_i\right| E_i^{\, *}, \nonumber\\
    &\nonumber\\
    &H\left|E_j\right\rangle \, = \, E_j\left|E_j\right\rangle \quad \stackrel{\text {DC}}{\longleftrightarrow} \quad \left\langle E_j\right| H^{\dagger} \, = \, \left\langle E_j\right| E_j^{\, *}. \\
    &\nonumber
\end{align}
Note that since $\eta$ is Hermitian, $\eta \left|E_{i}\right\rangle$ are also eigenvectors of $H^{\dagger}$ with eigenvalue $E_i$. Now applying the $\eta$ operator to the conjugate eigenvalue equation for $\left|E_i\right\rangle$ from the left and then multiplying $\left|E_j\right\rangle$ again from the left we get
\begin{align}
    \label{Distinct Eigenvalue and Eigenvector result}
    &\nonumber\\
    &\left\langle E_i\left|H^{\dagger}  \eta\right|E_j\right\rangle \, = \, E_i^{\, *}\left\langle E_i\left| \, \eta \, \right| E_j\right\rangle \, = \, \left\langle E_i\left|\eta H\right| E_j\right\rangle \, = \, E_j^{\, *}\left\langle E_i\left| \, \eta \, \right| E_j\right\rangle, \nonumber\\
    &\nonumber\\
    \implies &E_j\left\langle E_i\left| \, \eta \,\right| E_j\right\rangle \, = \, E_i^{\, *}\left\langle E_i\left| \, \eta \, \right| E_j\right\rangle \quad \text{or,} \quad \left(E_i^{\, *}-E_j\right)\left\langle E_i\left| \, \eta \, \right| E_j\right\rangle \, = \, 0, \nonumber\\
    &\nonumber\\
    \implies & \qquad \qquad \qquad \quad \quad \ \left(E_i^{\, *}-E_j\right)\left\langle\left\langle E_i | E_j\right\rangle\right\rangle_\eta \, = \, 0.\\
    &\nonumber
\end{align}
Eq.(\ref{Distinct Eigenvalue and Eigenvector result}) allows us to conjecture new propositions such as the following.\par

\hspace{-1.5em}\textbf{\textit{Theorem 2}:} A system that has a non-Hermitian Hamiltonian that is $\eta$-pseudo-Hermitian has the following properties, consequences of Eq.(\ref{Distinct Eigenvalue and Eigenvector result}):

\begin{itemize}
    \item[(i)] Eigenvectors that have complex eigenvalues (in $\mathcal{PT}$ symmetry then belong to the \textit{unbroken} regime) have vanishing $\eta$ norms or essentially $\eta$ seminorms (see Appendix \ref{Seminorm illustration}). Say, we take an eigenvector $\left|E_i\right\rangle$ of $H$ then, $H\left|E_i\right\rangle=E_i\left|E_i\right\rangle$ s.t. $E_i \notin \mathbb{R}$ then $E_i^{\, *}-E_i=-2\, i \, Im\left(E_i\right)$. As a result, we must have
        \begin{align}
        \label{vanishing eta seminorm}
        &\nonumber\\
        &\left(E_i^{\, *}-E_i\right)\left\langle\left\langle E_i | E_i\right\rangle\right\rangle_\eta \, = \, 0 \implies-2 i \, Im\left(E_i\right)\left\langle\left\langle E_i | E_i\right\rangle\right\rangle_\eta \,= \, 0 \nonumber\\
        &\implies \left\langle\left\langle E_i \mid E_i\right\rangle\right\rangle_\eta \, = \, \left(\left\|E_i\right\|_\eta\right)^2 \, = \, 0 \implies\left\|E_i\right\|_\eta \, = \, 0.\\
        &\nonumber
        \end{align}
    \item[(ii)] Two eigenvectors with eigenvalues that are complex conjugates of each other (\textit{just like in the unbroken} $\mathcal{PT}$\textit{-symmteric region}) are not $\eta$-orthogonal,
        \begin{align}
        \label{Two eigenvectors are eta orthogonal}
        &\nonumber\\
        E_i \neq E_j^{\,*} \implies \left\langle\left\langle E_i | E_j\right\rangle\right\rangle_\eta \, = \, 0.\\
        &\nonumber
        \end{align}
        Any two eigenvectors with non-degenerate and real eigenvalues must be $\eta$-orthogonal. 
\end{itemize}

Finally, we begin to see that a formalism based on the $\eta$ inner product is now taking shape. Soon, we will have a fully consistent quantum theory with theorems that are analogous to conventional quantum mechanics. \par

A vector space endowed with such an $\eta$ inner product must have fundamental properties that describe elements of the space. The following theorem explains these aspects. \par

\hspace{-1.5em}\textbf{\textit{Theorem 3}:} Let $\bm{V}$ be an inner product space endowed with a linear Hermitian automorphism $\eta$. Let the identity operator in this space be denoted by $\bm{1}: \bm{V} \rightarrow \bm{V}$, $O_1$, $O_2: \bm{V} \rightarrow \bm{V}$ be linear operators, and $z_1$,$z_2 \in \mathbb{C}$. Then the following properties must hold.
\begin{itemize}
    \item[(i)] $\bm{1}^{\, \#} = \bm{1}$, 
    \item[(ii)] $\left(O_1^{\, \#}\right)^{\, \#} = O_1$,
    \item[(iii)] $\left(z_1 \, O_1 + z_2 \, O_2\right)^{\, \#} = z_1^{\, *} \, O_1^{\, \#} + z_2^{\, *} \, O_2^{\, \#}$,
\end{itemize}
where $z_i^{\, *}$ denotes the complex conjugation of $z_i$. \par

\hspace{-1.5em}\textbf{\textit{Proof}:} The definition of $\eta$-pseudo-Hermitian adjoint makes (i) and (iii) trivial consequences. As for (iii), a simple calculation will verify the property.
\begin{align}
    &\nonumber\\
    \left(z_1 \, O_1+z_2 \, O_2\right)^{\, \#} \, = \, \eta^{-1}\left(z_1 \, O_1+z_2 \, O_2\right)^{\dagger} \eta \, = \, z_1^{\, *} \, \eta^{-1} \, O_1^{\dagger} \, \eta+z_2^{\, *} \, \eta^{-1} \, O_2^{\dagger} \, \eta \, = \, z_1^{\, *} \, O_1^{\, \#}+z_2^{\, *} \, O_2^{\, \#}. \nonumber \qed
\end{align}
\hspace{1.5em} Composition of linear transformations in pseudo-Hermitian quantum mechanics is similar to standard Hermitian quantum mechanics. \par

\hspace{-1.5em}\textbf{\textit{Theorem 4}:} Let there be three inner product spaces $\bm{V_i}$ such that $i \in {1,2,3}$, each endowed with Hermitian linear automorphisms $\eta_i$. If $O_1: \bm{V_1} \rightarrow \bm{V_2}$ and $O_2: \bm{V_2} \rightarrow \bm{V_3}$ are linear operators. Then, the pseudo-Hermitian adjoint of the composition of both has the property:
\begin{align}
    \label{PHA Composition of Linear Operators}
    &\nonumber\\
    (O_2 \, O_1)^{\#} \, = \, O_1^{\, \#} \, O_2^{\, \#}. \\
    &\nonumber
\end{align}
\textbf{\textit{Proof}:} A single-line calculation will verify this theorem:
\begin{align}
    &\nonumber\\
    \left(O_2 \, O_1\right)^{\#} \, = \, \eta_1^{-1} \left(O_2 \, O_1\right)^{\dagger}\eta_3 \, = \, \eta_1^{-1} \, O_1^{\dagger} \, \eta_2 \, \eta_2^{-1} \, O_2^{\dagger} \, \eta_3 \, = \, O_1^{\, \#} \, O_2^{ \, \#}. \qquad \text{\qedsymbol} \nonumber\\ 
    &\nonumber
\end{align}
An important aspect of quantum study is to look at unitary transformations that essentially rotate the system in the state space or Hilbert space. In standard quantum mechanics, the Hermiticity is invariant under a unitary transformation. We have something similar in pseudo-Hermitian quantum mechanics. The following theorem elucidates our claim.\par

\hspace{-1.5em}\textbf{\textit{Theorem 5}:} Let $\bm{V}$ be an inner product space endowed with a Hermitian linear automorphism $\eta$. A unitary operator in this space, $U: \bm{V} \rightarrow \bm{V}$, exists, and we also have a general linear operator $O: \bm{V} \rightarrow \bm{V}$. We define $\eta_{_U} := U^{\dagger} \, \eta \, U$ which is also a Hermitian linear automorphism. Then $O: \bm{V} \rightarrow \bm{V}$ is $\eta$-pseudo-Hermitian iff its unitary transformation $O_U := U^{\dagger} \, O \, U$ is $\eta_{_U}$-pseudo-Hermitian or pseudo-Hermiticity is unitary-invaraint. \par 
\hspace{-1.5em}\textbf{\textit{Proof}:} First, we check the Hermiticity of $\eta_{_U}$.
\begin{align}
    \label{Eta_U is Hermitian}
    &\nonumber\\
    \eta_{_U}^{\dagger} \, = \, U^{\dagger} \, \eta^{\dagger}\left(U^{\dagger}\right)^{\dagger} \, = \, U^{\dagger} \, \eta \, U \, = \, \eta_{_U}. \qquad \text{[Since, $\eta^{\dagger} \, = \, \eta$]}\\
    &\nonumber
\end{align}
Now that we know that $\eta_{_U}$ is Hermitian, we can move on to our main theorem. But before that we need to determine the inverse of $\eta_{_U}$.
\begin{align}
    \label{Inverse of Eta_U}
    &\nonumber\\
    \eta_{_U}^{-1} \, = \, \left(U^{\dagger} \, \eta \, U\right)^{\, -1} \, = \, U^{-1} \, \eta^{-1} \, \left(U^{\dagger}\right)^{-1} \, = \, U^{\dagger} \, \eta^{-1} \, U.\\
    &\nonumber
\end{align}
Now, $O_U^{\, \#} \, = \, \eta^{-1}_{_U} \, O_{U}^{\dagger} \, \eta_{_U}$ i.e. pseudo-Hermitian with respect to $\eta_{_U}$. Therefore,
\begin{align}
    \label{O_U is eta pseudo Hermitian iff O is eta Hermitian}
    &\nonumber\\
    O_{U}^{\, \#} \, = \, \eta_{_U}^{\, -1}(U^{\dagger} \, O \, U)^{\dagger} \, \eta_{_U} \, &= \, \eta_{_U}^{-1} \left(U^{\dagger} \, O^{\dagger} \, U\right) \eta_{_U} \, = \, U^{\dagger} \, \eta^{-1} \, U \left(U^{\dagger} \, O^{\dagger} \, U\right) U^{\dagger} \, \eta \, U  \nonumber\\
    &\nonumber\\
    &= \, U^{\dagger} \left(\eta^{-1} \, O^{\dagger} \, \eta\right)U \qquad \text{[ Since $U$ is unitary ]}. \\
    &\nonumber
\end{align}
We see $O$ is $\eta$-pseudo-Hermitian iff $O_{U}^{\#} = O_{U}$ or equivalently $O = \eta^{-1} \, O^{\dagger} \, \eta$, the $\eta$-pseudo-Hermitian condition must be true. \qedsymbol \par

\hspace{-1.5em}Theorem 5 can also be interpreted as a deduction that unitarity is maintained in pseudo-Hermitian quantum mechanics, just as in the ordinary one. So, if $O = \eta^{-1} \, O^{\dagger} \, \eta$ then we have
\begin{align}
    \label{Pseudo Hermitian quantum mechanics is unitary invariant}
    &\nonumber\\
    U^{\dagger} \, O \, U \, = \, \left(U^{\dagger} \, \eta \, U\right)^{-1}\left(U^{\dagger} \, O \, U\right)^{\dagger}\left(U^{\dagger} \, \eta \, U\right).\\
    &\nonumber
\end{align}
A unitary transformation of the Hermitian linear automorphism, $\eta$ and also of the linear operator $O$ does not affect their pseudo-Hermitian association. Therefore, a unitary transformation or effectively a rotation of the system leaves the pseudo-Hermitian properties intact. \par

The keen observer must have recognised and asked a very simple question: \textit{What if a single linear operator is pseudo-Hermitian with respect to two different Hermitian linear automorphisms?} Sounds like a love triangle! In fact, this is a very real possibility, and the next theorem will help us to understand the significance of such a situation. \par

\hspace{-1.5em}\textbf{\textit{Theorem 6}:} Let $\bm{V}$ be an inner product space endowed with two Hermitian linear automorphisms $\eta_1$ and $\eta_2$. Let $O: \bm{V} \rightarrow \bm{V}$ be a linear operator, then $O$ is $\eta$-pseudo-Hermitian with respect to both $\eta_1$ and $\eta_2$ iff $\eta_2^{-1} \, \eta_1$ commutes with $O$. \par

\hspace{-1.5em}\textbf{\textit{Proof}:} If $\eta_1^{-1} \, O^{\dagger} \, \eta_1 \, = \, \eta_2^{-1} \, O^{\dagger} \, \eta_2$, then 
\begin{align}
    \label{}
    &\nonumber\\
    &O^{\dagger} \, \eta_1 \, \eta_2^{-1} \, = \, \eta_1 \, \eta_2^{-1} \, O^{\dagger} \implies \left(O^{\dagger} \, \eta_1 \, \eta_2^{-1}\right)^{\dagger} \, = \, \left(\eta_1 \, \eta_2^{-1} \, O^{\dagger}\right)^{\dagger} \nonumber\\
    \implies \left(\eta_2^{-1}\right)^{\dagger} \eta_1^{\dagger} \, O \, &= \, O \left(\eta_2^{-1}\right)^{\dagger} \eta_1^{\dagger} \implies \eta_2^{-1} \, \eta_1 \, O \, = \, O \, \eta_2^{-1} \, \eta_1 \quad \text{[ Since $\eta$ is Hermitian]} \nonumber\\
    &\implies \qquad \quad \left[O,\eta_2^{-1} \,\eta_1\right] \, = \, 0. 
    \nonumber \qed
\end{align}
So, when we have an actual system with a working Hamiltonian $H$ then if this $H$ is pseudo-Hermitian with respect to two distinct $\eta_1$, and $\eta_2$ we have $\eta_2^{-1} \, \eta_1$ as a symmetry of the system. \par

Probabilistic interpretation of Hermitian systems gives the notion of observing a particular eigenstate and the wave function collapsing to that state so that further transition is not possible. The orthogonality of eigenstates of a Hermitian operator captures this scheme very well. Until now, our discussion has been solely focused on the Hamiltonian of the system, but we have ignored the basic eigenstate structure of the space that a non-Hermitian Hamlitonian can pose. The eigenstates produced by a non-Hermitian Hamiltonian may not be orthogonal and yet be a real and complete set. We have, from the very beginning, emphasised the fact that the reality of a spectrum is more than enough for a candidate Hamiltonian to be accepted physically. In the following section, we will illuminate the core structure of a space corresponding to a non-Hermitian Hamiltonian.

\section{What happens when the eigenstates have real eigevalues, and form a complete set but are not orthogonal?}
\label{Biorthogonal Eigenbasis for a pesudo-Hermitian Hamiltonian}

The physical viability of a quantum theory essentially depends on two factors: The eigenvalues are real, and the set of eigenstates is complete. It is therefore tolerable to relax the notion of orthogonality and thus replace it with a more general concept of biorthogonality. We will talk more about biorthogonal sets of eigenbasis in the next chapter, and for now we assume that it will make our lives easier to introduce a biorthogonal set of eigenbasis for a $\eta$-pseudo-Hermitian Hamiltonian with a discrete spectrum. 
\begin{align}
    \label{Biorthogonal set of eigenbasis for a eta-pseudo-Hermitian H}
    &\nonumber\\
    H\left|\psi_n, a\right\rangle=E_n\left|\psi_n, a\right\rangle, \quad H^{\dagger}\left|\phi_n, a\right\rangle=E_n^{\, *}\left|\phi_n, a\right\rangle. \\
    &\nonumber
\end{align}
Here, $E_n$'s are the eigenvalues and $a$ is a degeneracy label to keep things as general as possible. Eq.(\ref{Biorthogonal set of eigenbasis for a eta-pseudo-Hermitian H}) defines a biorthognal basis, but the realisation of orthogonality and completeness is understood through the delta function and identity operator relations. 
\begin{align}
    \label{Delta function and identity relation}
    &\nonumber\\
    \left\langle\phi_m, b | \psi_n, a\right\rangle \, &= \, \delta_{m n} \, \delta_{a b}, \\
    &\nonumber\\
    \sum_n \sum_{a \, = \,1}^{d_n} \left|\phi_n, a\right\rangle \left\langle\psi_n, a\right| \, &= \,\sum_n \sum_{a \, = \,1}^{d_n}\left| \psi_n, a\right\rangle\left\langle\phi_n, a\right| \, = \, \bm{1}. \\
    &\nonumber
\end{align}
The $d_n$'s represent the multiplicity or dimension of the set of eigenstates with identical eigenvalue, and $b$, similarly, is a degeneracy label. \par

We now present a theorem (and, of course, prove it!) that will finally put everything into perspective \cite{doi:10.1063/1.1418246}. Until now, there has been no connection between the last chapter and our present study. This theorem will bridge the gap between $\mathcal{PT}$-symmetry and pseudo-Hermitian quantum mechanics thus vindicating our methods. \par

\hspace{-1.5em}\textbf{\textit{Theorem 7}:} Let $H$ be a pseudo-Hermitian Hamiltonian with a biorothonormal eigenbasis. Then the complex eigenvalues of $H$ appear in complex conjugate pairs with the same multiplicity. \par

\hspace{-1.5em}\textbf{\textit{Proof}:} Using Eq.(\ref{Intertwining Relation}) and Eq.(\ref{Biorthogonal set of eigenbasis for a eta-pseudo-Hermitian H}), we have 
\begin{align}
    \label{Eta-inverse is a map from En to En*}
    &\nonumber\\
    H \, \eta^{-1}\left|\phi_n, a\right\rangle \, = \, \eta^{-1} H^{\dagger}\left|\phi_n, a\right\rangle \, = \, E_n^{\, *} \, \eta^{-1}\left|\phi_n, a\right\rangle. 
\end{align}
Since $\eta^{-1}$ is an invertible operator and $\left|\phi_n, a\right\rangle$ is arbitrary, their product does not vanish. Therefore, we see that $\eta^{-1}\left|\phi_n, a\right\rangle$ is also an eigenvector of $H$ with an eigenvalue of $E_n^{\, *}$. We have a sort of correspondence between $\eta^{-1} \left|\phi_n,a \right\rangle$' s and $\left|\psi_n,a\right\rangle$' s in the sense that $\eta^{-1}$ creates a map between subspaces of eigenvectors with eigenvalues $E_n$'s and $E_n^{\, *}$'s. Since $\eta^{-1}$ is invertible, this map is one-to-one and thus both eigensubspaces have the same multiplicity.
\begin{align}
    &\nonumber\\
    \left|\psi_n, a\right\rangle \, &\longleftrightarrow \, \eta^{-1}\left|\phi_n, a\right\rangle, \nonumber\\
    &\nonumber\\
    E_n \, &\longleftrightarrow \, E_n^{\, *}. \quad \qed \nonumber\\
    &\nonumber
\end{align}
The above proof divides the Hilbert space for a non-Hermitian Hamiltonian into three parts: a subspace of eigenvectors with real eigenvalues, and two subspaces of eigenvectors with eigenvalues that are complex conjugate pairs (one subspace with eigenvalue $E_n$ and the other one with $E_n^{\, *}$. Let us survey this eigenspace in the context of Eqs.(\ref{Biorthogonal Eigenbasis for a pesudo-Hermitian Hamiltonian})-(\ref{Eta-inverse is a map from En to En*}) to get a more deeper perspective of the underlying structure. \par

We denote the real eigenvalues and associated eigenvectors using the subscript `$_0$'. Complex eigenvalues and their conjugates are represented by `$\pm$', i.e. two separate spaces of eigenvectors mapped one-to-one by $\eta^{-1}$. Then we can write the completeness relation and the Hamiltonian for the system as
\begin{align}
    &\nonumber\\
    \bm{1} \, &= \, \sum_{n_0} \sum_{a \, = \, 1}^{d_{n_0}}|\psi_{n_0}, a\rangle\langle\phi_{n_0}, a|+\sum_{n_{+}} \sum_{\alpha \, = \, 1}^{d_{n_{+}}}(|\psi_{n_{+}}, \alpha\rangle\langle\phi_{n_{+}}, \alpha|+| \psi_{n_{-}}, \alpha\rangle\langle\phi_{n_{-}}, \alpha|), \label{Completeness relation with 3 diff eigenspaces}\\
    \nonumber\\
    H \, = \, \sum_{n_0}& \sum_{a \, = \, 1}^{d_{n_0}} E_{n_0}|\psi_{n_0}, a\rangle\langle\phi_{n_0}, a|+\sum_{n_{+}} \sum_{\alpha \, = \, 1}^{d_{n_{+}}}(E_{n_{+}}|\psi_{n_{+}}, \alpha\rangle\langle\phi_{n_{+}}, \alpha|+ \, E_{n_{+}}^{\, *}| \psi_{n_{-}}, \alpha\rangle\langle\phi_{n_{-}}, \alpha|). \label{Hamiltonian with 3 diff eigenspaces}\\
    &\nonumber
\end{align}
In Eq.(\ref{Eta-inverse is a map from En to En*}) we find the role of $\eta^{-1}$ but to find its matrix representation, a similar calculation is performed using the above Hamiltonian. We already know
\begin{align}
    \label{H acting on eta-inverse phi_n0}
    &\nonumber\\
    H \eta^{-1}|\phi_{n_0^{\prime}}, a\rangle \, = \, \eta^{-1} H^{\dagger}|\phi_{n_0^{\prime}}, a\rangle \, = \, E_{n_0^{\prime}}(\eta^{-1} |\phi_{n_0^{\prime}}, a\rangle) \quad [ \, \because E_{n_0^{\prime}} \in \mathbb{R} \, ]. \\
    &\nonumber
\end{align}
Using the Hamiltonian in Eq.(\ref{Hamiltonian with 3 diff eigenspaces}) on $\eta^{-1}|\phi_{n_0^{\prime}}, a\rangle$ and equating it to Eq.(\ref{H acting on eta-inverse phi_n0}) we get
\begin{align}
    \label{Equating Summation form H and algebraic H 1}
    &\nonumber\\
    \sum_{n_0} \sum_{b \, = \, 1}^{d_{n_0}} E_{n_0}|\psi_{n_0}, b \rangle\langle \phi_{n_0}, b| \eta^{-1}|\phi_{n_0^{\prime}}, a\rangle + \sum_{n_{+}} \sum_{\beta \, = \, 1}^{d_{n_{+}}} E_{n_{+}}|\psi_{n_{+}}, \beta \rangle\langle \phi_{n_{+}}, \beta|\eta^{-1} \nonumber\\
    |\phi_{n_0^{\prime}}, a\rangle+E_{n_{+}}^{\, *}|\psi_{n_1}, \beta \rangle\langle \phi_{n_{-}}, \beta| \eta^{-1}|\phi_{n_0^{\prime}}, a\rangle \,= \, E_{n_0^{\prime}}(\eta^{-1}|\phi_{n_0^{\prime}}, a\rangle.\\
    &\nonumber
\end{align}
This means that all $E_{n_{\pm}}$'s are zero along with all $E_{n_{0}}$'s that are not equal to $E_{n_{0}^{\prime}}$. The sum on the left-hand side that survives corresponds to the subspace with eigenvalue $E_{n_{0}^{\prime}}$ and multiplicity $d_{n_{0}^{\prime}}$. And we have
\begin{align}
    \label{Final expression after equating summation H and algebraic H 1}
    &\nonumber\\
    \sum_{b \, = \, 1}^{d_{n_0^{\prime}}} E_{n_0^{\prime}}(\langle\phi_{n_0^{\prime}}, b|\eta^{-1}| \phi_{n_0^{\prime}}, a\rangle)|\psi_{n_0^{\prime}}, b\rangle \,= \, E_{n_0^{\prime}}(\eta^{-1}|\phi_{n_0^{\prime}}, a\rangle)\\
    &\nonumber\end{align}
Removing the primes and exchanging the dummy variables $a$ and $b$ we get a clean expression.
\begin{align}
    \label{c_ab expression for E_n0}
    &\nonumber\\
    \eta^{-1}\left|\phi_{n_0}, a\right\rangle=\sum_{b=1}^{d_{n_0}} c_{b a}^{\left(n_0\right)}\left|\psi_{n_0}, b\right\rangle, \quad c_{a b}^{\left(n_0\right)}:=\left\langle\phi_{n_0}, a\left|\eta^{-1}\right| \phi_{n_0}, b\right\rangle.\\
    &\nonumber
\end{align}
$c_{ab}^{(n_0)}$ represents a matrix with complex entries and captures the essence of the relation that $\eta^{-1}$ forms in the various eigensubspaces. Similar matrices can be calculated for the complex eigensubspaces represented by the subscripts `$\pm$' and are shown in Appendix \ref{Calculation for c_alphabeta}. We will use these results to eventually find a matrix representation of $\eta$, thus generalising a method to find $\eta$ for any non-Hermitian Hamiltonian. \par

Since $\eta$ is a Hermitian linear automorphism, according to Eq.(\ref{c_ab expression for E_n0}), Eq.(\ref{c_ab expression for E_n-}), and Eq.(\ref{c_ab expression for E_n+}), the $c^(n_0)$ and $c^(n_\pm)$ matrices must also be Hermitian. This allows us to perform unitary transitions on the space such that these matrices become diagonal, and then rescale them to identity. Then these three equations, i.e. Eq.(\ref{c_ab expression for E_n0}), Eq.(\ref{c_ab expression for E_n-}), and Eq.(\ref{c_ab expression for E_n+}), take the form
\begin{align}
    &\nonumber\\
    \eta^{-1}\left|\phi_{n_0}, a\right\rangle\, = \, \left|\phi_{n_0}, a\right\rangle \quad &\text{or} \quad \left|\psi_{n_0}, a\right\rangle \, = \, \eta\left|\psi_{n_0}, a\right\rangle \text {, } \label{Final form of c matrices after u trans and rescalling 1}\\
    &\nonumber\\
    \eta^{-1} \, |\phi_{n_\pm}, \alpha\rangle \, = \, |\phi_{n_\pm}, \alpha\rangle \quad &\text{or} \quad |\psi_{n_\pm}, \alpha\rangle \, = \, \eta \, |\psi_{n_\pm}, \alpha\rangle. \label{Final form of c matrices after u trans and rescalling 2} \\
    &\nonumber
\end{align}
In Appendix \ref{Calculation for c_alphabeta} we have mentioned the fact that $\eta$ essentially defines the structure of the space by forming relationships between the eigensubspaces we have discussed. Now using Eq.(\ref{Delta function and identity relation}), Eq.(\ref{Final form of c matrices after u trans and rescalling 1}), and Eq.(\ref{Final expression after equating summation H and algebraic H 2}) the $\eta$-orthonormalisation conditions for the eigenvectors become
\begin{align}
    &\nonumber\\
    \left\langle\psi_{n_0}, a | \phi_{m_0}, b\right\rangle \, = \, \left\langle\psi_{n_0}, a| \, \eta \,| \psi_{m_0}, b\right\rangle \, = \, \left\langle\left\langle\psi_{n_0}, a | \psi_{m_0, b}\right\rangle\right\rangle_\eta \, = \, \delta_{n_0, \, m_0} \, \delta_{a_b}, \label{n_0 eta-orthonormalisation conditions} \\
    \nonumber\\
    \langle\psi_{n_{\pm}}, \alpha | \phi_{m_{\mp}}, \beta\rangle  \, = \, \langle\psi_{n_{\pm}}, \alpha| \, \eta \, | \psi_{m_{\mp}}, \beta\rangle \, = \, \langle\langle\psi_{n_{\pm}}, \alpha | \psi_{m_{\mp}}, \beta\rangle\rangle_\eta \, = \, \delta_{n_{\pm}, \, m_{\mp}} \, \delta_{\alpha \beta}. \label{n_pm eta-orthonormalisation conditions} \\
    &\nonumber
\end{align}
A definite form for $\eta$ for a Hamiltonian with biorthonormal eigenbasis can be found by solving Eq.(\ref{Final form of c matrices after u trans and rescalling 1}) and Eq.(\ref{Final form of c matrices after u trans and rescalling 2}), then substituting these results in the completeness relation Eq.(\ref{Completeness relation with 3 diff eigenspaces}). Thus, finally, we get the generalised expressions for $\eta$ and its inverse $\eta^{-1}$.
\begin{align}
    \label{eta in generalised form}
    &\nonumber\\
    \eta \, &= \, \sum_{n_0} \sum_{a \, = \, 1}^{d_{n_0}} \eta \, |\psi_{n_0}, a\rangle\langle\phi_{n_0}, a|+\sum_{n_{+}} \sum_{\alpha \, = \, 1}^{d_{n_{+}}} (\eta \, |\psi_{n_{+}}, \alpha\rangle\langle\phi_{n_{+}}, \alpha|+ \eta \,| \psi_{n_{-}}, \alpha\rangle\langle\phi_{n_{-}}, \alpha|) \nonumber\\
    &= \, \sum_{n_0} \sum_{a=1}^{d_{n_0}}|\phi_{n_0}, a\rangle\langle\phi_{n_0}, a|+\sum_{n_{+}} \sum_{\alpha=1}^{d_{n_{+}}}(|\phi_{n_{-}}, \alpha\rangle\langle\phi_{n_{+}}, \alpha|+| \phi_{n_{+}}, \alpha\rangle\langle\phi_{n_{-}}, \alpha|)\\
    &\nonumber
\end{align}
and consequently the inverse is given by 
\begin{align}
    \label{eta-inverse in generalised form}
    &\nonumber\\
    \eta^{-1} \, &= \, \sum_{n_0} \sum_{a \, = \, 1}^{d_{n_0}} \eta^{-1} \, |\psi_{n_0}, a\rangle\langle\phi_{n_0}, a|+\sum_{n_{+}} \sum_{\alpha \, = \, 1}^{d_{n_{+}}} (\eta^{-1} \, |\psi_{n_{+}}, \alpha\rangle\langle\phi_{n_{+}}, \alpha|+ \eta^{-1} \,| \psi_{n_{-}}, \alpha\rangle\langle\phi_{n_{-}}, \alpha|) \nonumber \\
    &= \, \sum_{n_0} \sum_{a=1}^{d_{n_0}}|\psi_{n_0}, a\rangle\langle\psi_{n_0}, a|+\sum_{n_{+}} \sum_{\alpha=1}^{d_{n_{+}}}(|\psi_{n_{-}}, \alpha\rangle\langle\psi_{n_{+}}, \alpha|+| \psi_{n_{+}}, \alpha\rangle\langle\psi_{n_{-}}, \alpha|). \\
    &\nonumber
\end{align}
\subsection{Connection between $\mathcal{PT}$-symmetry and pseudo-Hermiticity}

The above results for the explicit form of $\eta$ and $\eta^{-1}$ are crucial to the understanding of the composition of eigenvalues for a non-Hermitian Hamiltonian. In Theorem 7 we have proved that any non-Hermitian Hamiltonian with a biorthonormal eigenbasis and a discrete specturm must have eigenvalues are real and complex as well, where the complex ones appear in conjugate pairs with the same multiplicity \textbf{(\textit{region of broken}} $\bm{\mathcal{PT}}$\textbf{\textit{-symmetry})}. \par

In this section, above, we have just learnt that if a non-Hermitian Hamiltonian has a biorthonormal eigenbasis then we can express it in the form in Eq.(\ref{Hamiltonian with 3 diff eigenspaces}) such that it is pseudo-Hermitian with respect to the explicit forms of $\eta$ and $\eta^{-1}$ in Eq.(\ref{eta in generalised form}) and in Eq.(\ref{eta-inverse in generalised form}) respectively. The pseudo-Hermitian condition, Eq.(\ref{The Pseudo-Hermitian Condition is required}), for these explicit forms can be checked (due to the calulation being extremely long, it is not shown in this thesis, but anyone willing to see it can ask the author in a personal capacity). \par

\hspace{-1.5em}\textbf{\textit{Theorem 8}:} A non-Hermitian Hamiltonian with a discrete spectrum and a complete biorthonormnal set of eigenvectors is pseudo-Hermitian iff one of the conditions holds true
\begin{itemize}
    \item[(i)] The spectrum of the Hamiltonian is real.
    \item[(ii)] The nonreal or complex eigenvalues appear in complex conjugate pairs with the same multiplicity.
\end{itemize}
\par

\hspace{-1.5em}\textbf{\textit{Proof}:} The above arugments should make it clear that conditions (i) and (ii) are sufficient. This is because the explicit forms of the Hamiltonian and the Hermtian linear automorphism by construction satisfy Eq.(\ref{The Pseudo-Hermitian Condition is required}). \qedsymbol \par

A word of caution: Many non-Hermitian Hamiltonians may not have a biorthonomal set of eigenvectors. In that case, they may or may not be pseudo-Hermitian. \par 

\hspace{-1.5em}\textbf{\textit{Corollary}:} Every $\mathcal{PT}$-symmetric Hamiltonian with a discrete eigenvalue spectrum and a complete biorthonormal set of eigenvectors is always pseudo-Hermitian.\par

\hspace{-1.5em}\textbf{\textit{Proof}:} Let $H$ be a $\mathcal{PT}$-symmetric Hamiltonian then $H$ must commute with $\mathcal{PT}$, i.e. $\left[H, \mathcal{P T}\right] \, = \, \left[\mathcal{P T}, H\right] \, = \, 0$. Let $|E\rangle$ be an eigenvector of $H$ with eigenvalue $E$ then 
\begin{align}
    &\nonumber\\
    H \, |E\rangle \, = \,  E \, |E\rangle. \nonumber\\
    &\nonumber
\end{align}
We define $|E\rangle^{\prime} := \mathcal{PT} \, |E\rangle$. Then, since $H$ and $\mathcal{PT}$ commute, we have
\begin{align}
    &\nonumber\\
    H \, |E\rangle^{\prime} \, =& \, H \mathcal{PT} \, |E\rangle \, = \, \mathcal{PT} H \, |E\rangle \, = \, \mathcal{PT} E \, |E\rangle \, = \, E^{\, *} \, \mathcal{PT} \, |E\rangle \, = \, E^{\, *} \, |E\rangle^{\prime}, \nonumber\\
    &\implies |E\rangle^{\prime} \text{ is a eigenvector of } H \text{ with eigenvalue } E^{\, *}. \nonumber\\
    &\nonumber
\end{align}
Here, $\mathcal{T}$ is the complex conjugation operator and is therefore antilinear. But $H|E\rangle^{\prime} \, = \, H \, \mathcal{P T} \, |E\rangle \, = \, \mathcal{P T} H|E\rangle \, = \, \mathcal{P T} H(\mathcal{P T})^2|E\rangle \, = \, (\mathcal{P T}) H(\mathcal{TP)} \mathcal{P T} \, |E\rangle \, = \, \mathcal{P} H^{\dagger} \, \mathcal{P} \, |E\rangle^{\prime} \, = \, E^{\, *}|E\rangle^{\prime}$. Thus, we see that $H$ must be pseudo-Hermitian with respect to the $\mathcal{P}$ operator, and the biorthonormal structure arises naturally. \qedsymbol \par

In the following section, we will briefly elaborate some deeper insights of the biorthonormal system, which will tell us the reason for choosing such a basis and shed light on the matter of interpretation of operators as observables.  

\section{A primer on Hilbert Spaces with Biorthogonal Basis}

In the last section, we have extensively used the concept of a biorthonormal eigenbasis to give an underlying framework of pseudo-Hermitian quantum mechanics. Yet, it lacked the interpretation of a lot of fundamental concepts in quantum mechanics; the simplest example would be the probabilistic characterisation of observables. Biorthonormality requires a much more detailed illustration, for reasons which will soon be clear, and thus we dedicate a separate section on its development. \par

The idea of orthogonality was relaxed in order to reinforce pseudo-hermiticity with a proper foundation, but it must be clear that biorthogonality is fundamentally sensible. The probabilistic interpretation of standard quantum mechanics and the idea that once a state has been observed, it collapses to a definite value given by the norm squared of the state where any further transition is not allowed must be respected when building a quantum theory. To be consistent with this interpretation, the Hilbert space considered must consist of square-integrable functions with a complete basis, while operators must have real eigenvalues. The collapse of the wave function and the obstruction of further transitions are encrypted in the orthogonality of eigenstates, and Hermitian operators encode this by default. \par

The core idea of biorthogonal quantum mechanics involves the suspension of orthogonality and replacing it with biorthogonality. Complex non-Hermitian Hamiltonians that we have studied until now are perfect candidates for this. The eigenstates of these non-Hermitian Hamiltonians, being biorthogonal, are maximally separated in the ray space, thus making transitions after a measurement impossible. A complete set of eigenvectors with real eigenvalues is enough to make an operator a viable candidate for a physical observable. \par  

The work by Scholtz \textit{et al} \cite{SCHOLTZ199274} was one of the first papers to propose a metric operator approach to non-Hermitian quantum mechanics. We will briefly discuss this initial work at the end of this chapter. The idea of using a biorthogonal basis to explain quantum mechanics with operators that are not Hermitian was proposed as early as 1838 by Liouville \cite{Liouville1838} and in 1908 by Birkhoff \cite{10.2307/1988661}. Anna Johson Pell\footnotemark\footnotetext{Anna Johnson Pell was an American mathematician. She is best known for her early work on linear algebra in infinite dimensions, which has later become a part of functional analysis. To read about her inspiring story as a woman mathematician, visit the Wikipedia page on her: \href{https://en.wikipedia.org/wiki/Anna_Johnson_Pell_Wheeler}{Anna Johnson Pell Wheeler}.}, a distinguished woman of mathematics, has described in detail the properties of biorthogonal bases in real Hilbert spaces \cite{pell1911applications,10.2307/1988573}. In recent times, the primary work on biorthogonal bases was established by Curtright and Mezincescu \cite{curtright2007biorthogonal}. Mostafazadeh in 2010 published an almost exhaustive work on biorthogonal pseudo-Hermitian systems \cite{mostafazadeh2010pseudo}. However, a much more rigorous formulation was required that would make the idea of measurement processes and their probabilistic interpretations transparent. This was done by Brody \cite{Brody_2014} in 2014 and we will mostly devote this section to the salient features of pseudo-Hermitian biorhtogonal quantum mechanics of this paper.   

\subsection{Non-Hermitian complex Hamiltonians with biorthogonal eigenstates}

A complex Hamiltonian literally has a complex form with two Hermitian Hamiltonians coupled with a complex iota $i$. If $K$ is a complex Hamiltonian, then, in general, it has the configuration
\begin{align}
    \label{Complex Hamiltonian K}
    &\nonumber\\
    K \, = \, H - i \, \Gamma, \qquad \text{where $H$ and $\Gamma$ are Hermitian.}\\
    &\nonumber
\end{align}
Therefore, the complex conjugate transpose of $K$ is 
\begin{align}
    \label{Conjugate Transpose of Complex Hamiltonian K}
    &\nonumber\\
    K^{\dagger} = H^{\dagger} - (i \, \Gamma)^{\dagger} \, = \, H + i \, \Gamma.
\end{align}
In the previous section, we introduced a pair of sets that form the biorthonormal eigenbasis (one for the Hamiltonian and one for its complex transpose). Similarly, here, we introduce such a pair and assume nondegeneracy of the states. Let the set consisting of \\ 
\newpage
\hspace{-1.5em}the eigenstates of $K$ be $\{|\phi_n\rangle\}$ with eigenvalues $\{\kappa_n\}$, then,
\begin{align}
    \label{Eigenstates of K}
    &\nonumber\\
    K\left|\phi_n\right\rangle \, = \, \kappa_n\left|\phi_n\right\rangle \quad \text { and } \quad\left\langle\phi_n\right| K^{\dagger} \, = \, \kappa_n^{\, *}\left\langle\phi_n\right|.\\
    &\nonumber
\end{align}
Again, we introduce a complementary set, $\{|\chi_n\rangle\}$, comprising the eigenstates of $K^{\dagger}$ with the eigenvalues $\{\nu_n\}$. 
\begin{align}
    \label{Eigenstates of K-dagger}
    &\nonumber\\
    K^{\dagger}\left|\chi_{n}\right\rangle \, = \, \nu_n\left|\chi_{n}\right\rangle \quad \text { and } \quad\left\langle\chi_{n}\right| K \, = \, \nu_n^{\, *}\left\langle\chi_{n}\right|.\\
    &\nonumber
\end{align}
This is identical to what we did in the last section and the only difference is that we did not include the degeneracy. Since the operator $K$ is not Hermitian, the eigenstates may not be orthogonal, but might be complete with real eigenvalues. Therefore, the introduction of the complementary set is a requisite. Now that we have set the stage, finding exclusive properties of defined sets, other operators, and geometry of the space is straightforward. \par

First, as usual, let us look at the geometry of the space, that is, the inner product structure. 
\begin{align}
    \label{Inner product of phi_n}
    &\nonumber\\
    \left\langle\phi_m | \phi_n\right\rangle \, &= \, 2i \, \frac{\left\langle\phi_m\right|\left(\kappa_m^{\, *}-\kappa_n\right)\left|\phi_n\right\rangle}{2i \, (\kappa_m^{\, *}-\kappa_n)} \, = \, 2 \, \frac{\left\langle\phi_n\left|\left(\kappa_m^{\, *}+\kappa_n\right)\right| \phi_n\right\rangle}{2 \, (\kappa_m^{\, *}+\kappa_n)} \nonumber\\
    \nonumber\\
    \implies \left\langle\phi_m | \phi_n\right\rangle \, &= \, 2i \, \frac{\left\langle\phi_m\right|\left(K^{\dagger}-K\right)\left|\phi_n\right\rangle}{2i \, (\kappa_m^{\, *}-\kappa_n)} \, = \, 2 \, \frac{\left\langle\phi_n\left|\left(K^{\dagger}+K\right)\right| \phi_n\right\rangle}{2 \, (\kappa_m^{\, *}+\kappa_n)} \nonumber\\
    \nonumber\\
    \implies &\left\langle\phi_m | \phi_n\right\rangle \, = \, 2i \, \frac{\left\langle\phi_m\right|\Gamma\left|\phi_n\right\rangle}{\kappa_m^{\, *}-\kappa_n} \, = \, 2 \, \frac{\left\langle\phi_n\left|H\right| \phi_n\right\rangle}{\kappa_m^{\, *}+\kappa_n}. \\
    &\nonumber
\end{align}
Similarly, we can find the inner product for $\{ |\chi_n \rangle \}$.
\begin{align}
    \label{Inner product of chi_n}
    &\nonumber\\
    \left\langle\chi_m | \chi_n\right\rangle \, = \, 2i \, \frac{\left\langle\chi_m\right|\Gamma\left|\chi_n\right\rangle}{\nu_m-\nu_n^{\, *}} \, = \, 2 \, \frac{\left\langle\chi_n\left|H\right| \chi_n\right\rangle}{\nu_m+\nu_n^{\, *}}.
\end{align}
The set $\{ |\phi_n\rangle \}$ spans the space, but an important criterion for a basis is that the composing vectors must be linearly independent. Here, we can show this linearity using the corresponding set of $|\chi_n\rangle$'s (one more reason to choose a biorthogonal set up). \par

\hspace{-1.5em}\textbf{\textit{Theorem 1}:} The set $\{ |\phi_n\rangle \}$ is linearly independent. \par    

\hspace{-1.5em}\textbf{\textit{Proof}:} We proof this using the contrary argument, i.e., let the set $\{ |\phi_n\rangle\}$ be linearly dependent. Then
\begin{align}
    &\nonumber\\
    \sum_n c_n\left|\phi_n\right\rangle \, = \, 0, \nonumber\\
    &\nonumber
\end{align}
for all $c_n \neq 0$ or $\sum_n |c_n|^2 \neq 0$. Multiplying $\langle\chi_m|$ from the left of this equation, we get.
\begin{align}
    &\nonumber\\
    \sum_n c_n\left\langle\chi_m|\phi_n\right\rangle \, = \, \sum_n c_n \, \delta_{mn}\left\langle\chi_m|\phi_m\right\rangle \, = \, c_m \left\langle\chi_m|\phi_m\right\rangle \, = \, 0. \nonumber\\
    &\nonumber
\end{align}
The mixed inner product property involving the Kronecker Delta used here is kind of the analogue of orthogonality in standard quantum mechanics. See Appendix \ref{Mixed inner product of basis vectors in Biorthogonal Quantum Mechanics} for verification. Since the inner porducts are not zero, it must be $c_m = 0 \ \forall \ m$ or $\sum_m |c_m|^2 = 0$ in contradiction to our initial assumption. \qedsymbol \par   

The Hilbert space that we have defined here is based on the operator $K$ and its eigenvectors, which are linearly independent according to Theorem 1 above. Therefore, the set $\{ |phi_n\rangle \}$ forms a basis for this space. A general state $|\psi\rangle$ in this space, hence, can be expressed as a linear combination of the elements of $\{ |\phi_n\rangle \}$.
\begin{align}
    \label{A general state as lin combination of phi_n's}
    &\nonumber\\
    |\psi\rangle \, = \, \sum_m c_m\left|\phi_m\right\rangle. \\
    &\nonumber
\end{align}
Now multiplying Eq.(\ref{A general state as lin combination of phi_n's}) by $\langle\chi_n|$ from the left gives
\begin{align}
    \label{Multiply gen state with chi_n from left}
    &\nonumber\\
    \left\langle\chi_n|\psi\right\rangle \, = \, \sum_m c_m\left\langle\chi_n|\phi_m\right\rangle \, &=  \, \sum_m c_m \, \delta_{nm}\left\langle\chi_n|\phi_n\right\rangle \, = \, c_n \left\langle\chi_n|\phi_n\right\rangle, \nonumber\\
    \nonumber\\
    \implies c_n \, &= \, \frac{\left\langle \chi_n | \psi\right\rangle}{\left\langle \chi_n | \phi_n\right\rangle}.\\
    &\nonumber
\end{align}
Fitting this result in Eq.(\ref{A general state as lin combination of phi_n's}) and exchanging $m$ with $n$ leads to the completeness relation.
\begin{align}
    \label{BQM completeness relation}
    &\nonumber\\
    |\psi\rangle = \, \sum_n \left(\frac{\left\langle \chi_n|\psi\right\rangle}{\left\langle \chi_n| \phi_n\right\rangle}\right) \left|\phi_n\right\rangle \qquad \text{or,} \qquad   
    |\psi\rangle = \, \left(\sum_n \, \frac{\left|\phi_n \right\rangle \left\langle \chi_n \right|}{\left\langle \chi_n |\phi_n\right\rangle}\right)|\psi\rangle. \\
    &\nonumber
\end{align}
Thus, the completeness relation is  
\begin{align}
    \label{Completeness of the set phi_n and chi_n}
    &\nonumber\\
    \sum_n \, \frac{\left|\phi_n \right\rangle \left\langle \chi_n \right|}{\left\langle \chi_n |\phi_n\right\rangle} \, = \, \bm{1}.\\
    &\nonumber
\end{align}
A projection operator $[E^{2} = E]$ analogous to standard quantum mechanics can be defined as 
\begin{align}
    \label{Projection Operator in BQM}
    &\nonumber\\
    \Pi_n \, = \, \frac{\left|\phi_n\right\rangle\left\langle\chi_n\right|}{\left\langle\chi_n \mid \phi_n\right\rangle}. \\
    &\nonumber
\end{align}
Eq.(\ref{Projection Operator in BQM}) satisfies the relation $\Pi_n \, \Pi_m = \delta_{nm} \, \Pi_n$ just like any pair of indexed projection operators. \par

\subsection{Probabilistic interpretation of Biorthogonal quantum mechanics}

The formalism we have developed now needs a correct interpretation so as to be consistent with the physics of quantum mechanics. Measurement in quantum mechanics is associated with the norm of states and hence we must decide on conventions related to the norm of the biorthogonal eigenbases. Here, we must remember that the norm of a state can be greater than unity because we usually take the norm convention to be:
\begin{align}
    \label{Norm convention in BQM}
    &\nonumber\\
    \left\langle\chi_n | \phi_n\right\rangle \, = \, 1. \\
    &\nonumber
\end{align}
It must be clear that under this assumption normalisation of the eigenvectors (the sets $|\phi_n\rangle$ and $|\chi_n\rangle$) is not guaranteed. But the reason for taking this assumption will apparently be clear as it will make our lives much easier. \par

Orthogonality inherently restricts the transition of a state to another, but here in a biorthogonal system, we cannot expect this if we assume the standard quantum mechanics transition probability form $\langle \beta | \alpha\rangle\langle\alpha | \beta\rangle /\langle\alpha | \alpha\rangle\langle\beta | \beta\rangle$, between two states $|\alpha\rangle$ and $|\beta\rangle$. A transition between two eigenstates $|\phi_n\rangle$ and $|\phi_m\rangle$ is not possible even though they are not orthogonal to each other, i.e. $\langle\phi_m|\phi_n\rangle \neq 0$. In the state space, these two eigenstates are maximally separated, and hence there is no transition. But taking the standard form of transition probability leads to paradoxical reuslts here. Averting this requires a redefinition of the standard dual space in the following way: A general state $|\psi\rangle$ expanded in the biorthogonal basis is linked to an \textit{associated stated}. This \textit{associated state} is obtained using the correspondence 
\begin{align}
    \label{Redefined Duality}
    &\nonumber\\
    |\psi\rangle \, = \, \sum_n c_n &\left|\phi_n\right\rangle \, \longleftrightarrow \, \langle\widetilde{\psi}| \, = \, \sum_n c_n^*\left\langle x_n\right|, \\
    \nonumber\\
    \text{this means,} \qquad &|\widetilde{\psi}\rangle \, = \, \sum_n c_n\left|\chi_n\right\rangle.\\
    &\nonumber
\end{align}
Hence, the dual of a state $|\psi\rangle$ in the Hilbert space is given by $\langle\widetilde{\psi}|$ and then $|\widetilde{\psi}\rangle$ is the Hermitian conjugate of $\langle\widetilde{\psi}|$. The dual space, being a set of linear functionals, is then used to define the inner product: for two general states $|\alpha\rangle = \sum_n a_n |\phi_n\rangle$ and $|\beta\rangle = \sum_n b_n |\phi_n\rangle$ then define
\begin{align}
    \label{Inner Product in BQM}
    &\nonumber\\
    \langle\beta, \alpha\rangle \, \equiv \, \langle\widetilde{\beta} | \alpha\rangle \, = \, \sum_{n, m} \, b_{n}^{ \, *} \, a_m\left\langle\chi_n | \phi_m\right\rangle \, = \, \sum_n \, b_{n}^{ \, *} \, a_n.\\
    &\nonumber
\end{align}
We have used the assumpiton in Eq.(\ref{Norm convention in BQM}) along with the inner product relation in Appendix \ref{Mixed inner product of basis vectors in Biorthogonal Quantum Mechanics} Eq.(\ref{Mixed inner product in BQM}). Here, we will use another assumption that the state space is a Bloch sphere of unit radius. We have, for a general state $|\alpha\rangle$:  
\begin{align}
    \label{State Space of unit Bloch sphere}
    &\nonumber\\
    \langle\widetilde{\alpha} | \alpha\rangle \, = \, \sum_n \, a_n^{\, *} a_n \, = \, 1.\\
    &\nonumber
\end{align}
This assumption coupled with Eq.(\ref{Norm convention in BQM}) gives the transition probability of a general state $|\alpha\rangle$ to a specific eigenstate $|\phi_n\rangle$. 
\begin{align}
    \label{Transition Probability in BQM}
    &\nonumber\\
    p_n \, = \, a_n^{\, *} a_n \, = \, \frac{\left\langle \chi_n | \alpha\right\rangle\left\langle\tilde{\alpha} | \phi_n\right\rangle}{\langle\tilde{\alpha} | \alpha\rangle\left\langle \chi_n | \phi_n\right\rangle}. \\
    &\nonumber
\end{align}
A simple calculation verifies the above result. 
\begin{align}
    &\nonumber\\
    \frac{\left\langle \chi_n | \alpha\right\rangle\left\langle\tilde{\alpha} | \phi_n\right\rangle}{\langle\tilde{\alpha} | \alpha\rangle\left\langle \chi_n | \phi_n\right\rangle} \, &= \, \frac{\left(\sum_m a_m\left\langle \chi_n | \phi_m\right\rangle\right)\left(\sum_m a_m^{\, *}\left\langle \chi_m | \phi_n\right\rangle\right)}{\langle\widetilde{\alpha} | \alpha\rangle\left\langle \chi_n | \phi_n\right\rangle} \nonumber\\
    \nonumber\\
    &= \, \frac{\left(\sum_m a_m \, \delta_{nm}\left\langle \chi_n | \phi_n\right\rangle\right)\left(\sum_m a_m^{\, *} \, \delta_{mn}\left\langle \chi_m | \phi_m\right\rangle\right)}{\langle\widetilde{\alpha} | \alpha\rangle\left\langle \chi_n | \phi_n\right\rangle}\nonumber\\
    \nonumber\\
    &= \, \frac{a_n\left\langle \chi_n | \phi_n\right\rangle a_n^{\, *}\left\langle \chi_n | \phi_n\right\rangle}{\langle\widetilde{\alpha} | \alpha\rangle\left\langle \chi_n | \phi_n\right\rangle} \, = \, a_n^{\, *} a_n. \quad \text{[Using Eq.(\ref{Norm convention in BQM}) \& Eq.(\ref{State Space of unit Bloch sphere})]} \nonumber
\end{align}
Eq.(\ref{Transition Probability in BQM}) has a good analogy to measurement of a general state in standard quantum mechanics. Here, $p_n$ is the probability of obtaining an eigenvalue of $\kappa_n$ when a $\hat{K}$ measurement is performed on a general state $|\alpha\rangle$. \par 

\subsection{Characterising observables; Why use a biorthogonal basis?}

We have introduced a biorthogonal basis to counteract the fact that orthogonality of basis eigenstates is not available in a scenario where the Hamiltonian is complex. But this begs the question: How do we represent Hermiticity in the context of a complex system? This will be analogous to standard quantum mechanics and completes the picture when it comes to describing any operator $\mathcal{O}$ in such an arrangement. \par

Using a biorthogonal basis $\{|\phi_n\rangle,|\chi_n\rangle\}$, as in Eq.(\ref{Eigenstates of K}) and Eq.(\ref{Eigenstates of K-dagger}), to represent a generic operator $\mathcal{O}$ would mean, 
\begin{align}
    \label{Representing a general operator O in biorthogonal basis}
    &\nonumber\\
    \mathcal{O} \, = \, \sum_{n,m} \,  o_{nm} \, |\phi_n\rangle \langle\chi_m|, \\
    &\nonumber
\end{align}
where $\{ o_{nm} \}$ is a matrix. If the biorthogonal basis $\{|\phi_n\rangle,|\chi_n\rangle\}$ are eigenstates of $\mathcal{O}$ then ${o_{nm}}$ would be a diagonal matrix with the eigenvalues along the diagonal, i.e. $o_{nm} = \delta_{nm} \, o_n$. \par

Now, if we had used only the set $\{ |\phi_n\rangle \}$ to represent vectors and operators in this Hilbert space, we would have encountered a serious problem. Since this set is not orthogonal, let alone orthonormalised, the representation of an arbitrary operator $\mathcal{O}$ would look like.
\begin{align}
    \label{Representing a general operator O in complete not orthogonal basis}
    &\nonumber\\
    \mathcal{O} \, = \, \sum_{n,m} \, \theta_{nm} \, |\phi_n\rangle\langle\phi_m|.\\
    &\nonumber
\end{align}
Here, $\{ |\phi_n\rangle \}$ is complete but due to lack of orthogonality the set $\{\theta_{nm}\}$ forms an array rather than being represented as a matrix. This complication is resolved by the use of a biorthogonal basis, and thus, justifies our effort in developing all the theory above. \par

If two operators $A$ and $B$ are represented using a biorthogonal basis $\{|\phi_n\rangle,|\chi_n\rangle\}$ as in Eq. (\ref{Representing a general operator O in biorthogonal basis}) with co-effcients $\{ a_{nm} \}$ and $\{ b_{nm} \}$, respectively. Then their composition or product is given by
\begin{align}
    \label{Composition or product of two operators in BQM}
    &\nonumber\\
    (P)_{nm} \, = \, p_{nm} \, = \, \sum_r \, a_{nr} \, b_{rm},\\
    &\nonumber
\end{align}
just like any matrix multiplication. Another key aspect of using biorthogonal bases lies in the fact that transforming one basis into another cannot always be achieved through unitary transformations. This convenience is readily available in the Hermitian scenario. Therefore, when working with complex Hamiltonian systems, the theory is dependent on specifying or choosing the basis, an important distinction between the theories. \par

Hermiticity can now be easily redefined in the biorthogonal setting. For any operator described by Eq.(\ref{Representing a general operator O in biorthogonal basis}), \textit{biorthogonal Hermiticity} is defined as
\begin{align}
    \label{Biorthogonal Hermiticity}
    &\nonumber\\
    o_{nm}^{\, *} \, = \, o_{mn}.\\
    &\nonumber
\end{align}
This is a natural extension of Hermitian quantum mechanics. When we talk about measurements, the expectation values of operators with respect to pure states is a genuine sequel. The expectation value of an operator $\mathcal{O}$ with respect to a pure state $|\alpha\rangle$ is defined as
\begin{align}
    \label{Expectation value of an operator wrt a pure state in BQM}
    &\nonumber\\
    \langle\mathcal{O}\rangle \, = \, \frac{\langle\tilde{\alpha}| \, \mathcal{O} \, | \alpha\rangle}{\langle\tilde{\alpha} | \alpha\rangle}.\\
    &\nonumber
\end{align}
Analogously to standard quantum mechanics, here we state that if the operator $\mathcal{O}$ is \textit{biorthogonally Hermitian}, as in Eq.(\ref{Biorthogonal Hermiticity}), then Eq.(\ref{Expectation value of an operator wrt a pure state in BQM}) is real for an arbitrary pure state $|\alpha\rangle$. We can write Eq.(\ref{Expectation value of an operator wrt a pure state in BQM}) using the matrix representation in Eq.(\ref{Representing a general operator O in biorthogonal basis}) by expanding $|\alpha\rangle$ in the biorthogonal basis, $|\alpha\rangle \, = \, \sum_n \, a_n \, |\phi_n\rangle$ and substituting this in Eq.(\ref{Expectation value of an operator wrt a pure state in BQM}).
\begin{align}
    \label{Expectation value in matrix representation }
    &\nonumber\\
    \langle\mathcal{O}\rangle_{|\alpha\rangle} \, &= \, \frac{(\sum_k \, a_k^{\, *} \, \langle \chi_k|) \, (\sum_{n, m} \, o_{n m} \, |\phi_n \rangle\langle \chi_m|) \, (\sum_{l} \, a_l \, |\phi_l\rangle)}{(\sum_k \, a_k^{\, *} \, \langle \chi_k|) \, (\sum_{l} \, a_l \, |\phi_l\rangle)} \nonumber\\
    \nonumber\\
    &= \, \frac{\sum_{k,l,m,n} \, a_k^{\, *} \, a_l \, o_{nm} \, \langle \chi_k|\phi_n \rangle \langle \chi_m|\phi_l\rangle}{\sum_{k,l} \, a_k^{\, *} \, a_l \, \chi_k|\phi_l\rangle} \nonumber\\
    \nonumber\\
    &= \, \frac{\sum_{k,l,m,n} \, a_k^{\, *} \, a_l \, o_{nm} \, \delta_{kn} \, \delta_{ml} \, \langle \chi_k|\phi_k \rangle \langle \chi_m|\phi_m\rangle}{\sum_{k,l} \, a_k^{\, *} \, a_l \, \delta_{kl} \, \langle\chi_k|\phi_k\rangle} \quad \text{[Using Eq.(\ref{Mixed inner product in BQM})]}\nonumber\\
    \nonumber\\
    &= \, \frac{\sum_{m,n} \, a_n^{\, *} \, a_m \, o_{nm}}{\sum_{n} \, a_n^{\, *} \, a_n}. \quad \text{[Using Eq.(\ref{Norm convention in BQM})]}\\
    &\nonumber
\end{align}
Now, if we also assume that $\{|\phi_n\rangle\}$ are eigenstates of $\mathcal{O}$ then the expectation value takes the form consistent with the probabilistic interpretation.
\begin{align}
    \label{Expectation value of an operator wrt to probabilities in BQM}
    &\nonumber\\
    \langle\mathcal{O}\rangle_{|\alpha\rangle} \, = \, \sum_{n} \, p_n \, o_n, \quad \text{where} \ o_n \ \text{are eigenvalues, and} \ p_n \, = \, \frac{c_n^{\, *} \, c_n}{\sum_k \, c_k^{\, *} \, c_k}.\\
    &\nonumber
\end{align} \par
We have mentioned that the use of a biorthogonal basis has a disadvantage: an operator matrix representation $o_{nm}$ can be determined to be Hermitian if the basis is specified. Orthogonal bases do not have this problem; any arbitrary orthogonal basis does the job. To avoid this ambiguity, we can express the biorthogonal basis vectors using an orthonormal basis set. This leads to more rigid conditions. Let $\{ \epsilon \}$ be an arbitrary orthonormal basis for the Hilbert space and the biorthogonal eigenvectors can be expanded using this basis as
\begin{align}
    \label{Biorthonormal Basis}
    &\nonumber\\
    |\phi_n\rangle \, = \, \sum_k \, \upsilon_n^{\, k} \, |e_k\rangle, \qquad \qquad |\chi_n\rangle \, = \, \sum_k \, \nu_n^{\, k} \, |e_k\rangle. \\
    &\nonumber
\end{align}
Replacing these in Eq.(\ref{Representing a general operator O in biorthogonal basis}) we obtain 
\begin{align}
    \label{Representing O in an arbitrary orthonormal basis}
    &\nonumber\\
    \mathcal{O} \, &= \, \sum_{k,l} \, o_{k l} \, \left(\sum_n \, \upsilon_k^{\, n} \, |e_n\rangle\right)\left(\sum_m \, {\nu_l^{\, m^{\, *}}} \, |e_m\rangle\right) \nonumber\\
    &= \, \sum_{n, m}\left[\left(\sum_{k, l} \, o_{k l} \, \upsilon_k^{\, n} \, {\nu_l^{\, m^{\, *}}}\right)\left|e_n \rangle\langle e_{m l}\right|\right].\\
    &\nonumber
\end{align}
The reality of the energy eigenvalue spectrum of operator $\mathcal{O}$ is ensured by its biorthogonal Hermiticity. But Eq.(\ref{Representing O in an arbitrary orthonormal basis}) clearly shows that biorthognal Hermiticity is guaranteed by the condition 
\begin{align}
    \label{BQM Hermiticity is ensured}
    &\nonumber\\
    \sum_{k, l} \, o_{k l} \, \upsilon_k^{\, n} \, {{\nu_l}^{\, m^{\, *}}} \, = \, \sum_{k, l} \, o_{k l}^{\, *} \, \upsilon_k^{\, m^{\, *}} {{\nu_l}^{\, n}}. \\
    &\nonumber
\end{align} \par
We have talked about the basic formalism of Non-Hermitian quantum mechanics in this and the last chapter, but there is one detour that we will take before completing this one. \textit{Quasi-Hermitian quantum mechanics} is an important concept that was given shape in the early 1990s by Scholtz in his paper \cite{SCHOLTZ199274}. This paper was instrumental in the development of pseudo-Hermitian quantum mechanics and acted as a prequel to all modern developments. Therefore, it is imperative that we devote sometime to understanding the very beginnings of non-Hermitian quantum theory. \par

\section{A brief description of Quasi-Hermitian Quantum Mechanics}

Non-Hermitian operators appeared in various fields of physics, especially in the 1970s: theory of effective interactions \cite{1975,SCHUCAN1973483}, restoration of translational invariance in Hartree-Fock theory for finite systems \cite{JANSSEN1989270}. There was insufficient literature on understanding the physical consequences of non-Hermitian systems. Frederik G. Scholtz \textit{et al.} argued the need for a more natural framework for quantum mechanics that would include non-Hermitian operators \cite{SCHOLTZ199274}. The principle idea was to redefine the inner product so that the non-Hermitian operators would become Hermitian with respect to it. Their argument was to focus on the operators, which then defined the Hilbert space rather than the opposite, in which case the space is chosen \textit{a priori} and then the physical observables would have to be Hermitian with respect to the inner product. \par

The solution was to introduce the concept of a set of quasi-Hermitian operators. We will elucidate this idea in the following subsection. But briefly speaking, a set of operators that are Hermitian with respect to a redefined inner product is called a quasi-Hermitian set of operators. The new inner product is defined in terms of the initially chosen inner product. This is quite analogous to what we have discussed in pseudo-Hermiticity, where the $\eta$ operator is obtained based only on the Hamiltonian of the system rather than a set of operators. \par

A comprehensive discussion on quasi-Hermitian set of operators can be found in Marshal C. Pease's 1965 book on matrix algebra \cite{PeaseMethodsofMatrixAlgebra}. The immediate question one can ask is regarding the uniqueness of this redefined metric. Clearly, it is always possible to get multiple inner products with respect to which a set of non-Hermitian operators becomes Hermitian. This uniqueness is guaranteed when the set of operators is irreducible in the Hilbert space. This will also be touched upon in the following subsection. \par

\subsection{Quasi-Hermitian Formalism}

Let $H$ be a Hilbert space with inner product $(\cdot,\cdot)$ and let $\{ \mathcal{O}_i \}$ be a set of bounded linear operators on $H$. These operators are not necessarily Hermitian. The domain of the operators and their adjoints being the complete Hilbert space $H$, i.e. $D(\mathcal{O}_i) \, = \, D(\mathcal{O_i^{\, \dagger}}) \, = \, H$. Then the set of operators $\{ O_i \}$ is called quasi-Hermitian if $\exists$ a linear operator $T: H \rightarrow H$ such that
\begin{subequations}
\begin{align}
    \label{Set of Quasi-Hermitian observables}
    &\nonumber\\
    & \qquad \qquad \qquad \ \, D(T) \, = \, H, \\
    \nonumber\\
    & \qquad \qquad \qquad \ \, T^{\, \dagger} \, = \, T, \qquad \text{(Self-adjointness),}\\
    \nonumber\\
    (\alpha, T \, \alpha) \, > \, 0, \quad \forall \ \alpha \in & \ H \ \text{and} \ \alpha \neq 0 \qquad \text{(Positive definite inner product)},\\
    \nonumber\\
    &||T \, \alpha|| \, \leq \,  ||T|| \, ||\alpha|| \quad \forall \ \alpha \in H \qquad \text{(Bounded operator),}\\
    \nonumber\\
    &T \, \mathcal{O}_i \, = \, \mathcal{O}_i^{\, \dagger} \, T \quad \forall \ i \qquad \text{(Quasi-Hermitian condition).} \\
    &\nonumber
\end{align}
\end{subequations}
Here, the `dagger' ($\dagger$) represents the Hilbert space adjoint operation, as in standard quantum mechanics. For finite-dimensional spaces, properties (2.69a) and (2.69d) are easily satisfied. The boundedness of a linear operator is described in Appendix \ref{Boundedness of a linear operator}. Therefore, properties (2.69b), (2.69c), and (2.69e) need to be satisfied to obtain a quasi-Hermitian set. In infinite-dimensional spaces, all the properties in Eq.(\ref{Set of Quasi-Hermitian observables}) are difficult to obtain. \par 

We now define the modified inner product that we have mentioned so many times above. This inner product is based on the original inner product of the Hilbert space $H$, $(\cdot, \cdot)$. Let $\alpha$, $\beta \in H$ and then we have the modified inner porduct as
\begin{align}
    \label{Quasi-Hermitian Inner Product}
    &\nonumber\\
    (\alpha, \beta)_{\, T} \, \equiv \, (\alpha, T \beta) \quad \text{on $H$}.\\
    &\nonumber
\end{align} 
This inner product induces a norm on $H$: $||\alpha||_{\, T} \, = \, \sqrt{(\alpha, \alpha)_{\, T}} \, = \, \sqrt{(\alpha, T \alpha)}$. Together with this norm, the original Hilbert space $H$ is modified to form $H_{\, T}$, a fact that can be confirmed \cite[see Appendix A]{SCHOLTZ199274}. Now we can easily see that the operators $\{ \mathcal{O}_i \}$ are Hermitian w.r.t. to this newly defined inner product in the Hilbert space $H_{\, T}$. 
\begin{align}
    \label{Hermitian wrt quasi-Hermitian inner product}
    &\nonumber\\
    (\alpha, \mathcal{O}_i \, \beta)_{\, T} \, = \, (\alpha, T \mathcal{O}_i \, \beta) \, = \, (\alpha, \mathcal{O}_i^{\dagger} \, T \beta) \, = \, (\mathcal{O}_i \, \alpha, T \beta) \, = \, (\mathcal{O}_i \, \alpha, \beta)_{ \, T} \\
    &\nonumber
\end{align} 
The primary question that can be asked here is about the uniqueness of $T$. It can be shown that $T$ is unique on $H$ (upto a globalisation factor) if and only if $\{\mathcal{O}_i\}$ is irreducible on $H$ \cite[see Appendix A]{SCHOLTZ199274}. A set of operators $\{\mathcal{O}_i\}$ on $H$ is called irreducible if there does not exist any proper subset of $H$ that is invariant under each operator $O_i$. \par

This detour was necessary to complete the picture of the enormous development behind non-Hermitian quantum theory. $\mathcal{PT}$-symmetry was developed after the work done on quasi-Hermiticity which was followed by pseudo-Hermiticity. Although there is no order that we can assign to theoretical physics work, this discussion gives the reader a rough sketch of the groundbreaking work established by some of the best mathematical physicists. 


%%%%%%%%%%%%%%%%%%%%%%%%%%%%%%%%%%%%%%%%%%%%%%%%%%%%%%%%%%%%%%%%%%%%%%%%%%%%%%%%%


%\chapter[Time, it needs time]{Time, it needs time\raisebox{.3\baselineskip}{\normalsize\footnotemark}}{\footnotetext[1]{\textit{Still Loving You} by \text{Scorpions} is one of the greatest rock hits of all time. When Klaus Meine, their lead vocalist, starts singing the very first line (title of Chapter 3), it sends a chill down your spine. Check out this rendition with the Berlin Philharmonic: \href{https://www.youtube.com/watch?v=fPBRrl6eSw0}{https://www.youtube.com/watch?v=fPBRrl6eSw0}.}}
%\epigraph{\textbf{Scissors cuts paper, paper covers rock, rock crushes lizard, lizard poisons Spock, Spock smashes scissors, scissors decapitates lizard, lizard eats paper, paper disproves Spock, Spock vaporises rock, and as it always has, rock crushes scissors.}}{\textit{Sheldon Lee Cooper (Ph.D.), The Big Bang Theory: Season 5, Episode 17}}
%\input{chapters/chap3}

%%%%%%%%%%%%%%%%%%%%%%%%%%%%%%%%%%%%%%%%%%%%%%%%%%%%%%%%%%%%%%%%%%%%%%%%%%%%%%%%

\newpage

\fancyhf{}
\renewcommand{\headrulewidth}{0.0pt}
\fancyfoot[C]{\thepage}
%%%%%%%%%%%%%%%%%%%%%%%%%%%%%%%%%%%%%%%%%%%%%%%%%%%%%%%%%%%%%%%%%%


\chapter*{Conclusions}
\addcontentsline{toc}{chapter}{Conclusions}

\hspace{1.5em}The work presented here covers enormous literature and concepts vital for the evolution of non-Hermitian quantum mechanics. Our goal was to understand the physicality of non-Hermitian Hamiltonians in standard quantum mechanics. To build a framework consistent with the axioms of quantum mechanics, we saw that conventional quantum theory is not suffcient, and a different approach was required. $\mathcal{PT}$-symmetry allowed us to descibe the energy spectrum of a non-Hermitian Hamiltonian, dividing it into regimes of \textit{spontaneously broken} and \textit{unbroken} $\mathcal{PT}$\textit{-symmetry}. But this was again not enough to understand non-Hermitian Hamiltonians that were not $PT$ invariant,  and yet had the whole or a part of their spectra to be real. Pseudo-Hermitian quantum mechanics was introduced to counter these shortcomings of $\mathcal{PT}$-symmetry. Results and predictions of pseudo-Hermitian formalism subsumed the work of $\mathcal{PT}$-symmetry and allowed for a more broader formalism which descibed all time-independent non-Hermitian systems. \par

\subsection*{Future Plans}

The work we have presented to the reader can be followed by an investigation oftime-dependent Hamiltonians and linear automorphisms $H(t)$, and $\eta(t)$ respectively. Applications of time-dependent non-Hermitian Hamiltonian systems can prove to be indispensable in terms of solutions of Schr\"{o}dinger equations for systems that were not conceivable before. \par

We have seen transitions in the energy spectra for different non-Hermitian Hamiltonians with $\mathcal{PT}$-symmetry. Therefore, moving from an unbroken $\mathcal{PT}$-symmetric regime to the broken one can have far-reaching consequences in quantum many-body physics. Especially in the context of codensed matter physics to realise $\mathcal{PT}$-symmetry in experiments. \par

Pseudo-Hermitian and $\mathcal{PT}$-symmteric quantum mechanics can be investigated within the path-integral formulation. Relativistic Pseudo-Hermitian quantum mechanics would also be an intriguing sequel to the work on. But as we have seen throughout this thesis, in the static scenario itself, the theory of non-Hermitian quantum mechanics proves to be extremely rich and interesting.  


%%%%%%%%%%%%%%%%%%%%%%%%%%%%%%%%%%%%%%%%%%%%%%%%%%%%%%%%%%%%%%%%%%



\newpage



\bibliographystyle{these}
\bibliography{bib.bib}


\newpage

\appendix
\chapter{The $\mathcal{C}$ operator}
\section{Calculation of the $\mathcal{C}$ operator}
\label{Calculate C operator}

\begin{align}
    \label{All the elements of C}
    &\nonumber\\
    \mathcal{C}(1,2) \, &= \, \phi_1(1) \, \phi_1(2) \, + \, \phi_2(1) \, \phi_2(2)\nonumber\\
    &= \, \frac{1}{2 \cos \alpha} \, \left(e^{i \alpha / 2}\right) \, \left(e^{-i \alpha / 2}\right) \, + \, \frac{i^2}{2 \cos \alpha} \, \left(e^{-i \alpha / 2}\right) \, \left(-e^{i \alpha / 2}\right) \nonumber\\
    &= \, \frac{1}{\cos \alpha}. \\
    &\nonumber\\
    \mathcal{C}(2,1) \, &= \, \phi_1(2) \, \phi_1(1) \, + \, \phi_2(2) \, \phi_2(1) \nonumber\\
    &= \, \frac{1}{2 \cos \alpha}\left(e^{-i \alpha / 2}\right) \, \left(e^{i \alpha / 2}\right) \, + \, \frac{i^2}{2 \cos \alpha} \, \left(-e^{i \alpha / 2}\right) \, \left(e^{-i \alpha / 2}\right) \nonumber\\
    &= \, \frac{1}{\cos \alpha}. \\
    &\nonumber\\
    \mathcal{C}(2,2) \, &= \, \phi_1(2) \, \phi_1(2) \, + \, \phi_2(2) \, \phi_2(2) \nonumber\\
    &= \, \frac{1}{2 \cos \alpha} \, \left(e^{-i \alpha / 2}\right) \, \left(e^{-i \alpha / 2}\right) \, + \, \frac{i^2}{2 \cos \alpha} \, \left(-e^{i \alpha / 2}\right) \, \left(-e^{i \alpha / 2}\right) \nonumber\\
    &= \, \frac{1}{2 \cos \alpha} \, \left(e^{-i \alpha / 2} \, - e^{i \alpha}\right) \, = \, \frac{1}{\cos \alpha} \, (-i \sin \alpha).\\
    &\nonumber
\end{align}

\section{$\mathcal{C}$ operator is self-invertible}
\label{C self-invertibility}

The square of the $\mathcal{C}$ operator is unity. We can check this.
\begin{align}
    \label{C squared equals 1}
    &\nonumber\\
    \mathcal{C}^{\, 2} &= \, \frac{1}{\cos ^2 \alpha}\left(\begin{array}{cc}
                                                        i \sin \alpha & 1 \\
                                                        1 & -i \sin \alpha
                                                      \end{array}\right)
                                      \left(\begin{array}{rr}
                                            i \sin \alpha & 1 \\
                                            1 & -i \sin \alpha
                                            \end{array}\right) \nonumber\\
        &= \, \frac{1}{\cos ^2 \alpha}
        \left(\begin{array}{ll}
                1-\sin ^2 \alpha & i \sin \alpha-i \sin \alpha \nonumber\\
                i \sin \alpha-i \sin \alpha & 1-\sin^2 \alpha
              \end{array}\right) \\
        &= \, \left(\begin{array}{ll}
                        1 & 0 \\
                        0 & 1
                \end{array}\right).\\
    &\nonumber
\end{align}
\vspace{-17mm}
\section{The $\mathcal{CPT}$ inner product}
\label{CPT inner product of minus-plus eigenstates}

The application of the total $\mathcal{CPT}$ operator to eigenstates in Eq.(\ref{Eigenstates of H and PT}) gives
\begin{align}
    \label{CPT applied to plus eigenstate}
    &\nonumber\\
    \mathcal{C P T} \, \left|\varepsilon_{+}\right\rangle \, &= \,
    \frac{1}{\sqrt{2 \cos \alpha}} \, 
    \mathcal{CP}\left(\begin{array}{l}
                        e^{-i \alpha / 2} \\
                        e^{i \alpha / 2}
                      \end{array}\right) \nonumber\\
    &= \, \frac{1}{\sqrt{2 \cos \alpha}} \, \mathcal{C} 
    \left(\begin{array}{ll}
            0 & 1 \\
            1 & 0
          \end{array}\right)
    \left(\begin{array}{c}
            e^{-i \alpha / 2} \\
            e^{i \alpha / 2}
          \end{array}\right) \nonumber\\
    &= \, \frac{1}{\cos \alpha \sqrt{2 \cos \alpha}} 
    \left(\begin{array}{cc}
            i \sin \alpha & 1 \\
            1 & -i \sin \alpha
          \end{array}\right)
    \left(\begin{array}{c}
            e^{i \alpha / 2} \\
            e^{-i \alpha / 2}
          \end{array}\right) \nonumber\\
    &= \, \frac{1}{\cos \alpha \sqrt{2 \cos \alpha}}
    \left(\begin{array}{c}
            i \sin \alpha \, e^{i \alpha / 2}+e^{-i \alpha / 2} \\
            e^{i \alpha / 2}-i \sin \alpha \, e^{-i \alpha / 2}
          \end{array}\right) \nonumber\\
    &= \, \frac{1}{\cos \alpha \sqrt{2 \cos \alpha}} 
    \left(\begin{array}{l}
            e^{i \alpha / 2} \, \cos \alpha \\
            e^{-i \alpha / 2} \, \cos \alpha
          \end{array}\right) \nonumber\\
    &= \, \frac{1}{\sqrt{2 \cos \alpha}}
    \left(\begin{array}{l}
            e^{i \alpha / 2} \\
            e^{-i \alpha / 2}
          \end{array}\right) \\
    &\nonumber\\
    \mathcal{C P T} \, \left|\varepsilon_{-}\right\rangle \, &= \,
    \frac{1}{\sqrt{2 \cos \alpha}} \, \mathcal{C P} \, (-i)
    \left(\begin{array}{c}
            e^{i \alpha / 2} \\
            -e^{-i \alpha / 2}
          \end{array}\right) \nonumber\\
    &= \, \frac{-i}{\sqrt{2 \cos \alpha}} \, \mathcal{C}
    \left(\begin{array}{ll}
            0 & 1 \\
            1 & 0
          \end{array}\right)
    \left(\begin{array}{c}
            e^{i \alpha / 2} \\
            -e^{-i \alpha / 2}
          \end{array}\right)\nonumber\\
    &= \,\frac{-i}{\sqrt{2 \cos \alpha}} \, \mathcal{C}
    \left(\begin{array}{c}
            -e^{-i \alpha / 2} \\
            e^{i \alpha / 2}
          \end{array}\right) \nonumber\\
    &= \, \frac{-i}{\cos \alpha \, \sqrt{2 \cos \alpha}}
    \left(\begin{array}{cc}
            i \sin \alpha & 1 \\
            1 & -i \sin \alpha
          \end{array}\right)
    \left(\begin{array}{c}
            -e^{-i \alpha / 2} \\
            e^{i \alpha / 2}
          \end{array}\right)\nonumber\\
    &= \, \frac{-i}{\cos \alpha \, \sqrt{2 \cos \alpha}}
    \left(\begin{array}{l}
            -i \sin \alpha \, e^{-i \alpha / 2}+e^{i \alpha / 2} \\
            -e^{-i \alpha / 2}-i \sin \alpha \, e^{i \alpha / 2}
          \end{array}\right)\nonumber\\
    &= \, \frac{-i}{\cos \alpha \, \sqrt{2 \cos \alpha}}
    \left(\begin{array}{l}
            e^{-i \alpha / 2} \, \cos \alpha \\
            -e^{i \alpha / 2} \, \cos \alpha
          \end{array}\right) \nonumber\\
    &= \, \frac{-i}{\sqrt{2 \cos \alpha}}
    \left(\begin{array}{c}
            e^{-i \alpha / 2} \\
            -e^{i \alpha / 2}
          \end{array}\right). \\
    &\nonumber
\end{align}
\hspace{1.5em} Now we calculate the inner product for various permutations of the plus and minus eigenstates.
\begin{align}
    \label{CPT inner-product of diff permutations of plus-minus eigenstates}
    &\nonumber\\
    \left\langle\varepsilon_+ \mid \varepsilon_+\right\rangle \, &= \,
    \frac{1}{\sqrt{2 \cos \alpha}}
    \begin{pmatrix}
          e^{i \alpha / 2} & e^{-i \alpha / 2}
    \end{pmatrix}
    \frac{1}{\sqrt{2 \cos \alpha}}
    \left(\begin{array}{l}
            e^{i \alpha / 2} \\
            e^{-i \alpha / 2}
          \end{array}\right) \nonumber\\
    &= \, \frac{1}{2 \cos \alpha}\left[e^{i \alpha}+e^{-i \alpha}\right] \,= \, +1. \\
    &\nonumber\\
    \left\langle\varepsilon_{-} \mid \varepsilon_{-}\right\rangle \, &= \,
    \frac{-i}{\sqrt{2 \cos \alpha}}
    \begin{pmatrix}
            e^{-i \alpha / 2} & -e^{i \alpha / 2}             
    \end{pmatrix}
    \frac{i}{\sqrt{2 \cos \alpha}}
    \left(\begin{array}{c}
            e^{-i \alpha / 2} \\
            -e^{i \alpha / 2}
          \end{array}\right)\nonumber\\
    &= \, \frac{1}{2 \cos \alpha}\left[e^{-i \alpha}+e^{i \alpha}\right] \, = \, +1. \qquad \text{\textbf{[Positive!]}}\\
    &\nonumber\\
    \left\langle\varepsilon_{-} \mid \varepsilon_{+}\right\rangle \, &= \, \frac{-i}{\sqrt{2 \cos \alpha}}
    \begin{pmatrix}
            e^{-i \alpha / 2} & -e^{i \alpha / 2}
    \end{pmatrix}
    \frac{1}{\sqrt{2 \cos \alpha}}
    \left(\begin{array}{l}
            e^{i \alpha / 2} \\
            e^{-i \alpha / 2}
          \end{array}\right)\nonumber\\
    &= \, \frac{-i}{2 \cos \alpha}\left[e^0-e^0\right] \, = \, 0.\\
    &\nonumber\\
    \left\langle\varepsilon_{+} \mid \varepsilon_{-}\right\rangle \, &= \, \frac{1}{\sqrt{2 \cos \alpha}}
    \begin{pmatrix}
            e^{i \alpha / 2} & e^{-i \alpha / 2}
    \end{pmatrix}
    \frac{i}{\sqrt{2 \cos \alpha}}
    \left(\begin{array}{c}
            e^{-i \alpha / 2} \\
            -e^{i \alpha / 2}
          \end{array}\right)\nonumber\\
    &= \, \frac{i}{2 \cos \alpha}\left[e^0-e^0\right] \, = \, 0.\\
    &\nonumber
\end{align}

\section{Completeness condition of the eigenstates}
\label{completeness condition for the plus-minus eigenstates}
\begin{align}
    \label{completeness of pm eigenstates}
    &\nonumber\\
    & \qquad \qquad \qquad \quad \left|\varepsilon_{+}\right\rangle\left\langle\varepsilon_{+}|+| \varepsilon_{-}\right\rangle\left\langle\varepsilon_{-}\right| \, = \, \nonumber\\
    &= \, \frac{1}{\sqrt{2 \cos \alpha}}
    \left(\begin{array}{c}
            e^{i \alpha / 2} \\
            e^{-i \alpha / 2}
          \end{array}\right)
    \frac{1}{\sqrt{2 \cos \alpha}}
    \begin{pmatrix}
            e^{i \alpha / 2} &  e^{-i \alpha / 2}      
    \end{pmatrix}\nonumber\\
    & \qquad \qquad \qquad 
    +\frac{i}{\sqrt{2 \cos \alpha}}
    \left(\begin{array}{cc}
            e^{-i \alpha / 2} \\
            -e^{i \alpha / 2}
          \end{array}\right)
    \frac{i}{\sqrt{2 \cos \alpha}}
    \begin{pmatrix}
            -e^{-i \alpha / 2} & e^{i \alpha / 2}
    \end{pmatrix}\nonumber\\
    &= \, \frac{1}{2 \cos \alpha}
    \left(\begin{array}{cc}
            e^{i \alpha} & 1 \\
            1 & e^{-i \alpha}
          \end{array}\right)-
    \frac{1}{2 \cos \alpha}
    \left(\begin{array}{cc}
            -e^{-i \alpha} & 1 \\
            1 & -e^{i \alpha}
          \end{array}\right)\nonumber\\
    &= \, \frac{1}{2 \cos \alpha}
    \left(\begin{array}{cc}
            e^{i \alpha}+e^{-i \alpha} & 0 \\
            0 & e^{-i \alpha}+e^{i \alpha}
          \end{array}\right) \, = \, 
    \left(\begin{array}{cc}
            1 & 0 \\
            0 & 1
          \end{array}\right) \, = \, \bm{1}. \\
    &\nonumber
\end{align}



\chapter{Seminorm}
\section{What is a Seminorm?}
\label{Seminorm illustration}

Norms are thoroughly studied in functional analysis and in topology or geometry. The geometrical properties of a vector space are defined by its norm. The same vector space can behave much differently under non-identical norms. A vector space with its norm defined is termed a normed space or an inner-product space.  \par

\hspace{-1.5em}Let $\bm{V}$ be a vector space with over the field of real numbers $\mathbb{R}$ or complex numbers $\mathbb{C}$ with a real-valued function $\bm{f}:\bm{V} \rightarrow \mathbb{R}$ (\textbf{A functional}). Then $\bm{f}$ is a \textit{seminorm} if the following properties hold. 

\begin{itemize}
    \item[(i)] $\bm{f}(x+y) \leq \bm{f}(x) + \bm{f}(y), \ \forall \ x,y \in \bm{V}$ or \textbf{triangle inequaltity holds true}. 
    \item[(ii)] $\bm{f}(sx) = |s|\bm{f}(x), \ \forall \ x \in \bm{V}$ and all scalars $s$ or the space has \textbf{absolute homegeneity}. The consequences of this property are as follows.
    \begin{itemize}
        \item[$\bullet$] $\bm{f}(0)= 0$. (\textit{Trivial!})
        \item[$\bullet$] $\bm{f}(x) \geq 0, \ \forall \ x \in \bm{V}$. Since,
              \begin{align}
              \label{}
              &\bm{f}(0)\, = \, 0 \ \text { or, } \ \bm{f}(x-x)=0 \nonumber\\
              \implies & \bm{f}(x)+\bm{f}(-x) \, \geq \, \bm{f}(x-x) \, = \, 0 \nonumber\\
              \implies & \bm{f}(x)+|-1| \bm{f}(x) \, \geq \, 0 \nonumber\\
              \implies & 2\bm{f}(x) \, \geq \, 0 \ \text{ or, } \ \bm{f}(x) \, \geq \, 0. 
              \end{align}    
    \end{itemize}
\end{itemize}
An interesting observation about the seminorm is that it is not necessarily point separating, i.e., vectors that are not zero can have zero norms. In topology, these properties are determined by the \textit{separation axioms}. Seminorms are essentially Minkowski functionals on convex, balanced, and absorbing sets. \par

In quantum mechanics, states are vectors in \textit{normed spaces}. A normed space is a vector space $\bm{V}$ with a positive definite or point-separating seminorm. In other words, along with the above three, the norm must also have the following property.
\begin{itemize}
    \item[(iii)] $\bm{f}(x) = 0 \implies x = 0, \ \forall \ x \in \bm{V}$ or \textbf{only zero vector has zero norm}.
\end{itemize}

\chapter{The $c^{\, (n_\pm)}$ matrices}
\section{Calculating $c_{\beta \alpha}^{(n_{-})}$ \& $c_{\beta \alpha}^{(n_{+})}$}
\label{Calculation for c_alphabeta}

In Chapter 2, we introduced a method to calculate the $c$ matrices that elucidate the relation between $\eta^{-1}$, and the eigensubspaces that it connects. Here, we calculate $c^{\, (n_\pm)}$ matrices in the same way. \par 

For $c^{\, (n_-)}$ we first act $H$ algebraically on $\eta^{-1}|\phi_{n_-^{\prime}}, a\rangle$ 
\begin{align}
    \label{H acting on eta-inverse phi_n-}
    &\nonumber\\
    H \eta^{-1}|\phi_{n_{-}^{\prime}}, \alpha\rangle \, = \, \eta^{-1} H^{\dagger}|\phi_{n_{-}^{\prime}}, \alpha\rangle \, = \, (E_{n_+^{\prime}}^{\, *})^{*} (\eta^{-1} |\phi_{n_{-}^{\prime}}, \alpha\rangle).\\
    &\nonumber
\end{align}
Using the Hamiltonian in Eq.(\ref{Hamiltonian with 3 diff eigenspaces}) on $eta^{-1}|\phi_{n_{-}^{\prime}}, \alpha\rangle$ and equating it to Eq.(\ref{H acting on eta-inverse phi_n-}) we get
\begin{align}
    \label{Equating Summation form H and algebraic H 2}
    &\nonumber\\
    \sum_{n_0} \sum_{b \, = \, 1}^{d_{n_0}} E_{n_0}\left|\psi_{n_0}, b \right\rangle\left\langle \phi_{n_0}, b\right| \eta^{-1}|\phi_{n_{-}^{\prime}}, \alpha\rangle + \sum_{n_{+}} \sum_{\beta \, = \, 1}^{d_{n_{+}}} E_{n_{+}}\left|\psi_{n_{+}}, \beta \right\rangle\left\langle \phi_{n_{+}}, \beta\right|\eta^{-1} \nonumber\\
    |\phi_{n_{-}^{\prime}}, \alpha\rangle+E_{n_{+}}^{\, *}\left|\psi_{n_1}, \beta \right\rangle\left\langle \phi_{n_{-}}, \beta\right| \eta^{-1}|\phi_{n_{-}^{\prime}}, \alpha\rangle \,= \, E_{n_{+}^{\prime}}(\eta^{-1}|\phi_{n_{-}^{\prime}}, \alpha\rangle).\\
    &\nonumber
\end{align}
Here, we see that $E_{n_0}$'s, $E_{n_{+}}^{\, *}$'s, and $E_{n_+}$'s that are not equal to $E_{n_{+}^{\prime}}$ must be zero. The sum on the left-hand side that survives corresponds to the subspace with eigenvalue $E_{n_{+}^{\prime}}$ and multiplicity $d_{n_{-}^{\prime}}$. And we have
\begin{align}
    \label{Final expression after equating summation H and algebraic H 2}
    &\nonumber\\
    \sum_{\beta \, = \, 1}^{d_{n_{-}^{\prime}}} E_{n_{+}^{\prime}}(\langle\phi_{n_{+}^{\prime}}, \beta|\eta^{-1}| \phi_{n_{-}^{\prime}}, \alpha\rangle)|\psi_{n_{+}^{\prime}}, \beta\rangle \,= \, E_{n_{+}^{\prime}}(\eta^{-1}|\phi_{n_{-}^{\prime}}, \alpha\rangle).\\
    &\nonumber
\end{align}
Cancelling $E_{n_{+}^{\prime}}$ on both sides, removing the primes and exchanging the dummy variables $a$ and $b$, we get the final expression.
\begin{align}
    \label{c_ab expression for E_n-}
    &\nonumber\\
    \eta^{-1}|\phi_{n_{-}}, \alpha\rangle \, = \, \sum_{\beta=1}^{d_{n_{+}}} c_{\beta \alpha}^{(n_{-})}|\psi_{n_{+}}, \beta\rangle, \quad c_{\alpha \beta}^{(n_{-})} \, := \,\langle\phi_{n_{+}}, \alpha|\eta^{-1}| \phi_{n_{-}}, \beta\rangle.\\
    &\nonumber
\end{align}
Next, we find the $c$ matrix expression for the eigenvectors $\eta^{-1}|\phi_{n_-^{\prime}}, a\rangle$. Applying $H$ algebraically, we get
\begin{align}
    \label{H acting on eta-inverse phi_n+}
    &\nonumber\\
    H \eta^{-1}|\phi_{n_{+}^{\prime}}, \alpha\rangle \, = \, \eta^{-1} H^{\dagger}|\phi_{n_{+}^{\prime}}, \alpha\rangle \, = \, (E_{n_+^{\prime}})^{*} (\eta^{-1} |\phi_{n_{+}^{\prime}}, \alpha\rangle).\\
    &\nonumber
\end{align}
Again, using the Hamiltonian in Eq.(\ref{Hamiltonian with 3 diff eigenspaces}) on $eta^{-1}|\phi_{n_{+}^{\prime}}, \alpha\rangle$ and equating it to Eq.(\ref{H acting on eta-inverse phi_n+}) we get
\begin{align}
    \label{Equating Summation form H and algebraic H 3}
    &\nonumber\\
    \sum_{n_0} \sum_{b \, = \, 1}^{d_{n_0}} E_{n_0}\left|\psi_{n_0}, b \right\rangle\left\langle \phi_{n_0}, b\right| \eta^{-1}|\phi_{n_{+}^{\prime}}, \alpha\rangle + \sum_{n_{+}} \sum_{\beta \, = \, 1}^{d_{n_{+}}} E_{n_{+}}\left|\psi_{n_{+}}, \beta \right\rangle\left\langle \phi_{n_{+}}, \beta\right|\eta^{-1} \nonumber\\
    |\phi_{n_{+}^{\prime}}, \alpha\rangle+E_{n_{+}}^{\, *}\left|\psi_{n_1}, \beta \right\rangle\left\langle \phi_{n_{-}}, \beta\right| \eta^{-1}|\phi_{n_{+}^{\prime}}, \alpha\rangle \,= \, E_{n_{+}^{\prime}}^{\, *}(\eta^{-1}|\phi_{n_{+}^{\prime}}, \alpha\rangle).\\
    &\nonumber
\end{align}
Thus, $E_{n_0}$'s, $E_{n_+}$'s, and $E_{n_{+}}^{\, *}$'s that are not equal to $E_{n_{+}^{\prime}}^{\, *}$ must be zero. The sum on the left-hand side that survives corresponds to the subspace with eigenvalue $E_{n_{+}^{\prime}}^{\, *}$ and multiplicity $d_{n_{+}^{\prime}}$. And we have
\begin{align}
    \label{Final expression after equating summation H and algebraic H 3}
    &\nonumber\\
    \sum_{\beta \, = \, 1}^{d_{n_{+}^{\prime}}} E_{n_{+}^{\prime}}^{\, *}(\langle\phi_{n_{-}^{\prime}}, \beta|\eta^{-1}| \phi_{n_{+}^{\prime}}, \alpha\rangle)|\psi_{n_{-}^{\prime}}, \beta\rangle \,= \, E_{n_{+}^{\prime}}^{\, *}(\eta^{-1}|\phi_{n_{+}^{\prime}}, \alpha\rangle).\\
    &\nonumber
\end{align}
And again, we cancel $E_{n_{+}^{\prime}}^{\, *}$ on both sides, remove all primes, and exchange the dummy variables $a$ and $b$ to get the expression
\begin{align}
    \label{c_ab expression for E_n+}
    &\nonumber\\
    \eta^{-1}|\phi_{n_{+}}, \alpha\rangle \, = \, \sum_{\beta=1}^{d_{n_{+}}} c_{\beta \alpha}^{(n_{+})}|\psi_{n_{-}}, \beta\rangle, \quad c_{\alpha \beta}^{(n_{+})} \, := \, \langle\phi_{n_{-}}, \alpha|\eta^{-1}| \phi_{n_{+}}, \beta\rangle.\\
    &\nonumber
\end{align}
The coefficients in Eq.(\ref{c_ab expression for E_n0}), Eq.(\ref{c_ab expression for E_n-}), and Eq.(\ref{c_ab expression for E_n+}) describe the effect of $\eta^{-1}$ on the Hilbert space. Essentially, it forms a relationship between the $|\phi\rangle$'s, and the $|\psi\rangle$'s that is mediated by $\eta$ which in turn describes the pseudo-Hermiticity of the system. 

\chapter{Miscellaneous}
\section{Mixed inner product of basis vectors in Biorthogonal Quantum Mechanics}
\label{Mixed inner product of basis vectors in Biorthogonal Quantum Mechanics}

The result we want to show is the following.
\begin{align}
    \label{Mixed inner product in BQM}
    &\nonumber\\
    \left\langle\chi_n | \phi_m\right\rangle \, = \, \delta_{n m}\left\langle\chi_n | \phi_n\right\rangle.\\
    &\nonumber
\end{align}
Here, we assume that we have nondegenerate eigenvalues for each of the sets $\{ |\phi_n\rangle \}$ and $\{ |\phi_n\rangle \}$ separately. Then using Eq.(\ref{Eigenstates of K}) and Eq.(\ref{Eigenstates of K-dagger}) we have 
\begin{align}
    \label{Mixed basis vectors on K in BQM}
    &\nonumber\\
    \langle\chi_m|K| \phi_n\rangle \, = \, \kappa_n \, \langle\chi_m | \phi_n\rangle \, = \, \nu_m^{\, *} \, \langle\chi_m | \phi_n\rangle \quad \text{or,} \quad (\kappa_n - \nu_m^{\, *}) \langle\chi_m | \phi_n\rangle = 0.\\
    &\nonumber
\end{align}
Therefore, $\langle\chi_m | \phi_n\rangle = 0$ if $\kappa_n \neq \nu_m^{\, *}$ while $\langle\chi_m | \phi_n\rangle \neq 0$ if $\kappa_n = \nu_m^{\, *}$. We have arbitrarily chosen both $\phi_n\rangle$ and $\chi_n\rangle$, which means that all the mixed inner products cannot vanish (this would be trivial!). Hence, there is at least one such $\kappa_n = \nu_m^{\, *}$ and it must be the only one because we assume that the sets of eigenvalues ${\kappa_n}$ and $\{\nu_n\}$ are not degenerate. Label shuffling can be done to make such nonzero inner product states have the subscript $n$. Thus we get $\left\langle\chi_m | \phi_n\right\rangle = 0$ if $m \neq n$ and $\left\langle\chi_m | \phi_n\right\rangle \neq 0$ if $m=n$ which is exactly Eq.(\ref{Mixed inner product in BQM}) with the subscripts exchanged. 

\section{Bounded Operator}
\label{Boundedness of a linear operator}

Let $V$ and $W$ be vector spaces. Then a bounded linear operator is a linear transformation $T: V \rightarrow W$ that maps bounded subsets of $V$ to bounded subsets of $W$. Then $T$ is bounded iff $\exists$ some $M > 0$ such that $\forall \ \alpha \in V$,
\begin{align}
    \label{Bounded Operator}
    &\nonumber\\
    ||T \, \alpha||_{\, V} \, \leq \, M \, ||\alpha||_{\, W},\\
    &\nonumber
\end{align}
where the subscripts represent the norms in the respective vector spaces. The smallest $M$ for which Eq.(\ref{Bounded Operator}) becomes the operator norm of $T$ and is denoted by $||T||$.

\end{document}
