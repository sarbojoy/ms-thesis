

\thispagestyle{plain}
\addcontentsline{toc}{chapter}{Abstract}

\vspace*{1.5cm}

\begin{center}
    \textbf{\huge{Abstract}}
\end{center}
\vspace{15mm}
The development of quantum mechanics has been phenomenal since the beginning of the 20th century. But over time physicists realised that the standard Hermitian framework originally fomalised by Paul Dirac is inadequate and not inclusive toward non-Hermitian operators. Although it has been shown, through rigorous mathematical and well-founded physical principles, that non-Hermitian operators often have a real spectrum. \par

This dissertation tries to outline the major developments in the field of non-Hermitian quantun mechanics. In \textbf{Chapter 1} we explore $\mathcal{PT}$-symmetry, which was initially successful in explaining and justifying the consequences of non-Hermitian Hamiltonians that are invariant under the application of parity-time operator. Later in the chapter, a symmetry operator $\mathcal{C}$, similar to the charge conjugation operator in particle physics, is introduced to complete the theory of $\mathcal{PT}$-symmetric quantum mechanics. \par

\textbf{Chapter 2} introduces pseudo-Hermiticity, which subsumes $\mathcal{PT}$-symmetric theory, and establishes a more general framework. Pseudo-Hermitian quantum mechanics is the natural approach in understanding non-Hermitian systems. We briefly touch upon quasi-Hermitian operator theory, an important early development in the field. And finally end with our conclusions and remarks.  
